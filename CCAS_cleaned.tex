\newcommand{\captionfonts}{\tiny}
\makeatletter
\long\def\@makecaption#1#2{%
\vskip\abovecaptionskip
\sbox\@tempboxa{{\captionfonts #1: #2}}%
\ifdim \wd\@tempboxa >\hsize
{\captionfonts #1: #2\par}
\else
\hbox to\hsize{\hfil\box\@tempboxa\hfil}%
\fi
\vskip\belowcaptionskip}
\makeatother
\parindent 0mm
\parskip 3mm
\topmargin -15mm
\input{epsf}
\oddsidemargin 15mm
\textheight 237mm
\textwidth 145mm
\headsep .35in
\fboxrule 0.2mm
\fboxsep 6mm
\makeatletter
\long\def\@makecaption#1#2{%
\vskip\abovecaptionskip
\sbox\@tempboxa{{\captionfonts #1: #2}}%
\ifdim \wd\@tempboxa >\hsize
{\captionfonts #1: #2\par}
\else
\hbox to\hsize{\hfil\box\@tempboxa\hfil}%
\fi
\vskip\belowcaptionskip}
\makeatother
\parindent 0mm
\parskip 3mm
\topmargin -15mm
\input{epsf}
\oddsidemargin 15mm
%\textheight 237mm
%\textwidth 145mm
\headsep .35in
\fboxrule 0.2mm
\fboxsep 6mm
\documentclass[12pt,a4paper]{article}
\usepackage[utf8]{inputenc} 
\usepackage[english,icelandic]{babel}
\usepackage{xlop}
\usepackage{scalerel}
\usepackage{stackengine}
\usepackage{xcolor}
\usepackage[T1]{fontenc}
\usepackage{cmap}
\usepackage{hyperref}
\usepackage {verbatim}
\usepackage{listings}




%colord background
\usepackage{framed}
\usepackage{lipsum}
\usepackage{color}
%[dvipsnames]


\usepackage{amsmath}
\usepackage{amsfonts}
\usepackage{amssymb}
%\usepackage{amsthm} Tekinn út vegna ntheorem
\usepackage{pifont}
\usepackage{mdwlist,comment} 
\usepackage{mathrsfs,amsmath}
\usepackage{amsmath,accents}
% % % %nýir pakkar
\usepackage{mathptm}
\usepackage{a4,graphics,amsmath,amsfonts,amsbsy}
\usepackage[T1]{fontenc}
\usepackage{cmap}
\usepackage[numbers,sort&compress]{natbib}
\usepackage{float, rotating, subfigure}
\usepackage[font=scriptsize]{caption}
\usepackage{longtable}
\usepackage{polynom}
\usepackage{pictex}
\usepackage{wrapfig}
\usepackage{setspace}
\usepackage{tikz}

\usepackage{graphics}
\usepackage{graphicx}
\usepackage{pgf}
\usepackage{pstricks}
\usepackage{pstricks-add}
\usepackage{etex}
\input{xy}
\xyoption{all}


\usetikzlibrary{calc,arrows}
\usepackage{blindtext}
\usepackage{tkz-euclide}
\usepackage{yhmath}
\usepackage{pgfplots}

% Miscellaneous packages

\usepackage{listings}
\lstset{language=R, basicstyle=\ttfamily, breaklines=true, columns=fullflexible}
\usepackage{multicol}
\usepackage{multirow}

%Fleiri symbols
\usepackage{latexsym}

%Enumerate environment
\usepackage{enumerate}


\newcommand{\tikzmark}[1]{%
  \tikz[overlay,remember picture] \node (#1) {};}


\setlength{\parindent}{0cm}
\newcommand{\interior}[1]{\accentset{\smash{\raisebox{-0.12ex}{$\scriptstyle\circ$}}}{#1}\rule{0pt}{2.3ex}}
\newcommand\showdiv[1]{\overline{\smash{\hstretch{.5}{)}\mkern-3.2mu\hstretch{.5}{)}}#1}}
\newcommand\ph[1]{\textcolor{white}{#1}}
\newcommand{\C}{{\mathbb  C}}
\newcommand{\N}{{\mathbb  N}}
\newcommand{\Q}{{\mathbb  Q}}
\newcommand{\R}{{\mathbb  R}}
\newcommand{\T}{{\mathbb  T}}
\newcommand{\Z}{{\mathbb  Z}}
\newcommand{\cP}{{\cal P}}
%\newcommand{\E}{\mathbf E}
\newcommand{\E}{\mathbb E} %GS - note the remaining conflict between E and ev anyways
\newcommand{\M}{\rm Markov}
\newcommand{\Var}{\rm Var}
\newcommand{\nl}{\smallskip\noindent}

\newcommand{\Poi}{\rm Poi}
\newcommand{\Exp}{\rm Exp}
\newcommand{\Bin}{\rm Bin}
\newcommand{\Hyp}{\rm Hyp}
\newcommand{\Cauchy}{\rm Cauchy}
\newcommand{\U}{\rm U}
\newcommand{\G}{\rm Gamma}
\newcommand{\s}{\sigma}
\newcommand{\mean}{\bar{X}_n}
\newcommand{\dif}{\,d}
\newcommand{\dx}{\,dx}
\newcommand{\dy}{\, dy}
\newcommand{\dz}{\,dz}
\newcommand{\dt}{\,dt}
\newcommand{\dw}{\,dw}
\newcommand{\du}{\,du}

\renewcommand{\d}[2]{\frac{\dif{#1}}{\dif{#2}}}
\newcommand{\dd}[3]{\frac{\dif^{#1}{#2}}{\dif^{#1}{#3}}}
\newcommand{\p}[2]{\frac{\partial{#1}}{\partial {#2}}}
\newcommand{\pp}[3]{\frac{\partial^{#1}{#2}}{\partial{#3}^{#1}}}

\newcommand{\svi}[1]{\left(#1 \right)}
\newcommand{\psvi}[1]{\left[#1 \right]}
\newcommand{\ssvi}[1]{\biggl|#1 \biggr|}
\newcommand{\mork}[1]{\biggl[ #1 \biggr]}

\newcommand{\smat}[1]{\left( \begin{smallmatrix} #1 \end{smallmatrix} \right)}
\newcommand{\bsmat}[1]{\left[ \begin{smallmatrix} #1 \end{smallmatrix} \right]}
\newcommand{\pmat}[1]{\begin{pmatrix} #1 \end{pmatrix}}
\newcommand{\bmat}[1]{\begin{bmatrix} #1 \end{bmatrix}}
\newcommand{\vmat}[1]{\begin{vmatrix} #1 \end{vmatrix}}
\newcommand{\mat}[1]{\begin{matrix} #1 \end{matrix}}

\newcommand{\call}[1]{{\cal #1}}

\newcommand{\prob}[1]{{\mathbb P}\hspace*{-2.9pt}\left(\hspace*{-1pt} #1 \hspace*{-1.5pt}\right)}
\newcommand{\ev}[1]{{\mathbb E}\hspace*{-2.75pt}\left[\hspace*{-0.5pt} #1 \hspace*{-1.5pt}\right]}
\newcommand{\var}[1]{{\rm Var}\hspace*{-2.75pt}\left[\hspace*{-0.5pt} #1 \hspace*{-1.5pt}\right]}
%\newenvironment{regla}
%    {\begin{center} \it

%    }
%    { 
%	\end{center}
%    }
%\newenvironment{skilgr}
%    {\begin{center} \it

%   }
%    { 
%	\end{center}
%    }
\newenvironment{adferd}
    { \it

    }
    { 

    }
\newenvironment{sonnun}
    {\begin{center} \it

    }
    { 
	\end{center}
    }

\usepackage{etex}
\usepackage{pstricks}
\usepackage[framed,amsmath,amsthm,thmmarks]{ntheorem}
\shadecolor{red!80!green!20!blue!20!} 

%new environ stuff 
\makeatletter
%\newtheoremstyle{properbreak}%
%  {\item[\hskip\labelsep \theorem@headerfont
%          ##1\ ##2\theorem@separator]\mbox{}\\*}%
%  {\item[\hskip\labelsep \theorem@headerfont
%          ##1\ ##2\ (##3)\theorem@separator]\mbox{}\\*}
\newtheoremstyle{regla}%
{\item[\hskip\labelsep \theorem@headerfont ##1\ ##2\theorem@separator]}
{\item[\hskip\labelsep \theorem@headerfont ##1\ ##2\ (##3)\theorem@separator]}
\newtheoremstyle{defn}%
{\item[\hskip\labelsep \theorem@headerfont ##1\ ##2\theorem@separator]}
{\item[\hskip\labelsep \theorem@headerfont ##1\ ##2\ (##3)\theorem@separator]}
\makeatother

\theoremstyle{regla}
\theoremheaderfont{\normalfont\bfseries}
\theorembodyfont{\upshape}
%for shading
\def\theoremframecommand{% 
\psshadowbox[fillstyle=solid,fillcolor=red!10!green!20!blue!40!,linecolor=black]}
%define thm
\newshadedtheorem{thm}{Theorem}[section]

\def\theoremframecommand{% 
\psshadowbox[fillstyle=solid,fillcolor=red!80!green!20!blue!20!,linecolor=black]}
\newshadedtheorem{corr}{Corollary}[section]

%remarks
\theoremstyle{remark}
\newtheorem{notes}{Note}[section]
%\def\theoremframecommand{% 
%\psshadowbox[linecolor=red!10!green!20!blue!40!]}
%\newshadedtheorem{sonnun}{Sönnun}

\theoremstyle{definition}
\theoremheaderfont{\normalfont\bfseries}
\theorembodyfont{\upshape}
\def\theoremframecommand{% 
\psshadowbox[fillstyle=solid,fillcolor=red!20!green!80!blue!20!,linecolor=black]}
\newshadedtheorem{defn}{Definition}[section]

\def\theoremframecommand{% 
\psshadowbox[linecolor=red!90]} %fillstyle=solid,fillcolor=red!20,
\newshadedtheorem{xmpl}{Example}[section]

\def\theoremframecommand{%
\psshadowbox [linecolor=blue]}  %fillstyle=solid,fillcolor=blue,
\newshadedtheorem{asgn}{Assignment}[section]

\theoremstyle{nonumberbreak}
\theoremheaderfont{\normalfont\bfseries}
\theorembodyfont{\normalfont}
\def\theoremframecommand{% 
\psshadowbox[linecolor=red!90]}
\newshadedtheorem{sol}{Solution:}



\voffset=-1.0in
\hoffset=-0.3in
\textwidth=6in
\textheight=10.2in

\begin{document}

\DeclareGraphicsExtensions{.eps, .jpg}
%% Tutorial https://tutor-web.net/math/math612.0
\title{math612.0
A1: From numbers through algebra to calculus and linear algebra
}
\author{{Gunnar Stefansson (editor) with contributions from very many students}}
\maketitle
{\small{\bf Copyright}
This work is licensed under the Creative Commons Attribution-ShareAlike License. To view a copy of this license, visit http://creativecommons.org/licenses/by-sa/1.0/ or send a letter to Creative Commons, 559 Nathan Abbott Way, Stanford, California 94305, USA.
}



\vspace{12pt}
{\small{\bf Acknowledgements}
 
This project has received direct funding from the EU H2020 project Minouw, to provide technical support for students who take tutorials on the EAFM in general and discard models in particular.


Most of the content has been developed as a part of giving courses at the University of Iceland and at GRÓ-FTP, with additions and developments in 2019-2021 funded in part by FarFish.




MareFrame is a EC-funded RTD project which seeks to remove the barriers preventing more widespread use of the ecosystem-based approach to fisheries management.
\\ {\tt http://mareframe-fp7.org}



This project has received funding from the European Union's Seventh Framework Programme for research, technological development and demonstration under grant agreement no.613571.
\\ {\tt http://mareframe-fp7.org}



The University of Iceland uses the tutor-web in many courses and funds content-development as a part of this use. 

The University of Iceland Research Fund has funded many of the studies developing algorithms uses in tutor-web.
\\ {\tt http://www.hi.is/}



This project has received funding from the European Commission’s Horizon 2020 Research and Innovation Programme under Grant Agreement No. 634495 for the project Science, Technology, and Society Initiative to minimize Unwanted Catches in European Fisheries (MINOUW).
\\ {\tt http://minouw-project.eu/}




This project has received funding from the European Union's Horizon 2020 research and innovation programme under grant agreement no. 727891.
\\ {\tt www.farfish.eu}
}


\newpage
\tableofcontents
\newpage
%% Lecture https://tutor-web.net/math/math612.0/lecture110
\section{Numbers, arithmetic and basic algebra}
%% Slide https://tutor-web.net/math/math612.0/lecture110/slide10
\subsection{Natural Numbers}
\begin{fbox}
\begin{minipage}{0.97\textwidth}
The positive integers are called natural numbers.\\

These numbers can be added, multiplied together and so forth.\\

Notation: $\mathbb{N}=\{1,2,3,4,....\}$ \\

Subtraction and division are not defined on these numbers.\\

An arbitrary element of $\mathbb{N}$ is most commonly denoted by $i,\ j,\ n$, or $m$, but any symbol can be used.
\end{minipage}
\end{fbox}
\subsubsection{Details}
\begin{defn}
The set of positive integers is usually denoted by $\mathbb{N}$, i.e.
$\mathbb{N}=\{1,2,3,4,....\}$ and is called the set of \textbf{natural numbers}. In some cases the number zero is included as a natural number, but here we will use the symbol $\mathbb{N}_0$ to denote the integers 0, 1, 2 and up.
\end{defn}

Within this set of numbers it is possible to add and multiply numbers together. Arithmetic operations are denoted by $+$ for addition and $\cdot$ (or $\times$) for multiplication. A natural number can also be raised to the power of a natural number, e.g. $3^5=3\cdot 3\cdot 3\cdot 3\cdot 3$ 
or in general $m^n=m\cdot m \cdot \ldots \cdot m$ ($n$ times).\\

When stating general properties of the natural numbers one needs to use symbols to indicate that the property holds for an arbitrary number. It is not enough to just write the property for a few numbers. For example, to declare that one can interchange numbers in a sum, it is not enough to say $4+3=3+4$ but one must explicitly state "the addition operator has the property that any two natural numbers, $n,\ m\in \mathbb{N}$ satisfy $n+m=m+n$".\\


An arbitrary element of $\mathbb{N}$ is most commonly denoted by $i,\ j,\ n$,or $m$, but any symbol, $a,\ b,\ c, \ldots$, can be used.\\

Several rules of arithmetic apply (some by definition, others can be derived) such as

\begin{eqnarray*}
ab&=&ba\\
a+b&=&b+a\\
a+bc&=&a+(bc)\\
a(b+c)&=&ab+ac\\
(a+b)+c&=&a+(b+c)\\
(ab)c&=&a(bc)
\end{eqnarray*}

Subtraction and division are not generally defined. In addition, we define one integer, $n$, to the power of another, $m$, to mean $n$ multiplied by itself $m$ times: $n^m=\underbrace{n\cdot n \cdot \ldots \cdot n}{m}$. 

\begin{defn}
The power is an \textbf{operator} just like addition and multiplication, and is defined to have higher priority than the other two.
\end {defn}

\subsubsection{Examples}
\begin{xmpl} 
If we have $x=4$ and $y= 2$ and want to evaluate $$x^y+y^x$$
then we  replace the values of x and y in the expression, and evaluate it,
taking care to observe the correct order of operations:
$$4^2+2^4=16+16=32.$$
\end{xmpl} 


%% Slide https://tutor-web.net/math/math612.0/lecture110/slide20
\subsection{Starting with R}
\begin{fbox}
\begin{minipage}{0.97\textwidth}
Download R from the R website: \href{http://www.r-project.org/}{http://www.r-project.org/}

Look at on-line information on R, and take the tutor-web R tutorial: \href{http://tutor-web.net/stats/stats240.1}{http://tutor-web.net/stats/stats240.1}

Simple R commands: 

\begin{itemize}
\item Assignment: $\texttt{x<-2}$
\item Arithmetic: $\texttt{2*5+4}$
\end{itemize}
\end{minipage}
\end{fbox}
\subsubsection{Details}
To assign values to a variable in R one can use "$\texttt{<-}$"\text{ }or "$\texttt{=}$"; however, these are {\bf NOT} equivalent. Using the equals sign is confusing and therefore not recommended.
\subsubsection{Examples}
\begin{xmpl} 
Assigning values to a variable:
\begin{lstlisting} 
x<-2
y<-3
z<-x+y 
\end{lstlisting} 
\end{xmpl} 

\begin{xmpl} 

Viewing assigned values:\\
Type the name,i.e. "z", to view the assigned value. 

\begin{lstlisting} 
z
[1] 5
\end{lstlisting}

\end{xmpl} 



%% Slide https://tutor-web.net/math/math612.0/lecture110/slide30
\subsection{The Integers}
\begin{fbox}
\begin{minipage}{0.97\textwidth}
The set of positive and negative integers:

$\mathbb{Z} = \{.., .., -2, -1, 0, 1, 2, ......\}$

\end{minipage}
\end{fbox}
\subsubsection{Details}
\begin{defn}
The set of all integers is denoted by $\mathbb{Z}$, i.e.
$$\mathbb{Z} = \{.., .., -2, -1, 0, 1, 2, ......\} .$$
\end{defn}

\begin{notes}
Note that within this set it is possible to subtract as well as add and multiply. Within this set we cannot, however, in general, perform division.
\end{notes}

When preforming multiple mathematical operations within the same equation, i.e. $79 - 8\cdot 3$, there is a conventional order for which the operations must be performed.  

\begin{defn}
The conventional order of operations for equations with multiple mathematical operations is referred to as an \textbf{operator precedence}.
\end {defn}


\subsubsection{Examples}
\begin{xmpl}
To compute $79 - 8\cdot 3$ start by multiplying and then subtracting:\\
$79 - 8\cdot 3 = 79-24 = 55$
\end{xmpl} 
\begin{xmpl} 
To compute $15 - (24 + 36)$ we first note that the parentheses (brackets) imply a precedence; anything inside brackets should be evaluated first.

Thus, we first add 36 to 24 and then we subtract that from 15.\\ 
15 - (24+36) = 15 - 60 = - 45\\\

Note that the answer is a negative number. 
\end{xmpl} 
\begin{xmpl} 
Simple arithmetic in R is easily done at the command prompt.
\begin{verbatim}
79-8*3
[1] 55
15-(24+36)
[1] -45
\end{verbatim}
\end{xmpl} 


%% Slide https://tutor-web.net/math/math612.0/lecture110/slide40
\subsection{Rational numbers}
\begin{fbox}
\begin{minipage}{0.97\textwidth}
{\bf Rational numbers} are fractions denoted p/q, where p and q are integers. We can simplify fractions if the numerator and denominator contain common terms.
\end{minipage}
\end{fbox}
\subsubsection{Details}
\begin{figure}[h]
\hspace{0.5mm}
\begin{minipage}{0.48\textwidth}
\begin{picture}
1
\end{picture}


\end{minipage}
\end{figure}
\begin{defn}
{\bf Rational numbers} are fractions denoted $p/q$, where $p$ and $q$ are integers. The set of all rational numbers is usually denoted $\mathbb{Q}$.
\end{defn}

\begin{notes}
Note that every integer is a rational number (obtained by taking $q=1$). 
\end{notes}

We can simplify fractions if the numerator and denominator contain common terms.\\

When the rationals are ordered on to a line there are points missing, i.e. there are "gaps", for example there is no rational number $p/q$ such that $(p/q)^2=2$.

\subsubsection{Examples}
\begin{xmpl}
$\frac{2}{6}$=$\frac{2}{2 \cdot 3}$=$\frac{1}{3}$ 

\end{xmpl}

The rational numbers can be put in order
along a line as in the figure.

\begin{xmpl} As an elaborate example of a fraction, consider the evaluation of the quantity 
$$
\frac{\frac{2}{3}+\frac{2}{5}}{\frac{1}{3}+\frac{1}{2}}
$$
\end{xmpl}
\begin{xmpl}
Evaluate 
$$
\frac{\frac{2}{3}+\frac{2}{5}}{\frac{1}{3}+\frac{1}{2}}
$$

Solution:
We can either start by calculating the numerator

$$\frac{2}{3}+\frac{2}{5}$$

or the denominator

$$\frac{1}{3}+\frac{1}{2}.$$

Here we choose to start with the numerator. 
The first step is to make the two fractions in the numerator have a common denominator. 

We can either find the least common denominator or use the product of the two denominators. Here they are the same number, 15. 

So the first step is:

$$\frac{2}{3}\cdot \frac{5}{5}+\frac{2}{5}\cdot \frac{3}{3} = \frac{2\cdot 5}{3\cdot 5}+\frac{2\cdot 3}{5\cdot 3} = \frac{10}{15}+\frac{6}{15}.$$

Now it is possible to add the two fractions, which is the second step:

$$\frac{10+6}{15} = \frac{16}{15}$$


Next, the same process has to be performed for the original denominator. 

With the same method  (LCM - least common multiple) we get:

$$\frac{1\cdot 2}{3\cdot 2}+\frac{1\cdot 3}{2\cdot 3} = \frac{2}{6}+\frac{3}{6} =\frac{5}{6} $$


Then the total answer is:

$$\frac{\frac{16}{15}}{\frac{5}{6}} {=} \frac{16}{15} \cdot \frac{6}{5} = \frac{96}{75}= \frac{96/3}{75/3}=\frac{32}{25}$$

We can see that in the last step of the equation, the factor has been simplified. To do this we use factoring. 
Here we obtain: 

$$\frac{96}{75}$$

= $$\frac{3\cdot 32}{3\cdot 25}$$



We can now remove "3", or the multiplier, as it is on both sides of the fraction. So we have: 


$$\frac{32}{25}$$

= $$\frac{25}{25}+\frac{7}{25} 
=1\frac{7}{25} $$

In step 1 above we used Cross-Multiplication. 

\begin{defn}
{\bf Cross-Multiplication} is when we multiple the numerator by the reciprocal of the denominator. 
\end{defn}
So in this case we rewrite
$$\frac{\frac{16}{15}}{\frac{5}{6}}$$

or $$\frac{16}{15} \div \frac{5}{6}$$

as $$\frac{16}{15} \cdot \frac{6}{5}$$

As you can see all we are doing is turning $$\frac{5}{6}$$

upside down: and multiplying it with $$\frac{16}{15}$$

This gives:

$$\frac{96}{75}$$

\end{xmpl}

In some cases it is possible to draw a \textbf{square root} of a fraction $s=\frac{p}{q}$, i.e. find a number $r\in \mathbb{Q}$ such that $r^2=s$. The square root is denoted $\sqrt{r}$.

\begin{xmpl}

Consider the expression $$(\sqrt{\frac{1}{9}} \times 2^4) + (\frac{1}{5} \times \sqrt{25}).$$

To evaluate this expression, first consider separately the two parts on each side of the plus symbol. 

The first part is $$(\sqrt{\frac{1}{9}} \times 2^4)$$

and the second part is $$(\frac{1}{5} \times \sqrt{25}).$$

 

In addition, by definition of root, $$\sqrt{\frac{1}{9}} = \frac{1}{3}.$$

 

First part: $$(\sqrt{\frac{1}{9}} \times 2^4) = \frac{1}{3} \times 16 = \frac{16}{3}$$

Second part: $$(\frac{1}{5} \times \sqrt{25}) = \frac{1}{5} \times 5 = 1$$

Finally, add the first part and the second part: $$\frac{16}{3} + 1 = \frac{19}{3}$$
\end{xmpl}

\begin{xmpl}

Consider the following fraction example, to be solved step by step: 

$$\frac{\frac{4}{2}+(\frac{1}{4}\cdot\frac{5}{3})}{\frac{2}{6}\div\frac{1}{5}}$$

First we need to be aware of operator precedence, sometimes called BODMAS (brackets, multiplication/division, then addition/subtraction). 

$$(\frac{1}{4}\cdot\frac{5}{3}) = \frac{5}{12}$$

After solving the bracket we can proceed with adding $$\frac{4}{2}$$

to $$\frac{5}{12}$$

as there is no other action left for the nominator of the main fraction. So:

$$\frac{4}{2}+\frac{5}{12}$$

When adding fractions together we first have to find a common denominator, in this case 12 would work as $$2\cdot6=12$$

So we multiply both the numerator and the denominator of that fraction by 6 and then add the two numerators of the fractions together, keeping the same denominator. 
 
$$\frac{4}{2}+\frac{5}{12}=\frac{4\cdot6}{2\cdot6}+\frac{5}{12}=\frac{24}{12}+\frac{5}{12}=\frac{29}{12}$$

Now we have the top half of the fraction solved. We then proceed with dividing the two fractions of the bottom half. When dividing fractions we use the so called cross multiplication technique. This arithmetic trick is derived from the fact that if you divide a fraction by its duplicate you get 1. If you multiple a fraction by its reciprocal (it's reverse) you also get 1. Like so:

$$\frac{1}{2}\div\frac{1}{2}=1$$

and $$\frac{1}{2}\cdot\frac{2}{1}=1$$

These functions always provide the same result and therefore we can turn the fraction we are dividing by upside down and multiply it to the other fraction as that is usually much easier.\\

We can therefore rewrite $$\frac{2}{6}\div\frac{1}{5}$$

as $$\frac{2}{6}\cdot\frac{5}{1}=\frac{10}{6}$$

We've now solved both halves of the original fraction and can therefore proceed to solve it, again with the cross multiplication technique as fractions are after all just divisions:

$$\frac{29}{12}\div\frac{10}{6}=\frac{29}{12}\cdot\frac{6}{10}=\frac{174}{120}$$

Now $$\frac{174}{120}$$

is a pretty bad looking fraction and we'd preferably like to simplify it.\\ 

To do this we use factoring. 
\begin{defn}
{\bf Factoring} essentially means to break a number done into it's smallest factors or multipliable prime numbers. 
\end{defn}

In this case we get 
$$\frac{2\cdot3\cdot29}{2\cdot3\cdot20}$$

These are the smallest prime numbers that can multiply together into 174 and 120 respectively. \\

A way of doing this in your head is by first dividing both numbers (174,120) by two. Which gives us: 

$$\frac{2\cdot87}{2\cdot60}$$

and then dividing those numbers (87,60) by 3, since they can't be divided by 2. Dividing by 3 gives you $$\frac{3\cdot29}{3\cdot20}=\frac{29}{20}$$

which is a lot nicer than $$\frac{174}{120}$$

The reasoning behind this factoring simplification is that we can remove multipliers if they are on both sides of a fraction. This is because the result of a fraction where the numerator and the denominator are the same is always 1. Like so: 

$$\frac{1}{1}=1$$

or $$\frac{2}{2}=1$$

or $$\frac{3}{3}=1$$

The final answer therefore is

$$\frac{\frac{4}{2}+(\frac{1}{4}\cdot\frac{5}{3})}{\frac{2}{6}\div\frac{1}{5}}=\frac{29}{20}$$
\end{xmpl}


%% Slide https://tutor-web.net/math/math612.0/lecture110/slide50
\subsection{The real line}
\begin{fbox}
\begin{minipage}{0.58\textwidth}
Some obvious numbers are not fractions.

The set of numbers making up the real line is denoted by the symbol $\mathbb{R}$.
\end{minipage}
\hspace{0.5mm}
\begin{minipage}{0.38\textwidth}
\begin{picture}
2
\end{picture}

Figure:  The diagonal of a rectangle with unit side lengths of $\sqrt{2}$, Note that  $\sqrt{2}$ is not a fraction.
\end{minipage}
\end{fbox}
\subsubsection{Details}
Some obvious numbers, which commonly occur, are not fractions. These are in between the rational numbers (fractions).  Filling in the missing points to obtain a continuum results in the set of "real numbers".\\

Denoted by $\mathbb{R}$ the entire set of "real numbers"  which corresponds to "filling in"  the "missing pieces"  of the line. 

\subsubsection{Examples}

\begin{xmpl}
If $C$ is the circumference of a circle and $D$ is the diameter and we define BIL$\pi=\frac{C}{D}BIL$ then $\pi$ is not a fraction.
\end{xmpl}


\begin{xmpl}
One example of a non fraction is the number e (Euler's number) which can be defined by   

$$e = \sum_{n=0}^{\infty }\frac{1}{n!}$$
\end{xmpl}
\begin{xmpl}

If you have a right triangle with unit side length, what is the length of its hypotenuse and what class of numbers does it belong to?

An isosceles triangle is defined as having adjacent and opposite sides of same length, connected by a 
$90^{\circ}$ angle.
Unit side length of these, refers to a side length of $$1.$$
 
As we have a $90^{\circ}$ angle, we can use Pythagoras' theorem:

$$a^2+b^2=c^2$$

With
$$a=adjacent$$

$$b=opposite$$

$$c=hypotenuse$$

So with $$a,b=1:$$

$$c^2=1^2+1^2$$

$$c^2=1+1$$

$$c^2=2$$

We take the square root to get $$c$$

$$c=\sqrt{2}$$

Now that we answered the first part of the question, it needs to be defined, which class of number $$\sqrt{2}$$

belongs to. $$\sqrt{2}$$

is an irrational number, and belongs thereby to the set of real numbers $$\mathbb{R}$$

Real numbers can be imagined as points on an infinitely long line, which is also called the real line.
\end{xmpl}


{\bf Copyright}
2021, Gunnar Stefansson (editor) with contributions from very many students

This work is licensed under the Creative Commons
Attribution-ShareAlike License. To view a copy of this license, visit
http://creativecommons.org/licenses/by-sa/1.0/ or send a letter to
Creative Commons, 559 Nathan Abbott Way, Stanford, California 94305,
USA.
\clearpage
%% Lecture https://tutor-web.net/math/math612.0/lecture120
\section{Data vectors}
%% Slide https://tutor-web.net/math/math612.0/lecture120/slide10
\subsection{The plane}
\begin{fbox}
\begin{minipage}{0.97\textwidth}
Pairs of numbers can be depicted as points on a plane.

The plane is normally denoted by $\mathbb{R}^2$.
\end{minipage}
\end{fbox}
\subsubsection{Details}
Pairs of numbers can be depicted as points on a plane.\\

\begin{defn}
A {\bf plane} is a perfectly flat surface with no thickness and no end, it can extend forever in all directions. It has two-dimensions, length and width. We need two values to find a point on the plane. 
\end{defn}

Normally we talk about "the plane" as the collection of all pairs of numbers and denoted it by 
$$
\mathbb{R}^2 = \{ (x,y) : x,y \in \mathbb{R} \},
$$ 
giving coordinates to each point.


\subsubsection{Examples}
\begin{xmpl}
Plotting the point (2,4) in the x-y plane using R.
\begin{lstlisting}
plot(2,4,xlim=c(0,6),ylim=c(0,6),xlab="x",ylab="y",cex=2)
text(2,4,"(2,4)",pos=4,cex=2)
\end{lstlisting}

Additional points can be added using the $\texttt points$ function:

\begin{lstlisting}
points(3,5, cex = 0.5)      ## a point at (3,5)
\end{lstlisting}

\end{xmpl}

If you have 2 sets of coordinates on a plane you can calculate the distance between the 2 points and graph the line connecting the points

\begin{xmpl}
What is the distance between the 2 points (3,9) and (5,1)?

We will use the Pythagorean theorem:

$$d = \sqrt{(x_{2}-x_{1})^{2}+(y_{2}-y_{1})^{2}} $$

We insert our values into the formula:
$$d=\sqrt{(5-3)^{2}+(1-9)^{2}} $$

 

When we combine inside the parenthesis we get:
$$d=\sqrt{(2)^{2}+(-8)^{2}}$$

Squaring both terms:
$$d=\sqrt{4+64}$$

Then we take the square root:
$$d=\sqrt{68}$$

The result:
$$d=8.2462$$
\end{xmpl}


%% Slide https://tutor-web.net/math/math612.0/lecture120/slide20
\subsection{Simple plots in R}
\begin{fbox}
\begin{minipage}{0.58\textwidth}
Graphing functions in R
\begin{itemize}
\item plot - plots a scatter plot (as a line plot)
\item points - adds points to a plot
\item text - adds text to a plot
\item lines - adds lines to a plot
\end{itemize}
\end{minipage}
\hspace{0.5mm}
\begin{minipage}{0.38\textwidth}
\begin{picture}
3
\end{picture}

Figure:  Points on a plane, drawn with R.
\end{minipage}
\end{fbox}
\subsubsection{Examples}
\begin{xmpl}

\begin{lstlisting}
plot(2,3)
\end{lstlisting}

gives a single plot and

\begin{lstlisting}
plot(2,3, xlim=c(0,5), ylim=c(0,5))
\end{lstlisting}

gives a single plot but forces both axes to range from 0 to 5.
\end{xmpl}
\begin{xmpl}
The following R commands can be used to generate a plot with two points:

\begin{lstlisting}
plot(1,2,xlim=c(0,5),ylim=c(0,5),xlab="x",ylab="y")
points(3,1)
text(1,2,"(1,2)",pos=4, cex=2)
text(3,1,"(3,1)",pos=4, cex=2)
\end{lstlisting}
\end{xmpl}
\begin{xmpl}


In this example, we plot 3 points. The first two
points are by including vectors with a length of 2 as the x and y
arguments of the plot function. The third plot was added with the
points function. The second and third points were labeled using the
text function and a line was drawn between them using the lines
function.
\begin{notes}
Note that if you are unsure of what format the arguments of
an R function needs to be, you can call a help file by typing "?"
before the function name (e.g. "?lines")
\end{notes}

\begin{lstlisting}
plot(c(2,3),c(3,4),xlim=c(2,6),ylim=c(1,5),xlab="x",ylab="y")
points(4,2)
text(3,4,"(3,4)",pos=4, cex=2)
text(4,2,"(4,2)",pos=4, cex=2)
lines(c(3,4), c(4,2))
\end{lstlisting}
\end{xmpl}


%% Slide https://tutor-web.net/math/math612.0/lecture120/slide30
\subsection{Data}
\begin{fbox}
\begin{minipage}{0.97\textwidth}
Data are usually a sequence of numbers, typically called a vector.



\end{minipage}
\end{fbox}
\subsubsection{Details}
When we collect data these are one or more sequences of numbers, collected into data vectors. We commonly think of these data vectors as columns in a table.

\subsubsection{Examples}
\begin{xmpl}
In R, if the command

\begin{lstlisting}
x <- c(4,5,3,7)
\end{lstlisting}

is given, then $\verb1x1$ contains  a vector of numbers.
\end{xmpl}
\begin{xmpl}
Create a function in R, give it a name "Myfunction" which takes the sum of x,y.

\begin{lstlisting}
Myfunction<- function(x,y) {
 sum(x,y)
}
\end{lstlisting}

If you input the vectors 1:3 and 4:7 into the function it will calculate the sum of BIL$\verb|x<-(1+2+3)|BIL$ and $\verb|y<-(4+5+6+7)|$ as follows
\begin{lstlisting}
> Myfunction(1:3,4:7)
28
\end{lstlisting}
\end{xmpl}

%% Slide https://tutor-web.net/math/math612.0/lecture120/slide40
\subsection{Indices for a data vector}
\begin{fbox}
\begin{minipage}{0.97\textwidth}
If data are in a vector $x$, then we use \underline{indices} to refer to individual elements.
\end{minipage}
\end{fbox}
\subsubsection{Details}
If $i$ is an integer then $x_i$ denotes the $i$'th element of $x$. \\

Note that although we do not distinguish (much) between row- and column vectors, usually a vector is thought of as a column. If we need to specify the type of vector, row or column, then for vector $x$, the column vector would be referred to as  $x'$ and the row vector as $x^T$ (the {\bf transpose} of the original).
\subsubsection{Examples}
\begin{xmpl}
If $x=(4,5,3,7)$ then $x_1=4$ and $x_4=7$

\end{xmpl}
\begin{xmpl}
How to remove all indices below a certain value in R
\begin{lstlisting}
x <- c(1,5,8,9,4,16,12,7,11)
x
[1]  1  5  8  9  4 16 12  7 11
y <- x[x>10]
y
[1] 16 12 11
\end{lstlisting}
\end{xmpl}

\begin{xmpl}
Consider a function that takes to vectors $$a \in \mathbb{R}^n, b \in \mathbb{N}^m$$

as arguments with $$n \ge m$$

and $$1 \le b_1,...,b_m \le n.$$

The function returns the sum \begin{equation}\sum_{i = 1}^m {a_b}_i \end{equation}

Long version:\\
\texttt
fN <- function(a,b) {			\\
	result <- sum(a[b])		\\
	return(result)\\
}


Short version:\\
\texttt|fN <- function(a,b) sum(a[b])|


\end{xmpl}


%% Slide https://tutor-web.net/math/math612.0/lecture120/slide50
\subsection{Summation}
\begin{fbox}
\begin{minipage}{0.97\textwidth}
We use the symbol $\Sigma$ to denote sums.

In R, the sum function adds numbers.
\end{minipage}
\end{fbox}
\subsubsection{Examples}
\begin{xmpl}
If $x=(4,5,3,7)$

then

$$\sum_{i=1}^{4} x_i = x_1+x_2+x_3+x_4 = 4+5+3+7 = 19
$$

and

$$
\sum_{i=2}^{4} x_i = x_2+x_3+x_4 = 5+3+7 = 15 .
$$

Within R one can give the corresponding commands:
\begin{lstlisting}
 x<-c(4,5,3,7)
 x
[1] 4 5 3 7
 sum(x)
[1] 19
sum(x[2:4])
[1] 15

\end{lstlisting}
\end{xmpl}

{\bf Copyright}
2021, Gunnar Stefansson (editor) with contributions from very many students

This work is licensed under the Creative Commons
Attribution-ShareAlike License. To view a copy of this license, visit
http://creativecommons.org/licenses/by-sa/1.0/ or send a letter to
Creative Commons, 559 Nathan Abbott Way, Stanford, California 94305,
USA.
\clearpage
%% Lecture https://tutor-web.net/math/math612.0/lecture130
\section{More on algebra}
%% Slide https://tutor-web.net/math/math612.0/lecture130/slide10
\subsection{Some Squares}
\begin{fbox}
\begin{minipage}{0.97\textwidth}
If a and b are real numbers, then 

$$
(a+b)^2=a^2+2ab+b^2
$$
\end{minipage}
\end{fbox}
\subsubsection{Details}
If a, b are real numbers, then:

$(a+b)^2=a^2+2ab+b^2$

This can be proven formally with the following argument:

\begin{eqnarray*} 
(a+b)^2 &=& (a+b)(a+b)\\
       &=&( a+b)a+(a+b)b\\
       &=& a^2+ba+ba+b^2\\
       &=& a^2+2ab+b^2
\end{eqnarray*}



%% Slide https://tutor-web.net/math/math612.0/lecture130/slide20
\subsection{Pascal's Triangle}
\begin{fbox}
\begin{minipage}{0.97\textwidth}
Pascal's triangle is a geometric arrangement of the binomial coefficients in a triangle

$$
\begin{array}{ccccc}
  & & 1 & &\\
  & 1 & & 1&\\
  1 & & 2 & & 1\\
  \vdots \quad \vdots && \vdots && \vdots \quad \vdots
\end{array}
$$

\end{minipage}
\end{fbox}

\subsubsection{Details}

$$
\begin{array}{ccccccccc}
  n=0: & & & & &1& & & \\
  n=1: & & & &1& &1& & \\
  n=2: & & &1& &2& &1& \\
  n=3: & &1& &3& &3& &1
\end{array}
$$

To build Pascal's triangle, start with "1" at the top, and then continue placing numbers below it in a triangular pattern. Each number is just the two numbers above it added together (except for the edges, which are all "1").

\subsubsection{Examples}
\begin{xmpl}
The following function in R gives you the Pascal's triangle for $n= 0$ to $n=10$.

\begin{lstlisting}
fN <- function(n) formatC(n, width=2)
for (n in 0:10) {
    cat(fN(n),":", fN(choose(n, k = -2:max(3, n+2))))
    cat("\n")
}
\end{lstlisting}

\begin{verbatim}
 0 :  0  0  1  0  0  0
 1 :  0  0  1  1  0  0
 2 :  0  0  1  2  1  0  0
 3 :  0  0  1  3  3  1  0  0
 4 :  0  0  1  4  6  4  1  0  0
 5 :  0  0  1  5 10 10  5  1  0  0
 6 :  0  0  1  6 15 20 15  6  1  0  0
 7 :  0  0  1  7 21 35 35 21  7  1  0  0
 8 :  0  0  1  8 28 56 70 56 28  8  1  0  0
 9 :  0  0  1  9 36 84 126 126 84 36  9  1  0  0
10 :  0  0  1 10 45 120 210 252 210 120 45 10  1  0  0
\end{verbatim}

Changing the numbers in the line $\verb|for(n in 0:10)|$ will give different portions of the triangle.
\end{xmpl}


%% Slide https://tutor-web.net/math/math612.0/lecture130/slide30
\subsection{Factorials}
\begin{fbox}
\begin{minipage}{0.97\textwidth}
We define the factorial of an integer n as \\
$n!= n\cdot(n-1) \cdot(n-2)\cdot \ldots \cdot 3 \cdot 2 \cdot 1$

\end{minipage}
\end{fbox}
\subsubsection{Details}
\begin{defn}
We define the factorial of an integer n as
$$
n!= n\cdot(n-1) \cdot(n-2)\cdots \ldots \cdot 3 \cdot 2 \cdot 1 .
$$
\end{defn}
\subsubsection{Examples}
\begin{xmpl}

Suppose you have 6 apples, $\{a, b, c, d, e, f\}$ and you 
want to put each one into a different apple basket, $\{1,2,3,4,5,6\}$.\\

For the first basket you can choose from 6 apples $\{a, b, c, d, e,f\}$,
and for the second basket you have then 5 apples to choose from
and so it goes for the rest of the baskets, so for the last one you
only have 1 apple to choose from.\\

The end result would then be: 
$6! = 6 \cdot 5 \cdot 4 \cdot 3 \cdot 2 \cdot 1 = 720$ possible allocations.\\

This could also be calculated in R with the factorial function:
\begin{lstlisting}
factorial(6)
[1] 720
\end{lstlisting}
\end{xmpl}


%% Slide https://tutor-web.net/math/math612.0/lecture130/slide40
\subsection{Combinations}
\begin{fbox}
\begin{minipage}{0.97\textwidth}
The number of different ways one can choose a subset of size $x$ from a set of $n$ elements is determined using the following calculation:
$$
{n \choose x}= \frac{{n!}}{{x!\left( {n - x} \right)!}}
$$

\end{minipage}
\end{fbox}
\subsubsection{Details}
\begin{defn}
A {\bf combination} is an un-ordered collection of distinct elements
\end{defn}
Suppose we want to toss a coin $n$ times. In each toss we obtain head (H) or tail (T) resulting in a sequence of H,T,T,H, ... T.\\

How many of these possible sequences contain exactly $x$  tails? There are $n$ positions in the sequence, we can choose $x$ of these in $\binom{n}{x}$ ways and put our "Ts" in those positions. If the probability of landing tails is $p$ then each one of these sequences with exactly $x$ tails has probability $p^x(1-p)^{n-x}$ so the total probability of landing exactly $x$ tails in $n$ independent tosses is

$$
{n \choose x}= \frac{{n!}}{{x!\left( {n - x} \right)!}} .
$$

For convenience we define $0!$ to be 1.

\subsubsection{Examples}
\begin{xmpl}
Consider tossing a coin four times. \\

(a) How many times will this experiment result in exactly two tails?\\

There are a total of 16 possible sequences of heads and tails from four tosses. These can simply all be written down to answer a question like this.\\

We get two tails in 6 of these tosses. We can explicitly write the corresponding 
combinations of two tails as follows
\begin{verbatim}
HHTT
HTHT
HTTH
THTH
TTHH
THHT
\end{verbatim}

(b) How many times you will end up with 1 tail? The answer is 4 times and the output can be written as;
\begin{verbatim}
HHHT
HTHH
THHH
HHTH 
\end{verbatim}
The case of a single tail is easy: The single tail can come up in any one of four positions.
\end{xmpl}


%% Slide https://tutor-web.net/math/math612.0/lecture130/slide50
\subsection{The binomial theorem}
\begin{fbox}
\begin{minipage}{0.97\textwidth}
$$(a+b)^n  = \sum_{x=0}^n {n \choose x} a^xb^{n-x}$$
\end{minipage}
\end{fbox}
\subsubsection{Details}
If a and b are real numbers and n is an integer then the expression $(a+b)^n$  can be expanded as:

$ (a+b)^n = a^n+ {n \choose 1}a^{n-1}b +  {n \choose 2}a^{n-2}b^ + \ldots +{n \choose n-1}ab^{n-1}+b^n$  

$ (a+b)^n  = \sum_{i=1}^n  {n \choose x}a^xb^{n-x}$   

This can be seen by looking at $(a+b)^n$ as a product of $n$ parentheses
and multiply these by picking one item ($a$ or $b$) from each. If we picked
$a$ from BIL$xBIL$ parentheses and BIL$bBIL$ from $(n-x)$, then the product is
$a^x b^{n-x}$. We can choose the BIL$xBIL$ $a$'s in a total of $\binom{n}{x}$
ways so the coefficient of BIL$a^x b^{n-x}BIL$ is $\binom{n}{x}$.

\subsubsection{Examples}
\begin{xmpl}
Since
$$(a+b)^n  = \sum_{x=0}^n {n \choose x} a^xb^{n-x},$$
it follows that
$$2^n = (1+1)^n  = \sum_{x=0}^n {n \choose x}$$
i.e.
$$2^n = {n \choose 0} + {n \choose 1} + {n \choose 2}\ldots+ {n \choose n}$$

\end{xmpl}

{\bf Copyright}
2021, Gunnar Stefansson (editor) with contributions from very many students

This work is licensed under the Creative Commons
Attribution-ShareAlike License. To view a copy of this license, visit
http://creativecommons.org/licenses/by-sa/1.0/ or send a letter to
Creative Commons, 559 Nathan Abbott Way, Stanford, California 94305,
USA.
\clearpage
%% Lecture https://tutor-web.net/math/math612.0/lecture140
\section{Discrete random variables and the binomial distribution}
%% Slide https://tutor-web.net/math/math612.0/lecture140/slide10
\subsection{Simple probabilities}
\subsubsection{Details}
Of all the possible 3-digit strings, $\binom{3}{x}$ of them have $x$ heads. So the probability of landing $x$ heads is $\binom{3}{x}p^x(1-p)^{3-x}$.
\subsubsection{Examples}
\begin{xmpl}
Consider a biased coin which has probability $p$ of landing heads up. If we toss this coin 3 independent times the possible outcomes are:

$$
\begin{array}{c c c}
  \hline
  \text{sequence} & \text{probability}	& \text{Number of heads}\\
  \hline				
  \text{HHH} & p \cdot p \cdot p=p^3 & 3\\
  \text{HHT} & p^2(1-p) & 2\\
  \text{HTH} & p^2(1-p) & 2 \\
  \text{HTT} & p(1-p)^2 & 1\\
  \text{THH} & p^2(1-p) & 2\\
  \text{THT} & p(1-p)^2 & 1\\	
  \text{TTH} & p(1-p)^2 & 1\\
  \text{TTT} & (1-p)^3 & 0\\
  \hline
\end{array}
$$
\end{xmpl}

\begin{xmpl}
It is also possible to aggregate these values into a table and describe only the number of heads obtained:
$$
\begin{array}{c c}
  \hline
  \text{Heads} & \text{Probability}\ p(x)\\
  \hline				
  0 & (1-p)^3\\
  1 & 3p(1-p)^2\\
  2 & 3p^2(1-p)\\
  3 & p^3 \\
  \hline
\end{array}.
$$
If we are only interested in the number of heads, then
this table describes a \textbf{probability mass function} $p$, namely the probability $p(x)$ of every possible outcome $x$
of the experiment.
\end{xmpl}


\begin{xmpl}
Given that a year is 365 days and each day has the same probability of being someone's birthday. 
What's the probability of at least 2 people sharing a birthday in a group of 25 people? \\

Now, calculating each of the possible outcomes could become very tedious. That is calculating the odds that 2 people share a birthday, 3 people, 4 people, etc. So instead we try to find out the odds that no one in the group shares a birthday and subtract those odds from 1 (100\%). \\

First, let's look at the odds of only two people having distinct birthdays. 

$$\frac{365}{365}\cdot\frac{364}{365} = 0.9973$$

Person one can be born on any day and the odds of having a distinct birthday are therefore 1. The next person can be born on everyday but the 1 the other person was born on, so 364 days.\\ 

Now let's say we add the 3rd person and calculate his/her odds of having a distinct birthday. 

$$\frac{365}{365}\cdot\frac{364}{365}\cdot\frac{363}{365} = 0.9918$$

This can also be rewritten as

$$\frac{365\cdot364\cdot363}{365^3}$$

And we can do this on and on for all the 25 people we are interested in. But that may also become a bit tedious. So we use factorials instead. So instead of doing

$$\frac{365\cdot364\cdot363...\cdot341}{365^{25}}$$

we do

$$\frac{\frac{365!}{340!}}{365^{25}}=0.4313$$

Essentially the division of factorials here removes all the values < 341, leaving 340, 339, 338 ... 1

Now remember this is the probability that no one shares a birthday. So when we subtract this from 1 we get

$$1-0.4313=0.5687$$

or roughly 57\% odds of at least 2 people in a group of 25 sharing the same birthday.
\end{xmpl}


%% Slide https://tutor-web.net/math/math612.0/lecture140/slide20
\subsection{Random variables}
\begin{fbox}
\begin{minipage}{0.97\textwidth}
A random variable is a concept used to denote the outcome of an experiment before it is conducted.



\end{minipage}
\end{fbox}
\subsubsection{Examples}
 
\begin{xmpl}
Let $X$ denote the number of heads in a coin tossing experiment. We can then talk about the probabilities of certain events such as obtaining two heads, i.e. $X=2$. We write this as 

$$
P[X=2]={n \choose 2}p^2(1-p)^{n-2}
$$

In general: 
$$
P[X=x] = {n \choose x}p^x (1-p)^{n-x}
$$
where 
$x = 0,1,.....,n$

\end{xmpl} 

\subsubsection{Handout}
\begin{defn}
A {\bf random variable}, $X$, is a function defined on a sample space, with outcomes in the set of real numbers.
\end{defn} 


It is simpler to think of a random variable as a symbol used to denote the outcome of an experiment before it is conducted. 



\begin{notes}

Note that it is {\bf essential} to distinguish between upper case and lower case letters
when writing these probabilities - it makes no sense to write $P[x=x]$.
\end{notes} 
\begin{notes}

Random variables are generally denoted by upper case letters such as $X$, $Y$ and so on.
\end{notes} 
\begin{notes}
To see how a random variable is a function, it is useful to consider the actual outcomes of two coin tosses. These outcomes can be denoted $\{HH, HT, TH, TT\}$.
Now consider a random variable $X$ which describes the number of heads obtained. This random variable attributed 2 to the outcome $HH$ and 0 to $TT$, i.e.
$X$ is a function with $X(HH)=2$, $X(HT)=X(TH)=1$ and $X(TT)=0$.
\end{notes} 



%% Slide https://tutor-web.net/math/math612.0/lecture140/slide30
\subsection{Simple surveys with replacement}
\begin{fbox}
\begin{minipage}{0.97\textwidth}
If we randomly draw individuals (with replacement) and ask a question with two possible answers 
(positive or negative), then the number of positive answers will come from a binomial 
distribution.




\end{minipage}
\end{fbox}
\subsubsection{Examples}
\begin{xmpl}
Suppose we are participating in a lottery. We pick a number from a lottery bowl (a simple random sample). We can put the number aside, or we can put it back into the bowl. If we put the number back in the bowl, it may be selected more than once; if we put it aside, it can be selected only one time. 

\begin{defn}
When an element can be selected more than one time, we are sampling {\bf with replacement}. 
\end{defn}
\begin{defn}
When an element can be selected only one time, we are sampling {\bf without replacement}.
\end{defn}
\end{xmpl}

%% Slide https://tutor-web.net/math/math612.0/lecture140/slide40
\subsection{The binomial distribution}
\begin{fbox}
\begin{minipage}{0.97\textwidth}
If we toss a biased coin $n$ independent times, each with probability $p$ of landing heads up, then the probability of obtaining $x$ heads is

$$
{n \choose x}p^x (1-p)^{n-x}
$$

\end{minipage}
\end{fbox}
\subsubsection{Examples}
\begin{xmpl} 
Suppose we toss a coin, with probability $p$ of landing on heads $n$ times obtaining a sequence of Hs (when it lands heads) and Ts (when it lands tails). Any sequence, 
$$
HTH...HTHHH
$$
which has $x$ heads ($H$) and $n-x$ tails ($T$), has the probability $p^x(1-p)^{n-x}$. There are exactly $\binom{n}{x}$ such sequences, so the total probability of landing $x$ heads in $n$ tosses is
$$
\binom{n}{x}p^x(1-p)^{n-x}.
$$

\end{xmpl} 
\begin{xmpl} 


Let the probability that a certain football club wins a match be equal to 0.4.If the total number of matches played in the season is 30, what is the probability that the football club wins the match 10\% of the time?\\

We first calculate the number of times a match was played and won by multiplying the percentage of wins by the number of matches played.\\

10\% of 30 times = 3 times\\

We can now proceed to calculate the probability that they will win the match given that their probability of a winning is 0.4 if they play 3 times in a season. This can be computed as follows:\\

$$\binom{30}{3} \times (0.4)^3 \times (1-0.4)^{30-3} $$

$$=0.000265$$

\\
This can be calculated in R using the code below:
\begin{lstlisting}
dbinom(3,30,0.4)

[1] 0.0002659437
\end{lstlisting}
This is equal to the manual calculation using the binomial theorem.


\end{xmpl} 

\begin{xmpl} 
Suppose a youngster puts his shirt on by himself every day for five days. The probability that he puts it on the right way each time is $p=0.2$. We let $X$ be a random variable that describes the number of times the youngster puts his shirt on the right way. The youngster can either put the shirt on the wrong or the right way so $X$ follows the binomial distribution with the parameters $p=0.2$ (the probability of a successful trial) and $n=5$ (number of trials). We can now calculate for example the probability that the youngster will put it on the right way for at least 4 days.\\

Putting the shirt on the right way for at least 4 days means that the youngster will either put it on the right way for either four or five days (at least four or more days of five days total). We thus have to calculate the probability that the youngster will put his shirt on the right way for 4 and 5 days separately and then we add it together. We can write this process as follows:

$$P(X\geq4) = P(X=4) + P(X=5)$$

$$= \binom{5}{4}\times0.2^4\times(1-0.2)^{5-4} + \binom{5}{5}\times0.2^5\times(1-0.2)^{5-5}$$

$$= 5\times0.2^4\times0.8^1 + 1\times0.2^5\times0.8^0$$

$$= 5\times0.2^4\times0.8 + 0.2^5\times1$$

$$= 5\times0.8 \times0.2^4 + 0.2^5 $$

$$= 4\times0.2^4 + 0.2^5$$

 

$$= 4\times0.0016 + 0.00032$$

$$= 0.00672$$

The probability that the youngster will put his shirt on the right way for at least four out of five is thus 0,7\%.\\

This is possible to calculate in R in a several ways, either using the command dbinom or pbinom. The command dbinom calculates $$P(X = k)$$

and the command pbinom calculates $$P(X \leq k)$$

where $k$ is the number of successful trials. If $n$ is the number of trials and $p$ is the probability of a successful trials then the commands are used by writing: $dbinom$($k$,$n$,$p$) and pbinom($k$,$n$,$p$).\\

To calculate the probability that the youngster will put his shirt on the right way for at least four days of five we thus write the command:

\begin{lstlisting}
dbinom(4,5,0.2) + dbinom(5,5,0.2)
\end{lstlisting}
which gives 0.00672.\\

This is the same as writing:

\begin{lstlisting}
dbinom(c(4,5),5,0.2)
\end{lstlisting}

or

\begin{lstlisting}
dbinom(4:5,5,0.2)
\end{lstlisting}
which give two separate numbers: 0.00640 and 0.00032 which can be added together to get 0.00672.\\

There is also a command to add them together for us: 

\begin{lstlisting}
sum(dbinom(c(4,5),5,0.2))
\end{lstlisting}

or

\begin{lstlisting}
sum(dbinom(4:5,5,0.2))
\end{lstlisting}

They give the answer 0.00672.\\

The fourth way of calculating this in R is to use pbinom. As said before pbinom calculates $$P(X \leq k)$$

where $k$ is the number of successful trials. Here we want to calculate the probability that the youngster will put his shirt on the right way in 4 or 5 times (of 5 total) so the number of successful trials is 4 or greater. That means we want to calculate $$P(X \geq 4)$$

which equals $$1 - P(X \leq 3).$$

We thus put $k$ as 3 and the R command will be:

\begin{lstlisting}
1 - pbinom(3,5,0.2)
\end{lstlisting}

which also gives 0.00672.

\end{xmpl} 
\begin{xmpl} 

In a certain degree program, the chance of passing an examination is 20\%. What is the chance of passing at most 2 exams if the student takes five exams?\\

Solution:\\
In this problem, we compute the chance of a student passing, 0, 1 or 2 exams. This is given by, 
$$p(X=0 \text{ or }1 \text{ or }2)={5\choose 0}0.2^0 0.8^5 +{5\choose 1}0.2^1 0.8^4 +{5\choose 2}0.2^2 0.8^3 $$

$$=1\times0.2^0 0.8^5 +5\times 0.2^1 0.8^4 +10\times0.2^2 0.8^3 $$

$$=0.32768+0.4096+0.2048$$

$$=0.94208$$

In the R console, we can use the command,
$\verb|sum(dbinom(c(0:2),5,0.2))|$,
which also gives $$0.94208.$$

The same answer is obtained with

\begin{lstlisting}
dbinom(0,5,0.2)+dbinorm(1,5,0.2)+dbinom(2,5,0.2)
\end{lstlisting}

and with
\begin{lstlisting}
pbinom(2,5,0.2)
\end{lstlisting}

\end{xmpl} 
\begin{xmpl} 

Consider the probability of someone jumping off a cliff is 0.35. Suppose we randomly selected four individuals to participate in the cliff jumping activity.
What is the chance that exactly one of them will jump off the cliff?\\

Consider a scenario where one person jumps:\\
P (A =jump , B = refuse, C = refuse, D = refuse)

= P (A =jump) P (B = refuse) P (C = refuse) P (D = refuse)

$= (0.35)(0.65)(0.65)(0.65) = (0.35)^1 (0.65)^3 = 0.096$\\

But there are three other scenarios( B, C, or D) in which one only person decides to jump. In each of these cases, the probability is again 0.096. These four scenarios exhaust all the possible ways that exactly one of the four people jumps:

$4 \cdot (0.35)^1 (0.65)^3 = 0.38.$\\

In the R console we can use the command:
$\verb|dbinom(1,4,0.35)|$
which gives the answer as  0.384475.
\end{xmpl} 




%% Slide https://tutor-web.net/math/math612.0/lecture140/slide50
\subsection{General discrete probability distributions}
\begin{fbox}
\begin{minipage}{0.97\textwidth}
A general discrete probability distribution can be described by a list of all possible outcomes and associated probabilities.

\end{minipage}
\end{fbox}
\subsubsection{Details}
A general discrete probability distribution is described by the possible outcomes 
$$
x_1, x_2, \ldots
$$ 
and associated probabilities, denoted by 
$
p_1, p_2, \ldots$ or $p(x_1), p(x_2),\ldots$\\

If a random variable $X$ has this distribution, then we can write
$$
P[X=x_i] = p(x_i)= p_i
$$

or in general

$$
P[X=x] = p(x)
$$

where it is understood that $p(x) = 0$ if $x$ is not one of these $x_i$.

\subsubsection{Examples}
\begin{xmpl}
If $X$ is the number of heads ($H$) before obtaining the first tail ($T$) when tossing an unbiased coin 4 independent times, then the possible basic outcomes are:

$$
\begin{array}{c c c}
\hline
         & \text{Toss}    &   \\
\text{In binary} & 1\ 2\ 3\ 4 & \\#\text{H before T}\\
\hline				
  0000 & \text{H H H T} & 3\\
  0010 & \text{H H T H} & 2\\
  0011 & \text{H H T T} & 2\\
  0100 & \text{H T H H} & 1\\
  0101 & \text{H T H T} & 1\\
  0110 & \text{H T T H} & 1\\
  0111 & \text{H T T T} & 1\\
  1000 & \text{T H H H} & 0\\
  1001 & \text{T H H T} & 0\\
  1010 & \text{T H T H} & 0\\
  1011 & \text{T H T T} & 0\\
  1100 & \text{T T H H} & 0\\
  1101 & \text{T T H T} & 0\\
  1110 & \text{T T T H} & 0\\
  1111 & \text{T T T T} & 0\\
\hline
\end{array}
$$

Since the coin is unbiased, each of these has the 
same probability of occurring.  We can now count sequences to find the number of possibilities of a particular number of heads, $H$, before a tail in 4 coin tosses and thus obtain the corresponding probabilities as:\\

$$
\begin{array}{c r}
\hline
\text{Number of tosses before a heads} & \text{Probability}\\
 & p(x)\\
\hline
0 & \frac{8}{16}=\frac{1}{2}\\
1 & \frac{4}{16}=\frac{1}{4}\\
2 & \frac{2}{16}=\frac{1}{8}\\
3 & \frac{1}{16}\\
4 & \frac{1}{16}\\
\hline
\end{array}
$$

\end{xmpl}

%% Slide https://tutor-web.net/math/math612.0/lecture140/slide60
\subsection{The expected value or population mean}
\begin{fbox}
\begin{minipage}{0.97\textwidth}
The expected value is the sum of the possible outcomes, weighted with the respective probabilities (discrete variable). Think of this in terms of an urn full of marbles, each labelled with number.
\end{minipage}
\end{fbox}
\subsubsection{Details}
If the possible outcomes are $x_1, x_2...$ with probabilities $p_1, p_2...$ then the expected value is

$$
\mu=x_1 \cdot p_1+x_2 \cdot p_2 + \ldots .
$$

The fact that this is the only sensible definition of an expected value follows from
considering random draws from a finite population where there are $n_i$ possibilities of
obtaining the value $x_i$. If we set $n=\sum x_i$ and $p_i=n_i/n$ then the expected 
value above is the simple average of all the numbers in the original population.\\
In the case of the {\bf binomial distribution} with $n$ trials and success probability $p$ it turns out that 

$$
\mu=n \cdot p
$$

If $X$ is the corresponding random variable, we denote this quantity by $E[X]$.

\subsubsection{Examples}
\begin{xmpl}
 If we toss a fair coin 10 independent times, we expect on average $np=10\cdot\frac{1}{2}= 5$ heads.
\end{xmpl}
\begin{xmpl}
Toss a fair die and pay \$60 if a six comes up and nothing otherwise. The expected outcome is
$$
\frac{5}{6}\cdot\$0+\frac{1}{6}\cdot\$60= \$10.
$$
\end{xmpl}
\begin{xmpl}

In Las Vegas, a particular sports bet has about a 30\% chance of winning. If the bet wins, the bettor will win 15 dollars. If the bet loses, the bettor will lose 10 dollars.
The expected return of placing one of these bets is -2.50 dollars.

Detailed calculation:


$$\$15\cdot 0.3 - \$10\cdot 0.7 = -\$2.5$$
\end{xmpl}
\begin{xmpl}

Class starts at 8:00 and the last bus that will get you to class on time leaves at 7:30. The teacher has a policy that if you are late to class 6 of the 30 classes, then she drops your final grade by 1/10 points. You know that if you set your alarm for 7:15, you miss the 7:30 bus approximately every fourth time, but if you set it for 7:10, you'll only miss the bus approximately every eighth time. If you set it for 7:00, you'll only miss the bus every one hundredth time.\\

Part A: Assuming you try to go to class every time, can you expect to have your grade dropped in the following scenarios?

1 - You set your alarm for 7:15 throughout the duration of the class.

2 - You set your alarm for 7:15 until you reach 5 missed classes, then switch to 7:10.

3 - You set your alarm for 7:15 until you reach 5 missed classes, then switch to 7:00.\\

Part B: What is your expected grade in the course, assuming you would have had a 7/10 without the late penalty, and:

1 - You would never choose the first alarm-clock strategy and you would most likely choose scenario 2 (let's say 9/10 times), but there's a small chance you might choose the 3rd strategy (let's say 1/10 times).

2 - You would never choose the first alarm-clock strategy and you would most likely choose scenario 3 (let's say 9/10 times), but there's a small chance you might choose the 2nd strategy (let's say 1/10 times).\\

Answers:

A1 - Let's call X our random variable, which we want to be the number of times we make it to class on-time. With the alarm set to 7:15 we expect to make it to class on-time:
$$E[X]=30\times(1-\frac{1}{4})=22\frac{1}{2}$$

You're grade would most likely be dropped.\\

A2 - First we need to see how many classes we go to before we reach the 5-late-classes threshhold:
$$E[X] = n \times (1 - \frac{1}{4}) = n - 5 $$

$$E[X] = n ((1 - \frac{1}{4}) - 1) = - 5 $$

$$E[X] = n = \frac{- 5}{- \frac{1}{4}} $$

$$E[X] = n = \frac{20}{1} = 20 $$

So, the night before our 21st class, you get worried and change alarm-clock strategies. If you set it at 7:15 for the rest of the course (10 classes), you will be on time:
$$E[X] = 15 + 10 \times (1 - \frac{1}{8}) = 23 \frac {3}{4} $$

You're grade would most likely be dropped.\\

A3: If you instead start setting the alarm clock for 7:00 for the rest of the course, you will be on time:
$$E[X] = 15 + 10 \times (1 - \frac{1}{100}) = 24 \frac{1}{9} $$

You're grade would most likely NOT be dropped.\\

Part B: \textbf{This seems to contain errors} In Part A, we calculated the mean of several binomial distributions that described the expected number of days that you will arrive on-time to class. Each distribution corresponded to a different alarm-setting scenario. In this part, we are describing a different binomial distribution. It describes your expected grade. Therefore, the grade is the outcome n, weighted by the probability of you choosing the particular alarm-clock setting procedure:

$$1 - E[X] = 0 \times 6 + 0.9 \times 6 + 0.1 \times 7 = 6.1 $$

$$1 - E[X] = 0 \times 6 + 0.1 \times 6 + 0.9 \times 7 = 6.9 $$

Note that the probabilities of these three choices (0 + 0.9 + 0.1) must equal 1, since these are the only three choices defined.

\end{xmpl}

 



%% Slide https://tutor-web.net/math/math612.0/lecture140/slide70
\subsection{The population variance}
\begin{fbox}
\begin{minipage}{0.97\textwidth}
The (population) variance, for a discrete distribution, is

$$\sigma^2 = E\left[ \left ( X-\mu \right ) ^2 \right ] = (x_1 - \mu)^2 p_1 + (x_2 - \mu)^2 p_2 + ...  $$

where it is understood that the random variable $X$ has this distribution and $\mu$ is the expected value.\\

In the case of the binomial distribution, it turns out that:

$\sigma^2 = np(1 - p)$

\end{minipage}
\end{fbox}
\subsubsection{Details}
\begin{defn}
If $\mu$ is the expected value, then the {\bf variance of a discrete distribution} is
defined as
$$
\sigma ^2=(x_1 - \mu)^2 p_1 + (x_2 - \mu)^2 p_2 + \ldots .
$$
\end{defn}

If a random variable $X$ has associated probabilities, 
$p_i=P[X=x_i]$, then one can equivalently write
$$
\sigma^2 = V[X]=E\left [ \left ( X - \mu \right ) ^ 2\right ] .
$$

\subsubsection{Examples}
\begin{xmpl}
In the case of the binomial distribution, it turns out that:

$$
\sigma^2 = np(1 - p) .
$$
\end{xmpl}


{\bf Copyright}
2021, Gunnar Stefansson (editor) with contributions from very many students

This work is licensed under the Creative Commons
Attribution-ShareAlike License. To view a copy of this license, visit
http://creativecommons.org/licenses/by-sa/1.0/ or send a letter to
Creative Commons, 559 Nathan Abbott Way, Stanford, California 94305,
USA.
\clearpage
%% Lecture https://tutor-web.net/math/math612.0/lecture150
\section{Functions}
%% Slide https://tutor-web.net/math/math612.0/lecture150/slide10
\subsection{Functions of a single variable}
\begin{fbox}
\begin{minipage}{0.58\textwidth}
A function describes the relationship between variables. 

Examples:\\
$f(x) = x^2$

$y = 2+3\cdot x^4$

\end{minipage}
\hspace{0.5mm}
\begin{minipage}{0.38\textwidth}
\begin{picture}
4
\end{picture}


\end{minipage}
\end{fbox}
\subsubsection{Details}
Functions are commonly used in statistical applications, to describe relationships.
\begin{defn}
A {\bf function} describes the relationship between variables.  A variable $y$ 
is described as a function of a variable $x$ by completely specifying how $y$ can be 
computed for any given value of $x$.
\end{defn}

An example could be the relationship between a dose level and the response to the dose. \\ 

The relationship is commonly 
expressed by writing either $f(x) = x^{2}$ or $y = x^2$.\\

Usually names are given to functions, i.e. to the relationship itself. For example, $f$ might be the function and $f(x)$ could be its value for a given number $x$. Typically $f(x)$ is a number but $f$ is the function, but
the sloppy phrase "the function $f(x)=2x+4$" is also common.

\subsubsection{Examples}
\begin{xmpl}
$f(x) = x^2$ or $y = x^2$ specifies that the computed value of $y$ should always be $x^2$, for any given value of $x$. 
\end{xmpl}

%% Slide https://tutor-web.net/math/math612.0/lecture150/slide20
\subsection{Functions in R}
\begin{fbox}
\begin{minipage}{0.58\textwidth}
A function can be defined in R using the "function" command
\end{minipage}
\hspace{0.5mm}
\begin{minipage}{0.38\textwidth}
\begin{picture}
5
\end{picture}


\end{minipage}
\end{fbox}

%% Slide https://tutor-web.net/math/math612.0/lecture150/slide30
\subsection{Ranges and plots in R}
\begin{fbox}
\begin{minipage}{0.97\textwidth}
Functions in R can commonly accept a range of values and will return a corresponding vector with the outcome.


\end{minipage}
\end{fbox}
\subsubsection{Examples}
\begin{xmpl}
\begin{lstlisting}
f <- function(x) {return(x*12)}
x <- seq (-5,5,0,1)
y <- f(x)
plot {(x,y) type= 'l'} 
\end{lstlisting}
\end{xmpl}

%% Slide https://tutor-web.net/math/math612.0/lecture150/slide40
\subsection{Plotting functions}
\begin{fbox}
\begin{minipage}{0.58\textwidth}
In statistics, the function of interest is commonly called the response function.
If we write Y=f(x),
the outcome Y is usually called the response variable and x is the explanatory variable.  Function values are plotted on vertical axis while x values are plotted on horizontal axis. This plots Y against x.

\end{minipage}
\hspace{0.5mm}
\begin{minipage}{0.38\textwidth}


\end{minipage}
\end{fbox}
\subsubsection{Examples}
\begin{xmpl}
The following R commands can be used to generate a plot for function; Y= 2+3x

\begin{lstlisting}
x<- seq(0:10)
g <- function(x){
+ yhat <- 2+3*x
+ return(yhat)
+ }
 
x<-seq(0,10,0.1)
y<- g(x)
plot(x,y,type="l", xlab="x",ylab="y")
\end{lstlisting}
\end{xmpl}

%% Slide https://tutor-web.net/math/math612.0/lecture150/slide50
\subsection{Functions of several variables}
\subsubsection{Examples}
\begin{xmpl}
\begin{align}
z &= 2x+3y+4\\
v &= t^2+3x\\
w &= t^2+3b*x
\end{align}
\end{xmpl}

{\bf Copyright}
2021, Gunnar Stefansson (editor) with contributions from very many students

This work is licensed under the Creative Commons
Attribution-ShareAlike License. To view a copy of this license, visit
http://creativecommons.org/licenses/by-sa/1.0/ or send a letter to
Creative Commons, 559 Nathan Abbott Way, Stanford, California 94305,
USA.
\clearpage
%% Lecture https://tutor-web.net/math/math612.0/lecture160
\section{Polynomials}
%% Slide https://tutor-web.net/math/math612.0/lecture160/slide10
\subsection{The general polynomial}
\begin{fbox}
\begin{minipage}{0.97\textwidth}
The general polynomial:
 
$p(x)=a_{0}+a_{1}x+a_{2}x^{2}+...+a_{n}x^{n}$

The simplest:
$p(x)=a$
\end{minipage}
\end{fbox}
\subsubsection{Details}
\begin{defn}
A {\bf polynomial} describes a specific function consisting of linear combinations of positive integer powers of the explanatory variable. 
\end{defn}

The general form of a polynomial is:

$p(x)=a_{0}+a_{1}x+a_{2}x^{2}+...+a_{n}x^{n}$

The simplest of these is the constant polynomial $p(x)=a$.

%% Slide https://tutor-web.net/math/math612.0/lecture160/slide30
\subsection{The quadratic}
\begin{fbox}
\begin{minipage}{0.58\textwidth}
The general form of the quadratic (parabola) is $p(x) = ax^2 + bx + c$.

The simplest quadratic is $p(x) = x^2$
\end{minipage}
\hspace{0.5mm}
\begin{minipage}{0.38\textwidth}
\begin{picture}
6
\end{picture}

Figure:  Parabolas: Quadratic functions.
\end{minipage}
\end{fbox}
\subsubsection{Details}
The quadratic polynomial of the form $p(x) = ax^2 + bx + c$ describes a parabola when points $(x,y)$ with $y = p(x)$ are plotted.\\


The simplest parabola is $p(x) = x^2$ (Fig. a) which is always non-negative $p(x)\geq 0$ and $p(x)=0$ only when $x=0$. 
\begin{notes}
Note that $p(-x) = p(x)$ since $(-x)^2= x^2$.
\end{notes}


If the coefficient at the highest power is negative, then the parabola is "upside down"(Fig. b).\\

This is sometimes used to describe a response function.

%% Slide https://tutor-web.net/math/math612.0/lecture160/slide40
\subsection{The cubic}
\begin{fbox}
\begin{minipage}{0.58\textwidth}
The general form of a cubic polynomial is:

$p(x)=ax^3 + bx^2 + cx + d$

\end{minipage}
\hspace{0.5mm}
\begin{minipage}{0.38\textwidth}
\begin{picture}
7
\end{picture}

Figure:  $y=x^3-20x^2-30x-4$
\end{minipage}
\end{fbox}

%% Slide https://tutor-web.net/math/math612.0/lecture160/slide50
\subsection{The Quartic}
\begin{fbox}
\begin{minipage}{0.58\textwidth}
The general form of the quartic polynomial is $p(x) = ax^4 + bx^3 + cx^2 + dx + e$
















\end{minipage}
\hspace{0.5mm}
\begin{minipage}{0.38\textwidth}
\begin{picture}
8
\end{picture}

Figure:  The general shape. Here we used the following equation
$y=x^4-x^3-7x^2+x+6$
\end{minipage}
\end{fbox}

%% Slide https://tutor-web.net/math/math612.0/lecture160/slide60
\subsection{Solving the linear equation}
\begin{fbox}
\begin{minipage}{0.97\textwidth}
If the value of $y$ is given and we know that $x$ and $y$ are on a specific line so that $y = a + bx$, then we can find the value of $x$
\end{minipage}
\end{fbox}
\subsubsection{Details}
If a value of $y$ is given and we know that $x$ and $y$ lie on a specific straight line so that $y = a + bx$, then we can find the value of $x$ by considering $y = a+bx$ as an equation to be solved for $x$, since $y$, $a$ and $b$ are all known. \\

The general solution is found through the following steps:
\begin{itemize}
\item Equation: $y = a + bx$
\item Subtract $a$ from both sides
\begin{itemize}
\item
$y-a = bx$
\item
$bx=y-a$
\end{itemize}
\item
Divide by $b$ on both sides if $b$ is not equal to 0.
\begin{itemize}
\item $x=\frac{1}{b}(y-a).$
\end{itemize}
\end{itemize}

%% Slide https://tutor-web.net/math/math612.0/lecture160/slide70
\subsection{Roots of the quadratic equation}
\begin{fbox}
\begin{minipage}{0.97\textwidth}
The general solution of $ax^2 + bx + c = 0$ is given by $ x = \frac{-b \pm \sqrt{b^2 - 4ac}}{2a}$.
\end{minipage}
\end{fbox}
\subsubsection{Details}
Suppose we want to solve $ax^2 + bx + c = 0$, where 
$a \neq 0$.

The general solution is given by the formula

$$ x = \frac{-b \pm \sqrt{b^2 - 4ac}}{2a},$$

if  $b^2 - 4ac \geq 0$. On the other hand, if $b^2-4ac<0$, the quadratic equation has no real solution.

\subsubsection{Examples}
\begin{xmpl}

Solve $x^2 - 3x + 2 = 0$

Putting this into the context of the formulation $ax^2+bx+c=0$, the constants are;

$a = 1, b = -3 , c = 2$

Inserting this into the formula for the roots gives:

\begin{eqnarray*}
x &=& \frac{-(-3) \pm \sqrt{(-3)^2 - 4(1)(2)}}{2(1)}\\
x &=& \frac{3 \pm \sqrt{9 - 8}}{2}\\
x &=& \frac{3 \pm \sqrt{1}}{2}\\
x &=& \frac{3 + 1}{2} , \frac{3 - 1}{2}\\
x &=& \frac{4}{2} , \frac{2}{2}\\
x &=& 2 , 1 
\end{eqnarray*}

\end{xmpl}
\begin{xmpl}


Find the roots of the following polynomial

$$3x^{4} + 14x^{2} + 15$$

We can use the quadratic equation to solve for the roots of this polynomial if we substitute a variable for $$x^{2}$$

Let's use the letter $$a$$

$$3a^{2} + 14a + 15$$

We then plug the constants in to the quadratic equation.

$$x = \frac{-(14) \pm \sqrt{14^{2} - (4)(3)(15)}}{(2)(3)}$$

which simplifies to

$$\frac{-(14) \pm \sqrt{196 - 180}}{6}$$

which equals $$-1\frac{2}{3}$$

and $$-3.$$

Then, since we substituted a for $$x^2$$

we need to take the square root of these values to get the roots of the polynomial.

So, $$x_{1,2} = \pm \sqrt{-1\frac{2}{3}}$$

and $$x_{3,4} = \pm \sqrt{3}$$
\end{xmpl}


{\bf Copyright}
2021, Gunnar Stefansson (editor) with contributions from very many students

This work is licensed under the Creative Commons
Attribution-ShareAlike License. To view a copy of this license, visit
http://creativecommons.org/licenses/by-sa/1.0/ or send a letter to
Creative Commons, 559 Nathan Abbott Way, Stanford, California 94305,
USA.
\clearpage
%% Lecture https://tutor-web.net/math/math612.0/lecture170
\section{Simple data analysis in R}
%% Slide https://tutor-web.net/math/math612.0/lecture170/slide10
\subsection{Entering data; dataframes}
\begin{fbox}
\begin{minipage}{0.97\textwidth}
Several methods exist to enter data into R:

\begin{enumerate}
	\item Enter directly: x<-c(4,3,6,7,8)
	\item Read in a single vector: x<-scan("filename")
	\item Use: x<-read.table("file address")
\end{enumerate}
\end{minipage}
\end{fbox}
\subsubsection{Details}
The most direct method will not work if there are a lot numbers; therefore, the second method is to read in a single vector by x<-scan("filename"), "filename" = text string, either a full path name or refers to a file in the working directory.\\

The scan() command returns a vector, but the read.table() command returns a dataframe, which is a rectangular table of data whose columns have names.  A column can be extracted from a data frame, e.g., with x<- dat\$a where"dat" is the name of the data frame and "a" is the name of a column.

\begin{notes}
Note that for read.table("file address"), "file address" refers to the location of the file.  Thus, it can be the URL or the complete file directory depending on where the table is stored.
\end{notes}
\subsubsection{Examples}
\begin{xmpl}
Below are three examples using R code to enter data
\begin{enumerate}
	\item x<-c(4,3,6,7,8)
	\item x<-scan("lecture 70.txt")
	\item x<-read.table("http://notendur.hi.is/~gunnar/kennsla/alsm/data/set115.dat", header=T) 
\end{enumerate}
\end{xmpl}

%% Slide https://tutor-web.net/math/math612.0/lecture170/slide20
\subsection{Histograms}
\begin{fbox}
\begin{minipage}{0.58\textwidth}
A histogram is a graphical display of tabulated frequencies, shown as bars.

In R use the command: hist()
\end{minipage}
\hspace{0.5mm}
\begin{minipage}{0.38\textwidth}
\begin{picture}
9
\end{picture}


\end{minipage}
\end{fbox}
\subsubsection{Examples}
A histogram is a graphical display of tabulated frequencies, shown as bars.

\begin{xmpl}
If we toss a fair die 100 times and record the number of sixes, then we can view that as the outcome of a random variable $X$, which is binomial with $n=100$ and $p=\frac{1}{6}$, i.e $X \sim b(n=100,p=\frac{1}{6})$\\

Now this can be done e.g. 1000 times to obtain numbers, $x_1,...,x_{1000}$.  Within R this can be simulated using 

\begin{lstlisting}
x <- rbinom(1000,100,1/6)
\end{lstlisting}


We would typically plot these using a histogram, e.g. \\
hist(x)\\
or\\ % added OR - you only use one of these in a given plot
hist(x,nclass=50);l
\end{xmpl}


%% Slide https://tutor-web.net/math/math612.0/lecture170/slide30
\subsection{Bar Charts}
\begin{fbox}
\begin{minipage}{0.58\textwidth}
The bars in a bar chart usually correspond to frequencies in categories and are therefore kept apart.
\end{minipage}
\hspace{0.5mm}
\begin{minipage}{0.38\textwidth}
\begin{picture}
10
\end{picture}


\end{minipage}
\end{fbox}
\subsubsection{Details}
A bar chart is similar to the histogram but is used for categorical data.


%% Slide https://tutor-web.net/math/math612.0/lecture170/slide40
\subsection{Mean, standard error, standard deviations}
\subsubsection{Details}
%memo: use two dollar signs to get a stand-alone equation
The most familiar measure of central tendency is the arithmetic mean.  
\begin{defn}
An {\bf arithmetic mean} is the sum of the values divided by the number values, typically expressed as:

$$
\bar{y} = \frac{\sum_{i=1}^{n} y_i}{n}
$$
\end{defn}
\begin{defn}
The {\bf sample variance} is a measure of the spread of a set of values from the mean value:

$$s^2 = \frac{1}{n-1}\displaystyle\sum_{i=1}^{n}(x_i - \bar{x} )^2$$
\end{defn}

The sample standard deviation is more commonly used as a measure of the spread of a set of values from the mean value.  
\begin{defn}
The {\bf standard deviation} is the square root of the variance and may be expressed as:

$$s = \sqrt{\frac{1}{n-1}\displaystyle\sum_{i=1}^{n}(x_i - \bar{x} )^2}$$
\end{defn}

\begin{defn}
The {\bf standard error} is a method used to indicate the reliability of the sample mean:

%can do better:
%$SE_\bar{y} = \sqrt{\frac{s^2}{n}}$
$$SE_{\bar{y}} = \sqrt{\frac{s^2}{n}}$$
\end{defn}

If a vector x in R contains an array of numbers then:\\
$\verb;mean(x);$ returns the average, $\bar{x}$ \\
$\verb;sd(x);$ returns the standard deviation,$s$\\ 
$\verb;var(x);$ returns the variance, $s^2$ \\\\
We may also want to use several other related operations in R:\

$\verb;median(x);$, the median value in vector x \\
$\verb;range(x);$, which list the range: $\verb;max(x)-\verb;min(x);$\\
If the variable $\verb;x;$ contains discrete categories, 
$\verb;table(x);$ returns counts of the frequency in each category.



%% Slide https://tutor-web.net/math/math612.0/lecture170/slide50
\subsection{Scatter plots and correlations}
\begin{fbox}
\begin{minipage}{0.58\textwidth}
If we have paired explanatory and response data we are often interested in seeing if a  relationship exists between them. To do this, we first plot the data in a scatter plot.


\end{minipage}
\hspace{0.5mm}
\begin{minipage}{0.38\textwidth}
\begin{picture}
11
\end{picture}

Figure:  Scatter plot showing the length-weight relationship of fish species "X".
Data source : Marine Resource Institution - Iceland.
\end{minipage}
\end{fbox}
\subsubsection{Details}
A first step in analyzing data is to prepare different plots.  The type of variable will determine the type of plot.  For example, when using a scatter plot both the explanatory and response data should be continuous variables. \\  



The equation for the Pearson correlation coefficient is:  

%$r_x_y = \frac{\displaystyle\sum_{i=1}^{n}(x_i - \bar{x})(y_i - \bar{y})}{(n-1)S_xS_y}$,
% don't use double subscript 
% use double dollars for a standalone equation
% and then drop the displaystyle stuff

$$
r_{x,y} = \frac{\sum_{i=1}^{n}(x_i - \bar{x})(y_i - \bar{y})}{\sum_{i=1}^{n}(x_i - \bar{x})^2\sum_{i=1}^{n}(y_i - \bar{y})^2}, 
$$

% use lowar case s for st.dev.
% insert dollar signs around \bar
where $\bar{x}$ and $\bar{y}$ are the sample means of the x- and y-values.

The correlation is always between -1 and 1. 

\subsubsection{Examples}
The following R commands can be used to generate a scatter plot for vectors x and y
\begin{xmpl}
\begin{lstlisting}
plot(x,y) 
\end{lstlisting}
\end{xmpl}

{\bf Copyright}
2021, Gunnar Stefansson (editor) with contributions from very many students

This work is licensed under the Creative Commons
Attribution-ShareAlike License. To view a copy of this license, visit
http://creativecommons.org/licenses/by-sa/1.0/ or send a letter to
Creative Commons, 559 Nathan Abbott Way, Stanford, California 94305,
USA.
\clearpage
%% Lecture https://tutor-web.net/math/math612.0/lecture180
\section{Indices and the apply commands in R}
%% Slide https://tutor-web.net/math/math612.0/lecture180/slide10
\subsection{Giving names to elements}
\begin{fbox}
\begin{minipage}{0.97\textwidth}
We can name elements of vectors and data frames in R using the "names" command.




\end{minipage}
\end{fbox}
\subsubsection{Examples}
\begin{xmpl}
\begin{lstlisting}
X<-c(41, 3, 73)
names(X)<-c("One", "Two", "Three")
\end{lstlisting}

View the results by simply typing "X" and the output of "X" is given as follows:

\begin{lstlisting}
X
One   Two Three 
41     3    73
\end{lstlisting}

With this we can refer to the elements by name as well as locations using...

\begin{lstlisting}
X[1] 
One  
\end{lstlisting}

\begin{lstlisting}
X["Three"] 
Three  
73 
\end{lstlisting}

\end{xmpl}



%% Slide https://tutor-web.net/math/math612.0/lecture180/slide30
\subsection{Regular matrix indices and naming}
\begin{fbox}
\begin{minipage}{0.97\textwidth}
A matrix is a table of numbers. 
Typical matrix indexing: mat[i,j], mat[1:2,] etc\\

A matrix can have row and column names
Indexing with row and column names: mat["a","B"]

\end{minipage}
\end{fbox}
\subsubsection{Details}
% the main body of text underlying the slide goes here -- the percentage sign is a comment
\begin{defn}
A {\bf matrix} is a (two-dimensional) table of numbers, indexed by row and column numbers.
\end{defn}
\begin{notes}
Note that a matrix can also have row and column names so that the matrix can be indexed by its names rather than numbers.
\end{notes}

\subsubsection{Examples}
% \bf make the example bold -- the curly braces limit the extent of the bold
\begin{xmpl}
 Consider a matrix with 2 rows and 3 columns.  Consider extracting first element (1,2), then all of line 2 and then columns 2-3 in an R session:

\begin{lstlisting}
mat<-matrix(1:6,ncol=3)
mat
     [,1] [,2] [,3]
[1,]    1    3    5
[2,]    2    4    6

mat[1,2]
[1] 3

mat[2,]
[1] 2 4 6

mat[,2:3]
     [,1] [,2]
[1,]    3    5
[2,]    4    6
\end{lstlisting}

Next, consider the same matrix, but give names to the rows and columns.
The rows will get the names "a" and "b"
and the columns will be named "A", "B" and "C".
% \begin{verbatim} is used to start a "verbatim" environment

The entire R session could look like this: 
\begin{lstlisting}
mat<-matrix(1:6,ncol=3)
dimnames(mat)<-list(c("a","b"),c("A","B","C"))
mat
  A B C
a 1 3 5
b 2 4 6

mat["b",c("B","C")]
B C 
4 6 

\end{lstlisting}
\end{xmpl}

%% Slide https://tutor-web.net/math/math612.0/lecture180/slide40
\subsection{The apply command}
\begin{fbox}
\begin{minipage}{0.97\textwidth}
The apply command...

apply(mat,2,sum) -- applies the sum function within each column

apply(mat,1,mean) -- computes the mean within each row

\end{minipage}
\end{fbox}

%% Slide https://tutor-web.net/math/math612.0/lecture180/slide50
\subsection{The tapply command}
\begin{fbox}
\begin{minipage}{0.97\textwidth}
Commonly one has a data vector and another vector of the same length giving categories for the measurements.
In this case one often wants to compute the mean or variance (or median etc) within each category. To do this we use the tapply command in R.

\end{minipage}
\end{fbox}
\subsubsection{Examples}
\begin{xmpl}
\begin{lstlisting}
z<-c(5,7,2,9,3,4,8)
i<-c("m","f","m","m","f","m","f")
\end{lstlisting}

A. Find the sum within each group
\begin{lstlisting}
tapply(z,i,sum)
 f  m 
18 20 
\end{lstlisting}


B.Find the sample sizes
\begin{lstlisting}
tapply(z,i,length) 
f m 
3 4 
\end{lstlisting}

C.Store outputs and use names
\begin{lstlisting}
n<-tapply(z,i,length) 
n
f m 
3 4 
n["m"]
m 
4 
\end{lstlisting}
\end{xmpl}

%% Slide https://tutor-web.net/math/math612.0/lecture180/slide60
\subsection{Logical indexing}
\begin{fbox}
\begin{minipage}{0.97\textwidth}
A logical vector consists of $TRUE$ (1) or $FALSE$ (0) values. These can be used to index vectors or matrices.
\end{minipage}
\end{fbox}
\subsubsection{Examples}
\begin{xmpl}
\begin{lstlisting}
i<-c("m","f","m","m","f","m","f")
z<-c(5,7,2,9,3,4,8)

i=="m"
[1]  TRUE FALSE  TRUE  TRUE FALSE  TRUE FALSE

z[i=="m"]
[1] 5 2 9 4

z[c(T,F,T,T,F,T,F)]
[1] 5 2 9 4
\end{lstlisting}
\end{xmpl}

%% Slide https://tutor-web.net/math/math612.0/lecture180/slide70
\subsection{Lists, indexing lists}
\begin{fbox}
\begin{minipage}{0.97\textwidth}
A list is a collection of objects. Thus, data frames are lists.
\end{minipage}
\end{fbox}
\subsubsection{Examples}
\begin{xmpl}
\begin{lstlisting}
x<-list(y=2,z=c(2,3),w=c("a","b","c"))
x[["z"]]
[1] 2 3
names(x)
[1] "y" "z" "w"
x["w"]
[1] "a" "b" "c"
x$w
[1] "a" "b" "c"

\end{lstlisting}
\end{xmpl}

{\bf Copyright}
2021, Gunnar Stefansson (editor) with contributions from very many students

This work is licensed under the Creative Commons
Attribution-ShareAlike License. To view a copy of this license, visit
http://creativecommons.org/licenses/by-sa/1.0/ or send a letter to
Creative Commons, 559 Nathan Abbott Way, Stanford, California 94305,
USA.
\clearpage
%% Lecture https://tutor-web.net/math/math612.0/lecture190
\section{Functions of functions and the exponential function}
%% Slide https://tutor-web.net/math/math612.0/lecture190/slide10
\subsection{Exponential growth and decline}
\begin{fbox}
\begin{minipage}{0.58\textwidth}
Exponential growth is typically expressed as:

$y(t)=Ae^{kt}$





%Assuming no competition or other limiting factors and applying exponential growth, this implies that e.g. a resource can potentially grow at a constant rate per unit of time.  It is expressed as: $$x(1+r)^n$$

%
\end{minipage}
\hspace{0.5mm}
\begin{minipage}{0.38\textwidth}
\begin{picture}
12
\end{picture}

Figure:  Exponential growth curve
\end{minipage}
\end{fbox}
\subsubsection{Details}
\begin{defn}
{\bf Exponential growth} is the rate of population increase across time when a population is devoid of limiting factors (i.e. competition, resources, etc.) and experiences a constant growth rate.
\end{defn}

 Exponential growth is typically expressed as:

$y(t)=Ae^{kt}$

where \\
$A$ (sometimes denoted $P$)=initial population size\\
$k$= growth rate\\
$t$ =number of time intervals\\

\begin{notes}
Note that exponential growth occurs when $k>0$ and exponential decline occurs when $k<0$.
\end {notes}

%A bacterial colony or animal population with unlimited resources can potentially grow at a constant rate per unit time, as do (in principle) dollars in a bank.  The growth implies that the populations changes from $A$ to $Ae^k$. Exponential growth occurs when $k>0$ and exponential decline occurs when $k<0$. %

%% Slide https://tutor-web.net/math/math612.0/lecture190/slide20
\subsection{The exponential function}
\begin{fbox}
\begin{minipage}{0.97\textwidth}
An exponential function is a function with the form:
$f(x)=b^x$

%A special constant most commonly chosen when discussing exponential growth or decline is the constant %e\sim 2,71828183$ \\%%

\end{minipage}
\end{fbox}
\subsubsection{Details}
For the exponential function $f(x)=b^x$, $x$ is a positive integer and $b$ is a fixed positive real number.  The equation can be rewritten as:
$$f(x)=b^x=b\cdot b \cdot b...b.$$

When the exponential function is written as $f(x)=e^x$ then, it has a growth rate at time $x$ equivalent to the value of $e^x$ for the function at $x$.


%The constant $e$ appears naturally as the limit of %$(1+1/n)^n$ as we increase $n$ without bound.\\%
%The exponential function $f(x)=e^x$ is commonly %written $exp(x)$ \\%
%and often has a parameter $b$ so that e.g. %$f(x)=e^{bx}$ is also called an exponential %function.\\
%Note that $e^{bx}=(e^b)^x$ and we can always use %$e^{bx}$ instead of $a^x$ for any $a>0$, by %appropriate choice of $b$.%

%% Slide https://tutor-web.net/math/math612.0/lecture190/slide30
\subsection{Properties of the exponential function}
\begin{fbox}
\begin{minipage}{0.97\textwidth}
Recall that the methods of the basic arithmetic implies that:
$$
e^{a+b} = e^a e^b
$$ for any real numbers $a$ and $b$.
\end{minipage}
\end{fbox}

%% Slide https://tutor-web.net/math/math612.0/lecture190/slide40
\subsection{Functions of functions}
\subsubsection{Details}
Consider two functions, $f$ and $g$, each defined for some set of real numbers. Where $x$ can be solved in function $f$ using $Y = f(x)$ when
$g(Y)$ exists for all such resulting $Y$. If $Y = f(x)$ and $g(Y)$ exist then we can compute $g(f(x))$ for any $x$. 
\\
If \\
$f(x) = {x}^2$ and \\
$  g(y)= {e}^y$ then \\
$g(f(x))= {e}^{f(x)} = {e}^{x^2}$\\

If we call the resulting function ${h}$;\\
$ h(x) = g(f(x))$\\
Then ${h}$ is commonly written as \\
$ {h} = {g}\circ{f}$\\
\\

\subsubsection{Examples}
\begin{xmpl}
If\\
$  g(x)= {3}+ {2}x$  and \\
$f(x) = {5}{x}^2$\\
Then \\
$g(f(x)) = {3} +{2} f(x)$\\
$g(f(x)) = {3} +{10x}^2$\\
   
$f(g(x)) = {5}{(g(x))}^2$\\
$f(g(x)) = {5}{({3}+{2x})}^2$\\
$f(g(x)) = {45}+{60x}+{20x}^2$
\end{xmpl}

%% Slide https://tutor-web.net/math/math612.0/lecture190/slide50
\subsection{Storing and using R code}
\begin{fbox}
\begin{minipage}{0.97\textwidth}
As R code gets more complex (more lines) it is usually stored in files. Functions are typically stored in separate files.
\end{minipage}
\end{fbox}
\subsubsection{Examples}
\begin{xmpl}
Save the following file (test.r):
\begin{lstlisting}
x=4
y=8
cat("x+y is", x+y, "\n")
\end{lstlisting}
To read the file use: 
\begin{lstlisting}
source("test.r")
\end{lstlisting}
and the outcome of the equation is displayed in R
\end{xmpl}

%% Slide https://tutor-web.net/math/math612.0/lecture190/slide60
\subsection{Storing and calling functions in R}
\begin{fbox}
\begin{minipage}{0.97\textwidth}
To save a function in a separate file use a command of the form "function.r".
\end{minipage}
\end{fbox}
\subsubsection{Examples}
\begin{xmpl}

\begin{lstlisting}
f<-function(x) {
    return (exp(sum(x)))
    }
\end{lstlisting}
can be stored in a file function.r and
subsequently read using the source command.
\end{xmpl}

{\bf Copyright}
2021, Gunnar Stefansson (editor) with contributions from very many students

This work is licensed under the Creative Commons
Attribution-ShareAlike License. To view a copy of this license, visit
http://creativecommons.org/licenses/by-sa/1.0/ or send a letter to
Creative Commons, 559 Nathan Abbott Way, Stanford, California 94305,
USA.
\clearpage
%% Lecture https://tutor-web.net/math/math612.0/lecture195
\section{Inverse functions and the logarithm}
%% Slide https://tutor-web.net/math/math612.0/lecture195/Slide10
\subsection{Inverse Function}
\begin{fbox}
\begin{minipage}{0.97\textwidth}
If $f$ is a function, then the function $g$ is the inverse function of $f$ if 
$$g(f(x))=x$$

for all $x$ in which $f(x)$ can be calculated
\end{minipage}
\end{fbox}
\subsubsection{Details}
The inverse of a function $f$ is denoted by $f^{-1}$, i.e. $$f^{-1}(f(x))=x $$
\subsubsection{Examples}
\begin{xmpl}
If $f(x) = x^2$ for $x<0$
then the function $g$, defined as $g(y)=\sqrt{y}$ for $y>0$, is  not the inverse of $f$ since
$g(f(x))=\sqrt{x^2}= |x|= -x$ for $x<0$.

\end{xmpl}

%% Slide https://tutor-web.net/math/math612.0/lecture195/slide20
\subsection{When the inverse exists:  The domain question}
\begin{fbox}
\begin{minipage}{0.58\textwidth}
Inverses do not always exist. For an inverse of $f$ to exist, $f$ must be one-to-one, i.e. for each $x$, $f(x)$ must be unique. 


\end{minipage}
\hspace{0.5mm}
\begin{minipage}{0.38\textwidth}
\begin{picture}
13
\end{picture}

Figure:  The function $f(x)=x^2$ does not have an inverse since f(x)=1 has two possible solutions -1 and 1.
\end{minipage}
\end{fbox}
\subsubsection{Examples}
\begin{xmpl}

$f(x)=x^2$ does not have an inverse since $f(x)=1$ has two possible solutions -1 and 1.


\end{xmpl}

\begin{notes}

Note that iff $f$ is a function, then the function $g$ is the inverse function of $f$, if $g(f(x)) = x$ for all calculated values of $x$ in $f(x)$.\\

The inverse function of $f$ is denoted by $f^{-1}$, i.e. $f^{-1}(f(x)) = x$. 
\end{notes}
\begin{xmpl}
What is the inverse function, $f^{-1}$, of $f$ if $f(x) = 5 + 4x$.\\

The simplest approach is to write $y=f(x)$ and solve for $x$:

With $$f(x) = 5 + 4x$$

we write $$y = 5 + 4x$$

which we can now rewrite as
$$y - 5 = 4x$$

and this implies  $$\frac{y-5}{4} = x $$

And there we have it, very simple: 

$$f^{-1}(f(x)) = \frac{y - 5}{4}$$
\end{xmpl}


%% Slide https://tutor-web.net/math/math612.0/lecture195/slide30
\subsection{The base 10 logarithm}
\begin{fbox}
\begin{minipage}{0.97\textwidth}
When $x$ is a positive real number in $x=10^y$, $y$ is referred to as the base 10 logarithm of x and is written as: 
$$ y=\log_{10}(x) $$

or 
$$ y=\log(x) $$



\end{minipage}
\end{fbox}
\subsubsection{Details}
If $\log (x) = a$ and $\log (y)=b$, then $x = 10^a$ and $y = 10^b$, and
$$ x \cdot y = 10^a \cdot 10^b = 10^{a+b}$$
so that $$ \log(xy) = a+b $$
\subsubsection{Examples}
\begin{xmpl}

\begin{eqnarray*}
log(100)&=& 2 \\
log(1000)&=& 3
\end{eqnarray*}
\end{xmpl}
\begin{xmpl}
If $$\log(2) \approx 0.3$$

 
then $$10^y=2$$

 
\begin {notes}
Note that 
$$2^{10}=1024 \approx 1000 = 10^3$$

therefore
$$2 \approx 10^{3/10}$$

so
$$\log (2) \approx 0.3$$
\end{notes}
\end{xmpl}



%% Slide https://tutor-web.net/math/math612.0/lecture195/slide40
\subsection{The natural logarithm}
\begin{fbox}
\begin{minipage}{0.58\textwidth}
A logarithm with $e$ as a base is referred to as the  natural logarithm and is denoted as $ln$ : 
$$y=ln(x)$$

if 
$$x=e^y=exp(y)$$

Note that $ln$ is the inverse of $exp$.
\end{minipage}
\hspace{0.5mm}
\begin{minipage}{0.38\textwidth}
\begin{picture}
14
\end{picture}

Figure:  The curve depicts the fuction $y=\ln(x)$ and shows that $ln$ is the inverse of $exp$. Note that $\ln(1)=0$ and when $y=0$ then $e^0=1$.
\end{minipage}
\end{fbox}

%% Slide https://tutor-web.net/math/math612.0/lecture195/slide50
\subsection{Properties of logarithm(s)}
\begin{fbox}
\begin{minipage}{0.97\textwidth}
Logarithms transform multiplicative models into additive models, i.e.
$$\ln(a\cdot b) = \ln a + \ln b$$
\end{minipage}
\end{fbox}
\subsubsection{Details}
This implies that any statistical model, which is multiplicative becomes additive on a log scale, e.g.

$$y = a \cdot w^b \cdot x^c$$

$$\ln y = (\ln a) + \ln (w^b) + \ln (x^c)$$

Next, note that

\begin{eqnarray*}
\ln (x^2)&=& \ln (x \cdot x)\\
&=& \ln x + \ln x\\
&=& 2 \cdot ln x
\end{eqnarray*} 

and similarly $\ln (x^n) = n \cdot \ln x$ for any integer n.

In general $\ln (x^c) = c \cdot \ln x$ for any real number c (for x>0).

Thus the multiplicative model (from above)

$$y=a \cdot w^b \cdot x^c$$

becomes
$$y= (\ln a) + b \cdot \ln w + c \cdot \ln x$$
which is a linear model with parameters $(\ln a)$, $b$ and $c$.

In addition, the log-transform is often variance-stabilizing. 

%% Slide https://tutor-web.net/math/math612.0/lecture195/slide60
\subsection{The exponential function and the logarithm}
\begin{fbox}
\begin{minipage}{0.97\textwidth}
The exponential function and the logarithms are inverses of each other\\
$$x = e^y \Leftrightarrow y = \ln{x}$$
\end{minipage}
\end{fbox}
\subsubsection{Details}
\begin{notes}
Note the properties:

$$\ln (x \cdot y) = \ln (x) + \ln (y)$$

and
$$e^a \cdot e^b = e^{a+b}$$
\end{notes}
\subsubsection{Examples}
\begin{xmpl}

Solve the equation $$10e^{1/3x} + 3 = 24$$

for $x$.

First, get the 3 out of the way.

 $$10e^{1/3x} = 21$$

Then the 10.

$$e^{1/3x} = 2.1$$

Next, we can take the natural log of 2.1. Since $ln$ is an inverse function of $e$ this would result in 

$$\frac{1}{3}x = \ln(2.1)$$

This yields $$x = \ln(2.1) \cdot 3$$

which is $$\approx 2.23$$
\end{xmpl}


{\bf Copyright}
2021, Gunnar Stefansson (editor) with contributions from very many students

This work is licensed under the Creative Commons
Attribution-ShareAlike License. To view a copy of this license, visit
http://creativecommons.org/licenses/by-sa/1.0/ or send a letter to
Creative Commons, 559 Nathan Abbott Way, Stanford, California 94305,
USA.
\clearpage
%% Lecture https://tutor-web.net/math/math612.0/lecture210
\section{Continuity and limits}
%% Slide https://tutor-web.net/math/math612.0/lecture210/slide10
\subsection{The concept of continuity}
\begin{fbox}
\begin{minipage}{0.58\textwidth}
A function is continuous if it has no jumps.  Thus, small changes in each $x_0$, the input, correspond to small changes in the output, $f(x_0)$.
\end{minipage}
\hspace{0.5mm}
\begin{minipage}{0.38\textwidth}
\begin{picture}
15
\end{picture}

Figure:  The above figure is an example of linear growth.  Thomas Robert Malthus (1766-1834) warned about the dangers of uninhibited population growth.
\end{minipage}
\end{fbox}
\subsubsection{Details}
A function is said to be discontinuous if it has jumps.  The function is continuous if it has no jumps. Thus, for a continuous function, small changes in each $x_0$, the input, correspond to small changes in the output, $f(x_0)$.


\begin{notes}
Note that polynomials are continuous as are logarithms (for positive numbers). 
\end{notes}

%% Slide https://tutor-web.net/math/math612.0/lecture210/slide20
\subsection{Discrete probabilities and cumulative distribution functions}
\begin{fbox}
\begin{minipage}{0.58\textwidth}
The cumulative distribution function  for a discrete random variable is discontinuous.

\end{minipage}
\hspace{0.5mm}
\begin{minipage}{0.38\textwidth}
\begin{picture}
16
\end{picture}


\end{minipage}
\end{fbox}
\subsubsection{Details}
\begin{defn}
If $X$ is a random variable with a discrete probability distribution and the probability mass function of
$$
p(x)=P[X=x]
$$
then the \textbf{cumulative distribution function}, defined by
$$
F(X)=P[X\leq x]
$$ 
is
discontinuous, i.e. it jumps at points in which a positive probability occurs.
\end{defn}
\begin{notes}
When drawing discontinuous functions it is common practice to use a filled circle at $(x,f(x))$ to clarify what the function value is at a point $x$ of discontinuity.
\end{notes}
\subsubsection{Examples}
\begin{xmpl}
If a coin is tossed 3 independent times and $X$ denotes the number of heads, then $X$ can only take on the values 0, 1, 2 and 3.  The probability of  landing exactly $x$ heads, $P(X=x)$, is $p(x) = \binom{n}{x} p^n (1-p)^{n-x}$.
The probabilities are
\begin{verbatim}
x  | p(x) | F(x)
----------------
0  | 1/8  | 1/8
1  | 3/8  | 4/8
2  | 3/8  | 7/8
3  | 1/8  | 1
\end{verbatim}  

The cumulative distribution function, $F(x)=P[X \leq x] = \sum_{t\leq x} p(t)$ has jumps and is therefore discontinuous. 

\begin{notes}
Notice on the above figure how the circles are filled in, the solid circles indicate where the function value is.
\end{notes}
\end{xmpl}

%% Slide https://tutor-web.net/math/math612.0/lecture210/slide30
\subsection{Notes on discontinuous function}
\begin{fbox}
\begin{minipage}{0.58\textwidth}
A function is discontinuous for values or ranges of the variable that do not vary continuously as the variable increases.  In other words, breaks or jumps.
\end{minipage}
\hspace{0.5mm}
\begin{minipage}{0.38\textwidth}
\begin{picture}
17
\end{picture}

Figure:  $f(x) = \frac{1}{x}$, where $x\neq 0$
\end{minipage}
\end{fbox}
\subsubsection{Details}
A function can be discontinuous in a number of different ways.  Most
commonly, it may jump at certain points or increase without bound in
certain places.\\

Consider the function $f$, defined by $f(x)= 1/x$ when  $x\neq 0$.
Naturally, $1/x$ is not defined for $x=0$.
This function increases towards $+\infty$ as $x$ goes to zero from the
right but decreases to $-\infty$ as
$x$ goes to zero from the left. Since the function does not have the
same limit from the right and the left,
it can not be made continuous at $x=0$ even if one
tries to define $f(0)$ as some number.


%% Slide https://tutor-web.net/math/math612.0/lecture210/slide40
\subsection{Continuity of polynomials}
\begin{fbox}
\begin{minipage}{0.58\textwidth}
All polynomials, 
$
p(x)=a_0+a_1x+a_2x^2+\ldots +a_n x^n ,
$
are continuous.

\end{minipage}
\hspace{0.5mm}
\begin{minipage}{0.38\textwidth}
\begin{picture}
18
\end{picture}


\end{minipage}
\end{fbox}
\subsubsection{Details}
It is easy to show that simple polynomials such as $p(x)=x$, $p(x)=a+bx$, $p(x)=ax^2+bx+c$ are continuous functions.\\

It is generally true that a polynomial of the form
$$
p(x)=a_0+a_1x+a_2x^2+\ldots +a_n x^n
$$
is a continuous function.


%% Slide https://tutor-web.net/math/math612.0/lecture210/slide50
\subsection{Simple Limits}
\begin{fbox}
\begin{minipage}{0.58\textwidth}
A "limit" is used to describe the value that a function or sequence "approaches" as the input or index approaches some value. Limits are used to define continuity, derivatives and integrals.

\end{minipage}
\hspace{0.5mm}
\begin{minipage}{0.38\textwidth}
\begin{picture}
19
\end{picture}

Figure:  $f(x) = x^x$, for $x>0$
\end{minipage}
\end{fbox}
\subsubsection{Details}
\begin {defn}
A {\bf limit} describes the value that a function or sequence approaches as the input or index approaches some value. 
\end{defn}
Limits are essential to calculus (and mathematical analysis in general) and are used to define continuity, derivatives and integrals.\\

Consider a function and a point ${x}_0$. If $ f(x) $ gets steadily closer to some number $c$ as $x$ gets closer to a number $x_0$, then $c$ is called the limit of  $f(x)$ as $x$ goes to $x_0$ and is written as: 
$$ 
c= \lim_{x\to x_0}f(x)
$$

If $c = f(x_0)$ then $f$ is {\bf continuous} at $x_0$.
\subsubsection{Examples}
\begin{xmpl}
A simple example of limits: 

Evaluate the limit of $f(x) = \frac{x^{2}-16}{x-4} $ when $x\rightarrow 4$, or

$$\lim_{x\rightarrow 4} \frac{x^{2}-16}{x-4}.$$

 

Notice that in principle we can not simply stick in the value $x=4$ since we would then get 
$0/0$ which is not defined. However we can look at the numerator and try to factor it.

This gives us: 

$$\frac{x^{2}-16}{x-4} = \frac{(x-4)(x+4)}{x-4} = x +4$$

and the result has the obvious limit of $4+4=8$ as $x\to 4$.

\end{xmpl}
\begin{xmpl}

Consider the function

$$ g (x ) =  \frac{1}{x}$$

where $x$ is a positive real number. As $x$ increases, $g(x)$ decreases, approaching 0 but never getting there since $\frac{1}{x}=0$ has no solution. One can therefore say, “The limit of $g(x)$, as $x$ approaches infinity, is 0,” and write
$$ 
\lim_{x\to\infty} g(x)=0.
$$
\end{xmpl}


%% Slide https://tutor-web.net/math/math612.0/lecture210/slide60
\subsection{More on limits}
\begin{fbox}
\begin{minipage}{0.58\textwidth}
Limits impose a certain range of values that may be applied to the function.
\end{minipage}
\hspace{0.5mm}
\begin{minipage}{0.38\textwidth}
\begin{picture}
20
\end{picture}

Figure:  The function $ f(x)= \frac{1}{1+e^{-x}}$.
\end{minipage}
\end{fbox}
\subsubsection{Examples}
\begin{xmpl}
The Beverton-Holt stock recruitment curve is given by:
$$
R=\frac{\alpha S}{1+\frac{S}{K}}
$$

where $\alpha, K >0$ are constants and S = biomass and R= recruitment.\\

The behavior of this curve as S increases $S\rightarrow\infty$ is
$$
\lim_{S\to\infty}\frac{\alpha S}{1+\frac{S}{K}} =\alpha K .
$$
This is seen by rewriting the formula as follows:
$$
\lim_{S\to\infty}\frac{\alpha S}{1+\frac{S}{K}} =
\lim_{S\to\infty}\frac{\alpha }{\frac{1}{S}+\frac{1}{K}} =\alpha K .
$$

\end{xmpl}
\begin{xmpl}

A popular model for proportions is: 
$$
f(x) = \frac{1}{1+e^{-x}}
$$

As x increases, $e^{-x}$ decreases which implies that the
term $1+e^{-x}$ decreases and hence
$\frac{1}{1+e^{-x}}$ increases, from which it follows that $f$ is an
increasing function.

Notice that $f(0)=\frac{1}{2}$ and further, 
$$\lim_{x\to\infty} f(x) = 1.$$
This is seen from considering the 
components: \\
Since $e^{-x} = \frac{1}{e^{x}}$ and the exponential function goes to infinity as $x\to\infty$, $e^{-x}$
goes to $0$ and hence $f(x)$ goes to 1.\\

Through a similar analysis one finds that 
$$\lim_{x\to-\infty} f(x)=0 ,$$

 

since, as $x\rightarrow \infty$, first $-x\rightarrow \infty$ and second $e^{-x} \rightarrow \infty$.


\end{xmpl}
\begin{xmpl}
Evaluate the limit of $$f(x) = \frac{\sqrt{x + 4} - 2}{x}$$

as $$x \to 0$$

$$\lim_{x \to 0} \frac{\sqrt{x + 4} - 2}{x}$$

Since the square root is present we cannot just direct substitute the 0 as $x$. This will give us $\frac{0}{0}$, which is an indeterminate form. We must perform some algebra first. The way to get rid of the radical is to multiply the numerator by the conjugate.

$$\frac{\sqrt{x + 4} - 2}{x} \cdot \frac{\sqrt{x + 4} + 2}{\sqrt{x + 4} + 2}$$

This gives us $$\frac{(\sqrt{x + 4})^2 + 2(\sqrt{x+4}) - 2(\sqrt{x+4}) -4}{x(\sqrt{x + 4} + 2)}$$

The numerator reduces to $x$, and the $x$'s will cancel out leaving us with $$\frac{1}{\sqrt{x + 4} + 2}$$

At this point we can direct substitute 0 for $x$, which will give us $$\frac{1}{\sqrt{0 + 4} + 2}$$

Therefore, $$\lim_{x \to 0} \frac{\sqrt{x + 4} - 2}{x} = \frac{1}{4}$$
\end{xmpl}


%% Slide https://tutor-web.net/math/math612.0/lecture210/slide70
\subsection{One-sided limits}
\begin{fbox}
\begin{minipage}{0.58\textwidth}
$f(x)$ may tend towards different numbers depending on whether $x \rightarrow x_{0}$:

from the right ($x \rightarrow x_{0+}$) 

or from the left ($x \rightarrow x_{0-}$). 
\end{minipage}
\hspace{0.5mm}
\begin{minipage}{0.38\textwidth}
\begin{picture}
21
\end{picture}


\end{minipage}
\end{fbox}
\subsubsection{Details}
Sometimes a function is such that $f(x)$ tends to different numbers depending on whether $x \rightarrow x_0$ from the right ($x \rightarrow x_{0+}$) or from the left ($x \rightarrow x_{0-}$). \\

If 
$$\lim_{x \to x_{0+}} f(x)=f(x_0)
$$ 
then we say that $f$ is continuous from the right at $x_0$.


{\bf Copyright}
2021, Gunnar Stefansson (editor) with contributions from very many students

This work is licensed under the Creative Commons
Attribution-ShareAlike License. To view a copy of this license, visit
http://creativecommons.org/licenses/by-sa/1.0/ or send a letter to
Creative Commons, 559 Nathan Abbott Way, Stanford, California 94305,
USA.
\clearpage
%% Lecture https://tutor-web.net/math/math612.0/lecture220
\section{Sequences and series}
%% Slide https://tutor-web.net/math/math612.0/lecture220/slide10
\subsection{Sequences}
\begin{fbox}
\begin{minipage}{0.97\textwidth}
A {\bf sequence} is a string of indexed numbers $a_1, a_2, a_3, \ldots$. We denote this sequence with $(a_n)_{n\geq1}$.

\end{minipage}
\end{fbox}
\subsubsection{Details}
In a sequence the same number can appear several times in different places.
\subsubsection{Examples}
\begin{xmpl}

$(\frac{1}{n})_{n\geq1}$ is the sequence $1,\frac{1}{2}, \frac{1}{3}, \frac{1}{4}, \ldots$.
\end{xmpl}
\begin{xmpl}

$(n)_{n\geq1}$ is the sequence $1,2,3,4,5,\ldots$.
\end{xmpl}
\begin{xmpl}

$(2^nn)_{n\geq1}$ is the sequence $2,8, 24, 64,\ldots$.
\end{xmpl}

%% Slide https://tutor-web.net/math/math612.0/lecture220/slide20
\subsection{Convergent sequences}
\begin{fbox}
\begin{minipage}{0.97\textwidth}
A sequence $a_n$ is said to {\bf converge} to the number b if for every $\varepsilon >0$ we can find an $N\in \mathbb{N}$ such that $|a_n-b| < \varepsilon$ for all $n \geq N$. We denote this with $\lim_{n\to\infty}a_n=b$ or $a_n\to b$, as $n\to\infty$.
\end{minipage}
\end{fbox}
\subsubsection{Details}
A sequence $a_n$ is said to {\bf converge} to the number b if for every $\varepsilon >0$ we can find an $N\in \mathbb{N}$ such that $|a_n-b| < \varepsilon$ for all $n \geq N$. We denote this with $\lim_{n\to\infty}a_n=b$ or $a_n\to b$, as $n\to\infty$.


If x is a number then,

$ (1 + \frac{x}{n})^n \rightarrow e^x$ as $n\rightarrow\infty$




\subsubsection{Examples}
\begin{xmpl}

The sequence $(\frac{1}{n})_{n\geq\infty}$ converges to $0$ as $n\to\infty$
\end{xmpl}
\begin{xmpl}

If x is a number then,

$ (1 + \frac{x}{n})^n \rightarrow e^x$ as $n\rightarrow\infty$
\end{xmpl}

%% Slide https://tutor-web.net/math/math612.0/lecture220/slide30
\subsection{Infinite sums (series)}
\begin{fbox}
\begin{minipage}{0.97\textwidth}
We are interested in, whether infinite sums of sequences can be defined. \\



\end{minipage}
\end{fbox}
\subsubsection{Details}
Consider a sequence of numbers, $(a_n)_{n\to\infty}$. 

Now define another sequence $(s_n)_{n\to\infty},$ where 
$$s_n=\sum_{k=1}^na_k.$$

 
If $(s_n)_{n\to\infty}$ is convergent to $S=\lim_{n\to\infty}s_n,$ then we write 
$$S=\sum_{n=1}^{\infty}a_n.$$

 



\subsubsection{Examples}
\begin{xmpl}
If $$a_k = x^k, k=0,1,.....$$

\\ then 
$$s_n=\sum_{k=0}^{n}x^k=x^0+x^1+......+x^n$$
Note also that
$$xs_n=x(x^0+x^1+......+x^n)= x + x^2 + ..... + x^{n+1}$$
We have
$$s_n = 1 + x + x^2 + .... + x^n$$

$$xs_n = x + x^2 + ..... +x^n + x^{n+1}$$

$$s_n – xs_n = 1 - x^{n+1}$$
i.e.
$$s_n(1-x) = 1-x^{n+1}$$
and we have
	$$s_n =\frac{1-x^{n+1}}{1-x}$$
if $x\neq1$. 
If $0< x<1$ then $x^{n+1}\to 0$ as $n\to\infty$ and we obtain $s_n\to\frac{1}{1-x}$ so $\sum_{n=0}^{\infty}x^n=\frac{1}{1-x}$.
\end{xmpl}

%% Slide https://tutor-web.net/math/math612.0/lecture220/slide40
\subsection{The exponential function and the Poisson distribution}
\begin{fbox}
\begin{minipage}{0.97\textwidth}
The exponential function can be written as a series (infinite sum):
$$e^x=\sum_{n=0}^{\infty}\frac{x^n}{n!}.$$

 

The Poisson distribution is defined by the probabilities 
$$p(x)=e^{-\lambda}\frac{\lambda^x}{x!}\textrm{ for } x=0,\ 1,\ 2,\ \ldots$$

 

\end{minipage}
\end{fbox}
\subsubsection{Details}
The exponential function can be written as a series (infinite sum):
$$e^x=\sum_{n=0}^{\infty}\frac{x^n}{n!}.$$

 

Knowing this we can see why the Poisson probabilities 
$$p(x)=e^{-\lambda}\frac{\lambda^x}{x!}$$

 
add to one: 
$$\sum_{x=0}^{\infty}p(x)=\sum_{x=0}^{\infty}e^{-\lambda}\frac{\lambda^x}{x!}=e^{-\lambda}\sum_{x=0}^{\infty}\frac{\lambda^x}{x!}=e^{-\lambda}e^{\lambda}=1.$$



%% Slide https://tutor-web.net/math/math612.0/lecture220/slide65
\subsection{Relation to expected values}
\begin{fbox}
\begin{minipage}{0.97\textwidth}
The expected value for the Poisson is given by 

\begin{eqnarray*}
\sum_{x=0}^\infty x p(x) &=& \sum_{x=0}^\infty x e^{-\lambda} \frac{\lambda^x}{x!} \\
                         &=& \lambda
\end{eqnarray*}

\end{minipage}
\end{fbox}
\subsubsection{Details}
The expected value for the Poisson is given by 

\begin{eqnarray*}
\sum_{x=0}^\infty x p(x) &=& \sum_{x=0}^\infty x e^{-\lambda} \frac{\lambda^x}{x!} \\
                         &=& e^{-\lambda} \sum_{x=1}^\infty   \frac{x\lambda^x}{x!} \\
                         &=& e^{-\lambda} \sum_{x=1}^\infty   \frac{\lambda^x}{(x-1)!} \\
                         &=& e^{-\lambda} \lambda \sum_{x=1}^\infty   \frac{\lambda^{(x-1)}}{(x-1)!} \\
                         &=& e^{-\lambda} \lambda \sum_{x=0}^\infty   \frac{\lambda^{x}}{x!} \\
                         &=& e^{-\lambda} \lambda  e^{\lambda}\\
                         &=& \lambda
\end{eqnarray*}


{\bf Copyright}
2021, Gunnar Stefansson (editor) with contributions from very many students

This work is licensed under the Creative Commons
Attribution-ShareAlike License. To view a copy of this license, visit
http://creativecommons.org/licenses/by-sa/1.0/ or send a letter to
Creative Commons, 559 Nathan Abbott Way, Stanford, California 94305,
USA.
\clearpage
%% Lecture https://tutor-web.net/math/math612.0/lecture230
\section{Slopes of lines and curves}
%% Slide https://tutor-web.net/math/math612.0/lecture230/slide10
\subsection{The slope of a line}
\begin{fbox}
\begin{minipage}{0.58\textwidth}
Linear functions produce straight-line graphs. In general, a straight line follows the following equation:
$$ 
y = a + bx ,
$$
where $a$ and $b$ are fixed numbers.

The line on the graph is the set of points:
$$
\left \\{ (x,y):   x,y \in \mathbb{R}, y = a+bx\right \\} .
$$

\end{minipage}
\hspace{0.5mm}
\begin{minipage}{0.38\textwidth}
\begin{picture}
22
\end{picture}


\end{minipage}
\end{fbox}
\subsubsection{Details}
The slope of a straight line represents the change in the $y$ coordinate corresponding to a unit change in the $x$ coordinate.


%% Slide https://tutor-web.net/math/math612.0/lecture230/slide20
\subsection{Segment slopes}
\begin{fbox}
\begin{minipage}{0.58\textwidth}
Let’s assume we have a more general function

$ y = f(x)$.\\

To find the slope of a line segment, consider 2 $x$-coordinates, $x_0$ and $x_1$, and look at the slope between $(x_0, f(x_0))$ and  $(x_1, f(x_1))$.

\end{minipage}
\hspace{0.5mm}
\begin{minipage}{0.38\textwidth}
\begin{picture}
23
\end{picture}


\end{minipage}
\end{fbox}
\subsubsection{Details}
Consider two points, $(x_0,y_0)$ and $(x_1,y_1)$. The slope of the straight line that goes through these points is
$$ \frac {y_1 - y_0} {x_1 - x_0} .$$

Thus, the slope of a line segment passing throught the points $(x_0,f(x_0))$ and $(x_1,f(x_1))$, for some function, $f$, is 
$$ \frac {f(x_1) - f(x_0)} {x_1 - x_0}$$

If we let $x_1 = x_0 + h$ then the slope of the segment is
$$ \frac {f(x_0+h) - f(x_0)} {h} .$$


%% Slide https://tutor-web.net/math/math612.0/lecture230/slide30
\subsection{The slope of $y=x^2$}
\begin{fbox}
\begin{minipage}{0.58\textwidth}
Consider the task of computing the slope of the function $y=x^2$ at a given point. 
\end{minipage}
\hspace{0.5mm}
\begin{minipage}{0.38\textwidth}
\begin{picture}
24
\end{picture}


\end{minipage}
\end{fbox}
\subsubsection{Examples}
Consider the function $y = f(x) = x^2 $.\\ 

In order to find the slope at a given point $(x_0 )$, we look at 
$$ y = \frac{f (x_0 +h) - f(x_0)} {h}$$

 
for small values of $h$.\\

For this particular function, $ f (x) = x^2 $, and hence
$$ 
f (x_0 +h) = (x_0 +h) ^2  = x^2 + 2hx_0 + h^2 .
$$ 

The slope of a line segment is therefore given by
$$
\frac{f (x_0 +h) - f(x_0)} {h}= \frac{2hx_0 + h^2} {h} = 2x_0 + h .
$$

As we make $h$ steadily smaller, the segment slope, $2x_0 + h$, tends towards $2x_0 $. It follows that the slope, $y'$, of the curve {\em at a general point} $x$ is given by $ y' = 2x $.

%% Slide https://tutor-web.net/math/math612.0/lecture230/slide40
\subsection{The tangent to a curve}
\begin{fbox}
\begin{minipage}{0.58\textwidth}
A {\bf tangent} to a curve is a line that intersects the curve at exactly one point. The slope of a tangent for the function $y=f(x)$ at the point $(x_0,f(x_0))$ is $$\lim_{h\to0}\frac{f(x_0+h)-f(x_0)}{h}.$$

 
\end{minipage}
\hspace{0.5mm}
\begin{minipage}{0.38\textwidth}
\begin{picture}
25
\end{picture}


\end{minipage}
\end{fbox}
\subsubsection{Details}
To find the slope of the tangent to a curve at a point, we look at the slope of a line segment between the points $(x_0,f(x_0))$ and $(x_0+h,f(x_0+h))$, which is $$\frac{f(x_0+h)-f(x_0)}{h}$$

and then we take $h$ to be closer and closer to $0$. Thus the slope is $$\lim_{h\to0}\frac{f(x_0+h)-f(x_0)}{h}$$

when this limit exists. 
\subsubsection{Examples}
\begin{xmpl}

We wish to find tangent line for the function $f(x)=x^2$ at the point $(1,1)$. First we need to find the slope of this tangent, it is given as $$\lim_{h\to0}\frac{(1+h)^2-1^2}{h}=\lim_{h\to0}\frac{2h+h^2}{h}=\lim_{h\to0}(2+h)=2.$$

Then, since we know the tangent goes through the point $(1,1)$ the line is $y=2x-1$.
\end{xmpl}

%% Slide https://tutor-web.net/math/math612.0/lecture230/slide50
\subsection{The slope of a general curve}
\begin{fbox}
\begin{minipage}{0.58\textwidth}



\end{minipage}
\hspace{0.5mm}
\begin{minipage}{0.38\textwidth}
\begin{picture}
26
\end{picture}


\end{minipage}
\end{fbox}
\subsubsection{Details}
Imagine a nonlinear function whose graph is a curve described by the equation, 

$y = f(x) $.\\ 

Here we want to find the slope of a line tangent to the curve at a specific point $(x_0)$.


The slope of the line segment is given by the equation
$  \frac{f (x_0 +h) - f(x_0)} {h}$.\\ 

Reducing $h$ towards zero, gives the slope of this curve if it exists.



{\bf Copyright}
2021, Gunnar Stefansson (editor) with contributions from very many students

This work is licensed under the Creative Commons
Attribution-ShareAlike License. To view a copy of this license, visit
http://creativecommons.org/licenses/by-sa/1.0/ or send a letter to
Creative Commons, 559 Nathan Abbott Way, Stanford, California 94305,
USA.
\clearpage
%% Lecture https://tutor-web.net/math/math612.0/lecture240
\section{Derivatives}
%% Slide https://tutor-web.net/math/math612.0/lecture240/slide10
\subsection{The derivative as a limit}
\begin{fbox}
\begin{minipage}{0.97\textwidth}
The derivative of the function $f$ at the point $x$ is defined as
$$\lim_{h \to 0} \frac{f(x+h) - f(x)}{h}$$

if this limit exists.


\end{minipage}
\end{fbox}
\subsubsection{Details}
\begin{defn}
The derivative of the function f at the point x is defined as
$$\lim_{h \to 0} \frac{f(x+h) -f(x)}{h}$$

if this limit exists.
\end{defn}
When we write $y = f(x)$, we commonly use the notation 
$\frac{dy}{dx}$ or $f'(x)$ for this limit.

%% Slide https://tutor-web.net/math/math612.0/lecture240/slide20
\subsection{The derivative of $f(x)=a+bx$}
\begin{fbox}
\begin{minipage}{0.58\textwidth}
If $f(x) = a + bx$ then $f(x + h) = a+ b(x + h) = a + bx + bh$ and thus 
$$\lim_{h \to 0} \frac{f(x+h)-f(x)}{h} = \lim_{h \to 0} \frac{bh}{h}=b$$
\end{minipage}
\hspace{0.5mm}
\begin{minipage}{0.38\textwidth}
\begin{picture}
27
\end{picture}


\end{minipage}
\end{fbox}
\subsubsection{Details}
If $f(x) = a + bx$ then $f(x + h) = a+ b(x + h) = a + bx + bh$ and thus 
$$\lim_{h \to 0} \frac{f(x+h)-f(x)}{h} = \lim_{h \to 0} \frac{bh}{h}=b.$$

Thus $f'(x)=b$.

%% Slide https://tutor-web.net/math/math612.0/lecture240/slide30
\subsection{The derivative of  $f(x)=x^n$}
\begin{fbox}
\begin{minipage}{0.97\textwidth}
If $f(x)=x^n$, then  $f'(x)=nx^{n-1}$.
\end{minipage}
\end{fbox}
\subsubsection{Details}
Let $f(x)=x^n$, where $n$ is a positive integer. To calculate $f'$ we use the binomial theorem in the third step:

\begin{align*}
\frac{f(x+h)-f(x)}{h}&=\frac{(x+h)^n-x^n}{h}\\
&=\frac{\sum_{q=0}^{n-1}\binom{n}{q}x^qh^{n-q}}{h}\\
&=\sum_{q=0}^{n-1}\binom{n}{q}x^qh^{n-q-1}\to\binom{n}{n-1}x^{n-1}=nx^{n-1}
\end{align*}

Thus, we obtain $f'(x)=nx^{n-1}$.


%% Slide https://tutor-web.net/math/math612.0/lecture240/slide40
\subsection{The derivative of ln and exp}
\begin{fbox}
\begin{minipage}{0.97\textwidth}
If
	$$f(x)  = e^x$$
then
	$$f'(x) = e^x$$

If	
	$$g(x) = \ln(x)$$
then
	$$g'(x) = \frac{1}{x}$$

\end{minipage}
\end{fbox}
\subsubsection{Details}
The derivatives of the  exponential function is the exponential  function itself i.e. \\ if
	$$f(x)  = e^x$$
then
	$$f'(x) = e^x$$


The derivatives of the natural logarithm, $\ln(x)$, is $\frac{1}{x}$, i.e.
if	
	$$g(x) = \ln(x)$$
then
	$$g'(x) = \frac{1}{x}$$


%% Slide https://tutor-web.net/math/math612.0/lecture240/slide50
\subsection{The derivative of a sum and linear combination}
\begin{fbox}
\begin{minipage}{0.97\textwidth}
If $f$ and $g$ are functions then the derivative of $f+g$ is given by $f' + g'$. 
\end{minipage}
\end{fbox}
\subsubsection{Details}
Similarly, the derivative of a linear combination is the linear combination of the derivatives.

If $f$ and $g$ are functions and $k(x)=af(x) + bg(x)$ then $k'(x)=af'(x)+ bg'(x)$.
\subsubsection{Examples}
\begin{xmpl}
If $f(x) = 2+3x$ and $g(x)+x^3$ \\
then we know that\\ 
$f'(x)=3$, $g(x)=3x^2$
 and if we write 
$$h(x)=f(x)+g(x)=2+3x+x^3$$
 then 
$$h'(x)=3+3x^2$$
\end{xmpl}

%% Slide https://tutor-web.net/math/math612.0/lecture240/slide60
\subsection{The derivative of a polynomial}
\begin{fbox}
\begin{minipage}{0.97\textwidth}
The derivative of a polynomial is the sum of the derivatives of the terms of the polynomial.
\end{minipage}
\end{fbox}
\subsubsection{Details}
If 

$p(x)=a_0+a_1x+...+a_n x^n$

then

$p'(x)=a_1+2a_2x+3a_3x^2+4a_4x^3+...+na_n x^{(n-1)}$
\subsubsection{Examples}
\begin{xmpl}
If

$p(x)=2x^4+x^3$

then

$p'(x)=2\frac{dx^4}{dx}+\frac{dx^3}{dx}=2 \cdot 4x^3 +3x^2 = 8x^3 +3x^2$
\end{xmpl}

%% Slide https://tutor-web.net/math/math612.0/lecture240/slide70
\subsection{The derivative of a product}
\begin{fbox}
\begin{minipage}{0.97\textwidth}
If
$$
h(x)=f(x)\cdot g(x)
$$
then
$$
h'(x)=f'(x)\cdot g(x)+f(x)\cdot g'(x)
$$

\end{minipage}
\end{fbox}
\subsubsection{Details}
Consider two functions, $f$ and $g$ and their product, $h$:
$$
h(x)=f(x)\cdot g(x).
$$
The derivative of the product is given by
$$
h'(x)=f'(x)\cdot g(x)+f(x)\cdot g'(x).
$$

\subsubsection{Examples}
\begin{xmpl}
Suppose the function $f$ is given by
$$
f(x)=xe^x+x^2\ln x .
$$
Then the derivative can be computed step by step as
\begin{eqnarray*}
f(x)&=&\frac{dx}{dx}e^x+x\frac{de^x}{dx}+\frac{dx^2}{dx}\ln x +x^2\frac{d \ln x}{dx}\\
    &=&1\cdot e^x +     x \cdot e^x     + 2x \cdot \ln x     + x^2 \cdot \frac{1}{x}\\
    &=&e^x \left ( 1+x \right ) + 2x \ln  x +x
\end{eqnarray*}
\end{xmpl}


%% Slide https://tutor-web.net/math/math612.0/lecture240/slide80
\subsection{Derivatives of composite functions}
\begin{fbox}
\begin{minipage}{0.97\textwidth}
If $f$ and $g$ are functions and $h=f  \circ g$ so that\\

$h(x) = f(g(x))$ then \\

$h'(x) = \frac{dh(x)}{dx} = f'(g(x)) g'(x)$
\end{minipage}
\end{fbox}
\subsubsection{Examples}
\begin{xmpl}
For fixed $x$ consider:
	
	\begin{eqnarray*}
	f(p) &=& \ln(p^{x} (1-p)^{n-x})\\
	     &=& \ln p^{x} + \ln(1-p)^{n-x}\\%ln of a product is the sum of ln
	     &=& x \ln p + (n-x) \ln (1-p)\\
	\end{eqnarray*}
	
	\begin{eqnarray*}
  f'(p)&=& x \frac{1}{p} + \frac{n-x}{1-p}(-1)\\
	     &=& \frac{x}{p} - \frac{n-x}{1-p}\\
	\end{eqnarray*}

\end{xmpl}
\begin{xmpl}	
	$f(b) = (y-bx)^2$ ($y,x$ fixed)\\
	
	\begin{eqnarray*}
	f'(b)&=& 2 (y-bx) (-x)\\
	     &=& -2x (y-bx)\\
	     &=&(-2xy) + (2x^2)b   
	\end{eqnarray*}
\end{xmpl}	
	



{\bf Copyright}
2021, Gunnar Stefansson (editor) with contributions from very many students

This work is licensed under the Creative Commons
Attribution-ShareAlike License. To view a copy of this license, visit
http://creativecommons.org/licenses/by-sa/1.0/ or send a letter to
Creative Commons, 559 Nathan Abbott Way, Stanford, California 94305,
USA.
\clearpage
%% Lecture https://tutor-web.net/math/math612.0/lecture250
\section{Applications of differentiation}
%% Slide https://tutor-web.net/math/math612.0/lecture250/slide10
\subsection{Tracking the sign of the derivative}
\begin{fbox}
\begin{minipage}{0.97\textwidth}
If $f$ is a function, then the sign of its derivative, $f'$, indicates whether $f$ is increasing ($f'>0$), decreasing ($f'<0$), or zero. $f'$ can be zero at points where $f$ has a maximum, minimum, or a saddle point.

\end{minipage}
\end{fbox}
\subsubsection{Details}
If $f$ is a function, then the sign of its derivative, $f'$, indicates whether $f$ is increasing ($f'>0$), decreasing ($f'<0$), or zero. $f'$ can be zero at points where $f$ has a maximum, minimum, or a saddle point.\\

If $f'(x)>0$ for $x< x_0$, $f'(x_0)=0$ and $f'(x)<0$ for $x>x_0$ then $f$ has  a maximum at $x_0$ 

If $f'(x)<0$ for $x< x_0$, $f'(x_0)=0$ and $f'(x)>0$ for $x>x_0$ then  $f$ has   a minimum at $x_0$ 

If $f'(x)>0$ for $x< x_0$, $f'(x_0)=0$ and $f'(x)>0$ for $x< x_0$ then  $f$ has   a saddle point at $x_0$ 

If $f'(x)<0$ for $x< x_0$, $f'(x_0)=0$ and $f'(x)<0$ for $x< x_0$ then  $f$ has   a saddle point at $x_0$

\subsubsection{Examples}
\begin{xmpl}
If $f$ is a function such that its derivative is given by
$$
f'(x) = (x-1)(x-2)(x-3)(x-4),$$
then applying the above criteria for maxima and minima, we see that $f$ has maxima at $1$ and $3$ and $f$ has minima at $2$ and $4$.
\end{xmpl}

%% Slide https://tutor-web.net/math/math612.0/lecture250/slide20
\subsection{Describing extrema using $f''$}
\begin{fbox}
\begin{minipage}{0.97\textwidth}
$x_0$ with $f'(x_0)=0$ corresponds to a maximum if  $f''(x_0)<0$

$x_0$ with $f'(x_0)=0$ corresponds to a minimum if  $f''(x_0)>0$


\end{minipage}
\end{fbox}
\subsubsection{Details}
If $f'(x_0)=0$ corresponds to a maximum, then the derivative is decreasing and the second derivative can not be positive, (i.e. $f''(x_0)\leq 0$).  In particular, if the second derivative is strictly negative, ($f''(x_0) <0$), then we are assured that the point is indeed a maximum, and not a saddle point.\\

If $f'(x_0)=0$ corresponds to a minimum, then the derivative is increasing and the second derivative can not be negative, (i.e. $f''(x_0) \geq 0$).\\

If the second derivative is zero, then the point may be a saddle point, as happens with $f(x)=x^3$ at $x=0$.



%% Slide https://tutor-web.net/math/math612.0/lecture250/slide30
\subsection{The likelihood function}
\begin{fbox}
\begin{minipage}{0.97\textwidth}
If $p$ is the probability mass function (p.m.f.):
$$ p(x) = P [X = x] $$

then the joint probability of obtaining a sequence of outcomes from independent sampling is

$$ p(x_1) \cdot p(x_2) \cdot p(x_3) \ldots p(x_n)$$

Suppose each probability includes some parameter $\theta$, this is written,

$$ {p_{\theta}}(x_1),  \ldots {p_{\theta}}(x_n)$$

 

If the experiment gives $ x_1, x_2 \ldots, x_n$ we can write the probability as a function of the parameters:

$$ L_{\mathbf{x}}(\theta) = p_{\theta}(x_1),  \ldots p_{\theta}(x_n).$$

This is the {\emph likelihood function}.

\end{minipage}
\end{fbox}
\subsubsection{Details}
\begin{defn}
Recall that the {\bf probability mass function (p.m.f)} is a function giving the probability of outcomes of an experiment.
\end{defn}
We typically denote the p.m.f. by $p$ so $p(x)$ gives the probability of a given outcome, $x$, of an experiment. The p.m.f.  commonly depends on some parameter. We often write,

$$ p(x) = P [X = x].$$

If we take a sample of independent measurements, from $p$, then the joint probability of a given set of numbers is, 

$$ p(x_1) \cdot p(x_2) \cdot p(x_3) \ldots p(x_n)$$

 

Suppose each probability includes the same parameter $\theta$, then this is typically written,

$$ {p_{\theta}}(x_1),  \ldots {p_{\theta}}(x_n)$$

 

Now consider the set of outcomes $ x_1, x_2 \ldots, x_n$ from the experiment. We can now take the probability of this outcome as a function of the parameters.
\begin{defn}
$ L_{\mathbf{x}}(\theta) = p_{\theta}(x_1),  \ldots p_{\theta}(x_n)$ 

This is the \textbf{likelihood function} and we often seek to maximize it to estimate the unknown parameters.
\end{defn}
\subsubsection{Examples}
\begin{xmpl}
Suppose we toss a biased coin $n$ independent times and obtain x heads, we know the probability of obtaining x heads is,

$$\binom{n}{x}p^x (1-p)^{n-x}$$

The parameter of interest is $p$ and the likelihood function is,

$$ L(p) = \binom{n}{x}p^x (1-p)^{n-x}$$

If $p$ is unknown we sometimes wish to maximize this function with respect to $p$ in order to estimate the \emph{true} probability $p$.
\end{xmpl}

%% Slide https://tutor-web.net/math/math612.0/lecture250/slide40
\subsection{Plotting the likelihood}
\begin{fbox}
\begin{minipage}{0.97\textwidth}
missing slide -- want to give a numeric example and plot $L$
\end{minipage}
\end{fbox}
\subsubsection{Examples}
missing example -- want to give a numeric example and plot $L$

%% Slide https://tutor-web.net/math/math612.0/lecture250/slide50
\subsection{Maximum likelihood estimation}
\begin{fbox}
\begin{minipage}{0.97\textwidth}
If L is a likelihood function for a p.m.f. $ p_{\theta}$, then the value $\hat{\theta}$ which gives the maximum of L:
$$ 
L (\hat{\theta}) = \max_\theta ({L}_\theta) 
$$
is the maximum likelihood estimator (MLE) of $\theta$

\end{minipage}
\end{fbox}
\subsubsection{Details}
\begin{defn}
If L is a likelihood function for a p.m.f. $ p_{\theta}$, then the value $\hat{\theta}$ which gives the maximum of L:
$$ 
L (\hat{\theta}) = \max_\theta ({L}_\theta) 
$$
is the {\bf maximum likelihood estimator} of $\theta$
\end{defn}

\subsubsection{Examples}
\begin{xmpl}
If $x$ is the number of heads from $n$ independent tosses of a coin, the likelihood function is:

$$ 
L_x(p) = {n \choose x} p^x (1-p)^{n-x} 
$$

Maximizing this is equivalent to maximizing the logarithm of the likelihood, since logarithmic functions are increasing. The log-likelihood can be written as:

$$
\ln (L(p))= \ln \binom{n}{x} + x \ln (p) + (n-x) \ln (1-p).
$$

To find possible maxima , we need to differentiate this formula and set the derivative to zero

$$ 0 = \frac{dl (p)}{dp} = 0 + \frac{x}{p}+\frac{n-x}{1-p}(-1)$$

$$ 0 = p(1-p) \frac{(x)}{p} - p(1-p)  \frac{n-x}{1-p}$$

$$ 0 = (1-p)x  - p(n-x)  $$

$$ 0 = x  - px -pn + px = x-pn  $$

So,

$$ 0 =  x-pn  $$

$$ p = \frac{x}{n} $$

is the extreme and so we can write

$$ \hat{p} = \frac{x}{n}$$

for the MLE
\end{xmpl}

%% Slide https://tutor-web.net/math/math612.0/lecture250/slide60
\subsection{Least squares estimation}
\begin{fbox}
\begin{minipage}{0.97\textwidth}
Least squares:  Estimate the parameters $\theta$ by minimizing 
$$
\sum_{i=1}^{n}{(y_i - g_i (\theta))^2} 
$$

\end{minipage}
\end{fbox}
\subsubsection{Details}
Suppose we have a model linking data to parameters. In general we are predicting $y_i$ as $g_i$ ($\theta$).

In this case it makes sense to estimate parameters $\theta$ by minimizing 
$$
\sum_{i=1}^{n}{(y_i - g_i (\theta))^2} .
$$

\subsubsection{Examples}
\begin{xmpl}
One may predict numbers, $x_i$, as a mean, $\mu$, plus error.  Consider the simple
model $x_i = \mu + \epsilon_i$, where 
$\mu$ is an unknown parameter (constant) and $\epsilon_i$ is the error in measurement
when obtaining the $i$'th observations, $x_i$, $i=1,\ldots , n$.\\

A natural method to estimate the parameter 
is to minimize the squared deviations
$$
\min_{\mu} \sum_{i=1}^n \left (x - \mu \right )^2  .
$$

It is not hard to see that the $\hat{\mu}$ that minimizes this is the mean:
$$
\hat{ \mu} = \bar{x} .
$$
\end{xmpl}
\begin{xmpl}
One also commonly  predicts data $y_1 , \cdots ,y_n$ with values on a straight line, i.e. with $\alpha + \beta x_i$, where $x_1, \ldots , x_n$ are fixed numbers. \\

This leads to the {\emph regression problem} of finding parameter values for $\hat{\alpha}$ and $\hat{\beta}$ which gives the best fitting straight line in relation to least squares:

$$
\min_{\alpha,\beta} \sum \left ( y_i - ( \alpha + \beta x_i) \right ) ^2
$$ 
\end{xmpl}

\begin{xmpl}
As a general exercise in finding the extreme of a function, let's look at the function $f(\theta)=\sum_{i=1}^n(x_i\theta -3)^2$ where $x_i$ are some constants. We wish to find the $\theta$ that minimizes this sum. We simply differentiate $\theta$ to obtain $f'(\theta)=\sum_{i=1}^n2(x_i\theta -3)x_1=2\sum_{i=1}^n x^2_i\theta -2\sum_{i=1}^n3x_i$. Thus,
\begin{align*}
f'(\theta)&=2\theta \sum_{i=1}^n x^2_i-2\sum_{i=1}^n3x_i=0\\
&\Leftrightarrow \theta=\frac{\sum_{i=1}^n3x_i}{\sum_{i=1}^n x^2_i}.
\end{align*}
\end{xmpl}

{\bf Copyright}
2021, Gunnar Stefansson (editor) with contributions from very many students

This work is licensed under the Creative Commons
Attribution-ShareAlike License. To view a copy of this license, visit
http://creativecommons.org/licenses/by-sa/1.0/ or send a letter to
Creative Commons, 559 Nathan Abbott Way, Stanford, California 94305,
USA.
\clearpage
%% Lecture https://tutor-web.net/math/math612.0/lecture260
\section{Integrals and probability density functions}
%% Slide https://tutor-web.net/math/math612.0/lecture260/slide10
\subsection{Area under a curve}
\begin{fbox}
\begin{minipage}{0.58\textwidth}
The area under a curve between x=a and x=b (for a positive function) is called the integral of the function.
\end{minipage}
\hspace{0.5mm}
\begin{minipage}{0.38\textwidth}
\begin{picture}
28
\end{picture}

Figure:  Example 1, 2 and 3
\end{minipage}
\end{fbox}
\subsubsection{Details}
\begin{defn}
The area under a curve between x=a and x=b (for a positive function) is called the {\bf integral of the function} and is denoted:
$\int_{a}^{b} f(x)dx$ when it exists.
\end{defn}

%% Slide https://tutor-web.net/math/math612.0/lecture260/slide20
\subsection{The antiderivative}
\begin{fbox}
\begin{minipage}{0.97\textwidth}
Given a function $f$, if there is another function $F$ such that $F'=f$, we say that $F$ is the {\em antiderivative} of $f$. For a function $f$ the antiderivative is denoted by $\int f dx$.

Note that if $F$ is one antiderivative of $f$ and $C$ is a constant, then $G=F+C$ is also an antiderivative. It is therefore customary to always include the constant, e.g. $\int x dx=\frac{1}{2}x^2+C$.
 
\end{minipage}
\end{fbox}
\subsubsection{Examples}
\begin{xmpl}
The antiderivative of $x$ to a power raises the power. 
$\int x^n dx=\frac{1}{n+1}x^{n+1} +C$.
\end{xmpl}
\begin {xmpl}

$\int e^x dx=e^x+C$.

\end{xmpl}
\begin{xmpl}

$\int \frac{1}{x} dx=\ln(x)+C$.


\end{xmpl}
\begin{xmpl}

$\int 2xe^{x^2} dx=e^{x^2}+C$.
\end{xmpl}

%% Slide https://tutor-web.net/math/math612.0/lecture260/slide21
\subsection{The fundamental theorem of calculus}
\begin{fbox}
\begin{minipage}{0.97\textwidth}
If $f$ is a continuous function, and $F'(x)=f(x)$ for $x\in[a,b]$, then $\int_a^b f(x)dx=F(b)-F(a)$
\end{minipage}
\end{fbox}
\subsubsection{Detail}
It is not too hard to see that the area under the graph of a positive function $f$ on the interval $[a,b]$ must be equal to the difference of the values of its antiderivative at $a$ and $b$. This also holds for functions which take on negative values and is formally stated below.
\begin{defn}
\textbf{Fundamental theorem of calculus:}
If $F$ is the antiderivative of the continuous function $f$, i.e. $F'=f$ for $x\in[a,b]$, then $\int_a^b f(x)dx=F(b)-F(a)$. 

This difference is often written as $\int_a^b f dx$ or $[F(x)]_a ^b$.
\end{defn}
\subsubsection{Examples}
\begin{xmpl}

The area under the graph of $x^n$ between $0$ and $3$ is $\int_0^3 x^n dx = [\frac{1}{n+1}x^{n+1}]_0 ^3=\frac{1}{n+1}3^{n+1}-\frac{1}{n+1}0^{n+1}=\frac{3^{n+1}}{n+1}$
\end{xmpl}
\begin{xmpl}
The area under the graph of $e^x$ between $3$ and $4$ is $\int_3^4 e^x dx =[e^x]_3 ^4= e^4-e^3$

\end{xmpl}
\begin{xmpl}

The area under the graph of $\frac{1}{x}$ between $1$ and $a$ is $\int_1^a \frac{1}{x} dx =[\ln(x)]_1 ^a= \ln(a)-\ln(1)=\ln(a).$
\end{xmpl}


%% Slide https://tutor-web.net/math/math612.0/lecture260/slide50
\subsection{Density functions}
\begin{fbox}
\begin{minipage}{0.58\textwidth}
The probability density function (p.d.f.) and the cumulative distribution function (c.d.f.).
\end{minipage}
\hspace{0.5mm}
\begin{minipage}{0.38\textwidth}
\begin{picture}
29
\end{picture}


\end{minipage}
\end{fbox}
\subsubsection{Details}
\begin{defn}
If $X$ is a random variable such that \\
$$P(a\leq X\leq b)=\int\limits^{b}_{a}f(x)dx,$$

\\
for some function $f$ which satisfies $f(x)\geq0$ for all $x$ and\\
$$\int\limits^\infty_{-\infty} f(x)dx = 1$$

\\
then $f$ is said to be a {\bf probability density function (p.d.f.)} for $X$.
\end{defn}
\begin{defn}
The function
$$ F(x)= \int\limits^{x}_{-\infty} f(t)dt$$

is the {\bf cumulative distribution function (c.d.f.)}. 
\end{defn}
\subsubsection{Examples}
\begin{xmpl}
Consider a random variable $X$ from the uniform distribution, denoted by $X\sim U(0,1)$.  This distribution has density

$$
f(x) = 
\begin{cases}
  1 &\text{if } 0 \leq x \leq 1\\
  0 &\text{e.w.}
\end{cases}.
$$

The cumulative distribution function is given by

$$
P[X\leq x] = \int\limits^{x}_{-\infty} f(t)dt = 
\begin{cases}
  0 & \text{if } x<0\\
  x & \text{if } 0 \leq x \leq 1\\
  1 & \text{else}
\end{cases}.
$$

\end{xmpl}

\begin{xmpl}
Suppose $X \sim P(\lambda)$, where X may denote the number of events per unit time. The p.m.f. of X is described by $p(x)=P[X=x]=e^{-\lambda}\frac{\lambda^x}{x!}$ for $x=0,1,2,...$. Consider now the waiting time, T, between events, or simply until the first event. Consider the event $T>t$ for some number t>0. If $X\sim p(\lambda)$ denotes the number of events per unit time, then let $X_t$ denote the number of events during the time period for 0 through t. Then it is natural to assume

$X_t \sim P(\lambda t)$ and it follows that $T>t$ if and only if $X_t=0$ and we obtain $P[T>t]=P[X_t=0]=e^{-\lambda t}$.
It follows that the c.d.f. of T is $F_T(t)=P[T\leq t]=1-P[T>t]=1-e^{-\lambda t}$ for $t>0$.\\

The p.d.f. of T is therefore $f_T(t)=F_T'(t)=\frac{d}{dt}F_T(t)=\frac{d}{dt}(1-e^{-\lambda t}=0-e^{- \lambda t}*(-\lambda)=\lambda e^{-\lambda t}$ for $t \geq 0$ and $f_T(t)=0$ for $t=0$. \\

The resulting density 

$$f(t) =
\begin{cases}
  \lambda e^{-\lambda t} & \text{for}& t \geq0\\
  0 & \text{for} & t<0
\end{cases}.
$$

describes the exponential distribution.

This distribution has the expected value
$$E[T]=\int\limits^{\infty}_{-\infty} tf(t)dt=\int\limits^{\infty}_{0} t \lambda e^{-\lambda t}dt.$$

\textbf{the stuff below is all messed up...}

We set $u=\lambda t$ and $du=\lambda dt$ to obtain
$$\int ue^{-u}du= \frac{1}{\lambda}\int\limits^{\infty}_{0} ue^{-u}du=\frac{1}{\lambda}=
\int\limits^{\infty}_{0} 1 \cdot e^{-u}du$$

$$=\left [ -ue^{-u}\right ] _0^{\infty}$$

$$=\left[\frac{1}{\lambda}(-e^{-u})\right]_{0}^{\infty} -0=\frac{1}{\lambda} .$$
\end {xmpl}


%% Slide https://tutor-web.net/math/math612.0/lecture260/slide60
\subsection{Probabilities in R: The normal distribution}
\begin{fbox}
\begin{minipage}{0.97\textwidth}
R has functions to compute values of probability density functions (p.d.f.) and cumulative distribution functions (c.m.d.) for most common distributions.  
\end{minipage}
\end{fbox}
\subsubsection{Details}
The p.d.f. for the normal distribution is


$$p(t)=\frac{1}{\sqrt{2\pi}}e^{-\frac{t^2}{2}}$$

The c.d.f. for the normal distribution is


$$\Phi(x)=\int\limits_{-\infty}^x\frac{1}{\sqrt{2\pi}}e^{-\frac{t^2}{2}}dt$$



\subsubsection{Examples}
\begin{xmpl}
dnorm() gives the value of the normal p.d.f.
\end{xmpl}
\begin{xmpl}
pnorm() gives the value of the normal c.d.f.
\end{xmpl}

%% Slide https://tutor-web.net/math/math612.0/lecture260/slide80
\subsection{Some rules of integration}
\subsubsection{Examples}
\begin{xmpl}

Using integration by parts we obtain 
$$\int \ln(x)x dx= \frac{1}{2}x^2\ln(x)-\int \frac{1}{2}x^2\cdot \frac{1}{x} dx = \frac{1}{2}x^2\ln(x)-\int \frac{1}{2}x dx=\frac{1}{2}x^2\ln(x)-\frac{1}{4}x^2.$$
\end{xmpl}

\begin{xmpl}
Consider $\int_1^2 2xe^{x^2} dx$. By setting $x=g(t)=\sqrt{t}$ we obtain 
$$\int_1^2 2xe^{x^2} dx = \int_1^4 2\sqrt{t}e^{t}\frac{1}{2\sqrt{t}}dt=\int_1^4 e^t dt=e^4-e.$$
\end{xmpl}
\subsubsection{Handout}
The two most common "tricks" applied in integration are a) integration by parts and b) integration by substitution.\\

a) {\bf Integration by parts}\\

$$(fg)'=f'g+fg'$$
by integrating both sides of the equation we obtain:
$$ fg=\int f'g dx + \int fg' dx \Leftrightarrow \int fg' dx=fg-\int f'g dx$$

\\

b) {\bf Integration by substitution}\\

Consider the definite integral $\int_a^b f(x) dx$ and let $g$ be a one-to-one differential function for the interval $(c,d)$ to $(a,b)$. Then
$$\int_a^b f(x) dx=\int_c^d f(g(y))g'(y) dy$$

{\bf Copyright}
2021, Gunnar Stefansson (editor) with contributions from very many students

This work is licensed under the Creative Commons
Attribution-ShareAlike License. To view a copy of this license, visit
http://creativecommons.org/licenses/by-sa/1.0/ or send a letter to
Creative Commons, 559 Nathan Abbott Way, Stanford, California 94305,
USA.
\clearpage
%% Lecture https://tutor-web.net/math/math612.0/lecture270
\section{Principles of programming}
%% Slide https://tutor-web.net/math/math612.0/lecture270/slide10
\subsection{Modularity}
\begin{fbox}
\begin{minipage}{0.97\textwidth}
Modularity involves designing a system that is divided into a set of functional units (named modules) that can be composed into a larger application.  \\

Any programming project should be split into logical module pieces of code which are combined into a complete program.


\end{minipage}
\end{fbox}
\subsubsection{Details}
Typically input, initialization, analysis, and output commands are grouped into separate parts.
\subsubsection{Examples}
\begin{xmpl}
Input
\begin{lstlisting}
dat<-read.table("http://notendur.hi.is/~gunnar/kennsla/alsm/data/set115.dat", header=T)
cols<- c("le", "osl")
\end{lstlisting}

Analysis
\begin{lstlisting}
Mn<-mean(dat[, cols[1]])
\end{lstlisting}


Output
\begin{lstlisting}
print (Mn)
\end{lstlisting}
\end{xmpl}

%% Slide https://tutor-web.net/math/math612.0/lecture270/slide20
\subsection{Modularity and functions}
\begin{fbox}
\begin{minipage}{0.97\textwidth}
In many cases groups of commands can be collected together into a function. 
\end{minipage}
\end{fbox}
\subsubsection{Details}
Typically a project has several such functions.
\subsubsection{Examples}
\begin{xmpl}

Suppose you want to plot the weight vs. length for several datasets in

$$\verb|http://hi.is/~gunnar/kennsla/alsm/data|$$

A function can then be set up with the file number as an argument:
\begin{lstlisting}
 plotwtle<-function (fnum){
 fname<-paste(
 "http://hi.is/~gunnar/kennsla/alsm/data/set",fnum,".dat",sep="")
  cat("The URL B", fname,"\n")
  dat<-read.table(fname,header=T)
  ttl<-paste("Data from file number", fnum)
  plot(dat$le,dat$osl,main=ttl)
  }
\end{lstlisting}

Now call this with
\begin{lstlisting}
plotwtle(105)
\end{lstlisting}
\end{xmpl}


%% Slide https://tutor-web.net/math/math612.0/lecture270/slide30
\subsection{Modularity and files}
\begin{fbox}
\begin{minipage}{0.97\textwidth}
It is advisable to split larger projects into several manageable files. 
\end{minipage}
\end{fbox}
\subsubsection{Details}
Once a project reaches more than five lines of code, it should be stored in one or more separate files. In order to combine these files a single “source” command file can be created.\\

Typically function definitions are stored in separate files, so one may have several separate files like:\\

    "input.r" \\
    "function.r" \\
    "analysis.r" \\
    "output.r" \\

While developing the analysis, the data would only be read once with\\

    source(“input.r”) \\

The goal of this practice is to end up with a set of files which are completely self-contained, so one can start with an empty R session and give only the commands like:\\

    source (“input.r”) \\
    source (“functions.r”) \\
    source (“analysis.r”)\\ 

Furthermore, this ensures repeatability. 
\subsubsection{Examples}
\begin{xmpl}
For a given project “input”, “functions” “analysis” and “output” files can be created as below. 

\underline{input.r}
\begin{lstlisting}
dat<-read.table("http://notendur.hi.is/~gunnar/kennsla/alsm/data/set115.dat", header=T)
\end{lstlisting}

\underline{functions.r}

\begin{lstlisting}
plotwtle<-function(fnum){
 fname<-paste("http://notendur.hi.is/~gunnar/kennsla/alsm/data/set",fnum,".dat",sep="")
  cat("The URL is",fname,"\n")
  dat<-read.table(fname,header=T)
  ttl<-paste("My data set was",fnum)
  plot(dat$le,dat$osl,main=ttl,xlab="Length(cm)",ylab="Live weight (g)")
}
\end{lstlisting}


\underline{output.r}
\begin{lstlisting}
source("functions.r")
for(i in 101:150){
  fnam<-paste("plot",i,".pdf",sep="")
  pdf(fnam)
  plotwtle(i)
  dev.off()
}
\end{lstlisting}

These files can be executed with source commands as below:
\begin{description}
\item source (“input.r”)
\item source (“functions.r”)
\item source (“output.r”)
\end{description}
		
\end{xmpl}


%% Slide https://tutor-web.net/math/math612.0/lecture270/slide40
\subsection{Structuring an R project}
\subsubsection{Details}
We already covered how to split code into different functions and linking them together with the help of one executable file that is "sourcing" the others. 
However, when you undertake a larger project, there will be a lot of different data and files and it is very advisable to have a consistent structure 
throughout the project.\\ 

A common project layout is to allocate all project files into a folder, something along the lines of:
\begin{verbatim}
/project
	/data
	/src
	/doc
	/figs (or /out)
\end{verbatim}
Such a structure is quite normal in programming languages such as C, Matlab, and R.\\

Purpose of the different folders:\\

/data: Contains all important data to the project, which you will use. This folder should be read-only! No function is allowed to write anything into this folder. \\\\
/src:   (abbreviation for "source(-code)") Here you will store all the functions that you programmed. You can decide to store the executable function here as well or, alternatively, have that one in the root project folder. \\\\
/doc:  Contains further documentation material about your project. This could be, for example, readme files for other people who use your functions, or the paper you wrote about the project, or the latex files while you're writing.\\\\
/figs or /out: Here your functions are allowed to write and can produce the different results, like graphs, figures or anything else. \\

Finally, a large programming project should at some stage be split into packages and stored on dedicated servers such as github or CRAN.

\subsubsection{Examples}
\begin{xmpl}
Consider first the issue of maintaining the code itself. It is common for R beginners to only work interactively within the command-line interface. 
However, it is essential that the code be kept in one or more files.\\

For large projects these will be several different files, each with its own purpose. To run a complete analysis one would typically set up one file to run all the tasks by reading in data through analyses to 
outputs.\\

For example, a file named "run.r" could contain the sequence of commands:

\begin{description}
\item source("setup.r")
\item source("analysis.r")
\item source("plot.r")
\end{description}

\end{xmpl}


%% Slide https://tutor-web.net/math/math612.0/lecture270/slide50
\subsection{Loops, for}
\begin{fbox}
\begin{minipage}{0.97\textwidth}
If a piece of code is to be run repeatedly, the for-loop is normally used. 
\end{minipage}
\end{fbox}
\subsubsection{Details}
If a piece of code is to be run repeatedly, the for-loop is normally used. The R code form is:
\begin{lstlisting}
	for(index in sequence){
	commands
	}
\end{lstlisting}
\subsubsection{Examples}
\begin{xmpl}
To add numbers we can use

\begin{lstlisting}
tot <- 100
for(i in 1:100){
  tot <- tot + i
}
cat ("the sum is ", tot, "\n")
\end{lstlisting}
\end{xmpl}
\begin{xmpl}

Define the plot function

\begin{lstlisting}

plotwtle <- AS BEFORE
\end{lstlisting}

To plot several of these we can use a sequence:
\begin{lstlisting}
plotwtle(101)
plotwtle(102)
.
.
.
\end{lstlisting}
or a loop
\begin{lstlisting}
for (i in 101:150){
  fname<- paste("plot", i, ".pdf", sep="")
  pdf(fname)
  plotwtle(i)
  dev.off()
}
\end{lstlisting}
\end{xmpl}

%% Slide https://tutor-web.net/math/math612.0/lecture270/slide60
\subsection{The if and ifelse commands}
\begin{fbox}
\begin{minipage}{0.97\textwidth}
The "if" statement is used to conditionally execute statements.\\
The "ifelse" statement conditionally replaces elements of a structure.
\end{minipage}
\end{fbox}
\subsubsection{Examples}
\begin{xmpl}
If we want to compute $x^x$ for $x$-values in the range 0 through 5, we can use
\begin{lstlisting}
xlist<-seq(0,5,0.01)
y<-NULL
for(x in xlist){
  if(x==0){
    y<-c(y,1)
  }else{
    y<-c(y,x**x)
  }
}

\end{lstlisting}
\end{xmpl}
\begin{xmpl}
\begin{lstlisting}
x<-seq(0,5,0.01)
y<-ifelse(x==0,1,x^x)

\end{lstlisting}
\end{xmpl}
\begin{xmpl}
\begin{lstlisting}
dat<-read.table ("file")
dat<-ifelse (dat==0,0.01,dat)
\end{lstlisting}
\end{xmpl}
\begin{xmpl}

\begin{lstlisting}
x<-ifelse (is.na(x),0,x)
\end{lstlisting}
\end{xmpl}

%% Slide https://tutor-web.net/math/math612.0/lecture270/slide70
\subsection{Indenting}
\begin{fbox}
\begin{minipage}{0.97\textwidth}
% Lisa, lal@hi.is

Code should be properly indented! \\

\end{minipage}
\end{fbox}
\subsubsection{Details}
fFunctions, for-loops, and if-statements should always be indented.

%% Slide https://tutor-web.net/math/math612.0/lecture270/slide80
\subsection{Comments}
\begin{fbox}
\begin{minipage}{0.97\textwidth}
All code should contain informative comments. Comments are separated out from code using the pound symbol (\#).
\end{minipage}
\end{fbox}
\subsubsection{Examples}
\begin{xmpl}

\#\#\#\#\#\#\#\#\#\#\#\#\#\#\#\#\#\#\#\#


\#\#\#\#SETUP DATA\#\#\#\#


\#\#\#\#\#\#\#\#\#\#\#\#\#\#\#\#\#\#\#\#

dat<-read.table(filename)

x<-log(dat\$le)  \#log-transformation of length

y<-log(dat\$wt)  \#log-transformation of weight

\#\#\#\#\#\#\#\#\#\#\#\#\#\#\#\#\#\#\#\#\#\#


\#\#\#\#THE ANALYSIS\#\#\#\#


\#\#\#\#\#\#\#\#\#\#\#\#\#\#\#\#\#\#\#\#\#\#

\end{xmpl}

{\bf Copyright}
2021, Gunnar Stefansson (editor) with contributions from very many students

This work is licensed under the Creative Commons
Attribution-ShareAlike License. To view a copy of this license, visit
http://creativecommons.org/licenses/by-sa/1.0/ or send a letter to
Creative Commons, 559 Nathan Abbott Way, Stanford, California 94305,
USA.
\clearpage
%% Lecture https://tutor-web.net/math/math612.0/lecture280
\section{The Central Limit Theorem and related topics}
%% Slide https://tutor-web.net/math/math612.0/lecture280/slide10
\subsection{The Central Limit Theorem}
\begin{fbox}
\begin{minipage}{0.58\textwidth}
If measurements are obtained independently and come from a process with finite variance, 
then the distribution of their mean tends towards a Gaussian (normal) distribution as the sample size increases.
\end{minipage}
\hspace{0.5mm}
\begin{minipage}{0.38\textwidth}
\begin{picture}
30
\end{picture}

Figure:  The standard normal density
\end{minipage}
\end{fbox}
\subsubsection{Details}
\begin{thm}
The {\bf Central Limit Theorem} states that if $X_1, X_2, \ldots$ are independent and identically 
distributed random variables with mean $\mu$ and (finite) variance $\sigma^2$, 
then the distribution of $\bar{X}_n:= \frac{X_1+\dots+X_n}{n}$ tends towards a normal distribution. 
\end{thm}

It follows that for a large enough sample size $n$, 
the distribution random variable $\bar{X}_n$ can be approximated by $n(\mu,\sigma^2/n)$.

The standard normal distribution is given by the p.d.f.\\
$$
\varphi(z) = \frac{1}{\sqrt{2\pi}} e^{\frac{-z^2}{2}}$$ 
 for $z\in \mathbb{R}$.\\

The standard normal distribution has an expected value of zero, 
$$
\mu = \int z\varphi (z)dz =0
$$
and a variance of 
$$
\sigma^2 = \int ({z-\mu})^2 \varphi(z)dz=1
$$

If a random variable $Z$ has the standard normal (or Gaussian)  distribution, we write $Z\sim n(0,1)$. 

If we define a new random variable, $Y$, by writing $Y=\sigma Z + \mu$, then $Y$ has an expected value of $\mu$, a variance of $\sigma^2$ and a density (p.d.f.) given by the formula:
$$
f(y) = \frac{1}{\sqrt{2\pi}\sigma}   \ e^{\frac{-(y-\mu)^2}{2\sigma^2}}.
$$
This is general normal (or Gaussian) density, with mean $\mu$ and variance $\sigma^2$.

The Central Limit Theorem states that if you take the mean of several independent random variables, the distribution of
that mean will look more and more like a Gaussian distribution (if the variance of the original random variables is finite).

More precisely, the cumulative distribution function of
$$
\frac{\bar{X}_n - \mu}{\sigma/\sqrt{n}}
$$
converges to $\Phi$, the $n(0,1)$ cumulative distribution function.


\subsubsection{Examples}
\begin{xmpl}
If we collect measurements on waiting times, these are typically assumed to come 
from an exponential distribution with density 

\begin{equation*}
f(t)=\lambda e^{-\lambda t},\textrm{ for } t>0 
\end{equation*}

The Central Limit Theorem states that the mean of several such waiting times will tend to have a normal distribution.
\end{xmpl}

\begin{xmpl}
We are often interested in computing
$$
w=\frac{\bar{x}-\mu_0}{\frac{s}{\sqrt{n}}}
$$
which comes from a t-distribution (see below), if the $x_i$ are independent outcomes from a normal distribution. 

However, if $n$ is large and $\sigma^2$ is finite then $w$ values will look as though they came from a normal distribution. 
This is in part a consequence of the Central Limit Theorem, but also of the fact that $s$ will become close to $\sigma$ as $n$ increases.
\end{xmpl}

%% Slide https://tutor-web.net/math/math612.0/lecture280/slide20
\subsection{Properties of the binomial and Poisson distributions}
\begin{fbox}
\begin{minipage}{0.97\textwidth}
The binomial distribution is really a sum of 0 and 1 values (counts of failures = 0 and successes =1). So, a simple, single binomial outcome will correspond to coming from a normal distribution if the count is large enough.
\end{minipage}
\end{fbox}
\subsubsection{Details}
Consider the binomial probabilities:

$$p(x)=\binom{n}{x}p^x(1-p)^{n-x}$$

for $x=0,1,2,3, \cdots,n$ where $n$ is a non-negative integer.  Suppose $p$ is a small positive number, specifically consider a sequence of decreasing $p$-values, specified with $p_n= \frac{\lambda}{n}$ and consider the behavior of the probability as $n \rightarrow \infty$. We obtain:

\begin{eqnarray}
\binom{n}{x}p_n^x(1-p_n)^{n-x}& = &\frac{n!}{x!(n-x!)} \left ( \frac{\lambda}{n} \right )^x \left ( 1-\frac{\lambda}{n} \right )^{n-x}\\
& = &\frac{n(n-1)(n-2)\cdots (n-x+1)}{x!} \frac{\frac{\lambda}{n}^x}{\left ( 1-\frac{\lambda}{n} \right ) ^x} \left ( 1-\frac{\lambda}{n} \right )^n\\
& = &\frac{n(n-1)(n-2)\cdots (n-x+1)}{x!n^x} \frac{\lambda^x}{\left ( 1-\frac{\lambda}{n} \right ) ^x} \left ( 1-\frac{\lambda}{n} \right )^n
\end{eqnarray}
\begin{notes}
Notice that $\frac{n(n-1)(n-2)\cdots (n-x+1)}{n^x}\to 1$ as $n\to\infty$. Also notice that $(1-\frac{\lambda}{n})^x\to 1$ as $n\to\infty$. Also

$$\lim_{n \to \infty} \bigg( 1-\frac{\lambda}{n} \bigg) = e^{- \lambda}$$

 

and it follows that

$$ \lim_{n \to \infty} \binom{n}{x}p_n^x(1-p_n)^{n-x} = \frac{e^{- \lambda} \lambda^x}{x!}, x= 0,1,2, \cdots , n $$

and hence the binomial probabilities may be approximated with the corresponding Poisson.   
\end{notes}


\subsubsection{Examples}
\begin{xmpl}

The mean of a binomial (n,p) variable is $\mu=n\cdot p$ and the variance is $\sigma^2=np(1-p)$.\\

The R command $dbinom(q,n,p)$ calculates the probability of $q$ successes in $n$ trials assuming that the probability of a success is $p$ in each trial (binomial distribution), and the R command $pbinom(q,n,p)$ calculates the probability of obtaining $q$ or fewer successes in $n$ trials.

The normal approximation of this distribution can be calculated with $pnorm(q,mu,sigma)$ which becomes $pnorm(q,n*p,sqrt(n*p(1-p))$. Three numerical examples (note that pbinom and pnorm give similar values for large n):
\begin{lstlisting}
pbinom(3,10,0.2)
[1] 0.8791261
pnorm(3,10*0.2,sqrt(10*0.2*(1-0.2)))
[1] 0.7854023

pbinom(3,20,0.2)
[1] 0.4114489
pnorm(3,20*0.2,sqrt(20*0.2*(1-0.2)))
[1] 0.2880751

pbinom(30,200,0.2)
[1] 0.04302156
pnorm(30,200*0.2,sqrt(200*0.2*(1-0.2)))
[1] 0.03854994
\end{lstlisting}
\end{xmpl}
\begin{xmpl}


We are often interested in computing $w=\frac{\bar{x}-\mu}{s/\sqrt{n}}$ which has a t-distribution if the $x_i$ are independent outcomes from a normal distribution. If $n$ is large and $\sigma^2$ is finite, this will look as if it comes from a normal distribution.\\

The numerical examples below demonstrate how the t-distribution approaches the normal distribution.

\begin{lstlisting}
qnorm(0.7)
[1] 0.5244005 
#This is the value which gives the cumulative probability of p=0.7 for a n~(0,1)
qt(0.7,2)
[1] 0.6172134
#The value, which gives the cumulative probability of p=0.7 with n=2 for the t-distribution.
qt(0.7,5)
[1] 0.5594296
qt(0.7,10)
[1] 0.541528
qt(0.7,20)
[1] 0.5328628


qt(0.7,100)
[1] 0.5260763
\end{lstlisting}
\end{xmpl}

%% Slide https://tutor-web.net/math/math612.0/lecture280/slide30
\subsection{Monte Carlo simulation}
\begin{fbox}
\begin{minipage}{0.58\textwidth}
If we know an underlying process we can simulate data from the process and evaluate the distribution of any quantity based on such data.

\end{minipage}
\hspace{0.5mm}
\begin{minipage}{0.38\textwidth}
\begin{picture}
31
\end{picture}

Figure:  A simulated set of $t$-values based on data from an exponential distribution.
\end{minipage}
\end{fbox}
\subsubsection{Examples}
\begin{xmpl}

Suppose our measurements come from an exponential distribution and we want to compute

$$ t = \frac{\overline x - \mu}{s / \sqrt{n}}$$

but we want to know the distribution of those when $\mu$ is the true mean.\\

For instance, $n=5$ and $\mu = 1$, we can simulate (repeatedly) $x_1, \ldots , x_5$ and compute a t-value for each. The following R commands can be used for this:

\begin{lstlisting}
library(MASS)
n<-5          
mu<-1         
lambda<-1     
tvec<-NULL    
for(sim in 1:10000){ 
   x<-rexp(n,lambda)  
   xbar<-mean(x)      
   s<-sd(x)           
   t<-(xbar-mu)/(s/sqrt(n)) 
   tvec<-c(tvec,t)          
}                       

#then do...                

 truehist(tvec) 	#truehist gives a better histogram 
 sort(tvec)[9750]     
 sort(tvec)[250]      
\end{lstlisting}
\end{xmpl}


{\bf Copyright}
2021, Gunnar Stefansson (editor) with contributions from very many students

This work is licensed under the Creative Commons
Attribution-ShareAlike License. To view a copy of this license, visit
http://creativecommons.org/licenses/by-sa/1.0/ or send a letter to
Creative Commons, 559 Nathan Abbott Way, Stanford, California 94305,
USA.
\clearpage
%% Lecture https://tutor-web.net/math/math612.0/lecture290
\section{Miscellanea}
%% Slide https://tutor-web.net/math/math612.0/lecture290/slide10
\subsection{Simple probabilities in R}
\begin{fbox}
\begin{minipage}{0.97\textwidth}
R has functions to compute probabilities based on most common distributions.\\\\

If $X$ is a random variable with a known distribution, then R can typically compute values of the cumulative distribution function or:
$$F(x)=P[X \leq x]$$

\end{minipage}
\end{fbox}
\subsubsection{Examples}
\begin{xmpl}

If $X \sim b(n,p)$ has binomial distribution, i.e.
$$ P(X = x) = {n \choose x}p^x(1-p)^{n-x},$$
then cumulative probabilities can be computed with $pbinom$, e.g.
\begin{lstlisting}
pbinom(5,10,0.5) 
\end{lstlisting} 

gives  $$P[X \leq 5] = 0.623$$
where $$X \sim b(n=10,p= \frac{1}{2}).$$

This can also be computed by hand. Here we have $n=10$, $p=1/2$ and the probability 
$P[X \leq 5] $ is obtained by adding up the individual probabilities, 
$P[X =0]+P[X =1]+P[X =2]+P[X =3]+P[X =4]+P[X =5]$ 

$$
P[X \leq 5]  = \sum_{x=0}^5 {10\choose x} \frac{1}{2}^x\frac{1}{2}^{10-x}.
$$
This becomes
$$
P[X \leq 5]  = {10\choose 0} \frac{1}{2}^0\frac{1}{2}^{10-0} +{10\choose 1} \frac{1}{2}^1\frac{1}{2}^{10-1}+{10\choose 1} \frac{1}{2}^2\frac{1}{2}^{10-2}+{10\choose 3} \frac{1}{2}^3\frac{1}{2}^{10-3}+{10\choose 4} \frac{1}{2}^4\frac{1}{2}^{10-4}+{10\choose 5} \frac{1}{2}^5\frac{1}{2}^{10-5}
$$
or
$$
P[X \leq 5]  = {10\choose 0} \frac{1}{2}^{10} +{10\choose 1} \frac{1}{2}^{10}+{10\choose 1} \frac{1}{2}^{10}+{10\choose 3} \frac{1}{2}^{10}+{10\choose 4} \frac{1}{2}^{10}+{10\choose 5} \frac{1}{2}^{10}=\frac{1}{2}^{10}\left[1+10+45+...\right ].
$$


Furthermore,
\begin{lstlisting} 
pbinom(10,10,0.5)
[1] 1
\end{lstlisting}
and
\begin{lstlisting} 
pbinom(0,10,0.5) 
[1] 0.0009765625
\end{lstlisting}



It is sometimes of interest to compute $P[X=x]$ in this case, and this is given by the $dbinom$ function, e.g.

\begin{lstlisting} 
dbinom(1,10,0.5)
[1] 0.009765625
\end{lstlisting}

or $ \frac{10}{1024}$

\end{xmpl}
\begin{xmpl}

Suppose $X$ has a uniform distribution between $0$ and $1$, i.e. $X \sim U(0,1)$.  Then the $punif$ function will return probabilities of the form $$ P[X \leq x]= \int_{-\infty}^{x} f(t)dt= \int_{0}^{x} f(t)dt$$

where $f(t)=1$ if $0 \leq t \leq 1$
and $f(t)=0$.
For example:
\begin{lstlisting}
punif(0.75)
[1]  0.75
\end{lstlisting}

To obtain $P[a \leq X \leq b],$ we use $punif$ twice, e.g.
\begin{lstlisting}
punif(0.75)-punif(0.25)
[1]  0.5
\end{lstlisting}
\end{xmpl}


%% Slide https://tutor-web.net/math/math612.0/lecture290/slide20
\subsection{Computing normal probabilities in R}
\begin{fbox}
\begin{minipage}{0.97\textwidth}
To compute probabilities $X\sim n(\mu,\sigma^2)$
is usually transformed, since we know that
$$
Z:=\frac{X-\mu}{\sigma} \sim(0,1)
$$
The probabilities can then be computed for either $X$ or $Z$ with the $pnorm$ function in R.
\end{minipage}
\end{fbox}
\subsubsection{Details}
Suppose $X$ has a normal distribution with mean $\mu$ and variance 
$$
X\sim n(\mu,\sigma^2)
$$
then to compute probabilities, $X$ is usually transformed, since we know that
$$
Z=\frac{X-\mu}{\sigma} \sim(0,1)
$$
and the probabilities can be computed for either $X$ or $Z$ with the $pnorm$ function.
\subsubsection{Examples}
\begin{xmpl}
If $Z \sim n(0,1)$ then we can e.g. obtain $P[Z\leq1.96]$ with
\begin{lstlisting}
pnorm(1.96)
[1] 0.9750021

pnorm(0)
[1] 0.5

pnorm(1.96)-pnorm(1.96)
[1] 0

pnorm(1.96)-pnorm(-1.96)
[1] 0.9500042
\end{lstlisting}
The last one gives the area between -1.96 and 1.96.
\end{xmpl}
\begin{xmpl}
If $X \sim n(42,3^2)$ then we can compute probabilites either by transforming
\begin{align*}
P[X\leq x] &= P[\frac{X-\mu}{\sigma} \leq \frac{x-\mu}{\sigma}]\\
           &= P[Z \leq \frac{x-\mu}{\sigma}]
\end{align*}
and calling $pnorm$ with the computed value $z=\frac{x-\mu}{\sigma}$, or call $pnorm$ with $x$ and specify $\mu$ and $\sigma$.\\
\\
To compute $P[X\leq 48]$, either set $z=(48-42)/3=2$ and obtain
\begin{lstlisting}
pnorm(2)
[1] 0.9772499
\end{lstlisting}

or specify $\mu$ and $\sigma$

\begin{lstlisting}
pnorm(42,42,3)
[1] 0.5
\end{lstlisting}
\end{xmpl}



%% Slide https://tutor-web.net/math/math612.0/lecture290/slide30
\subsection{Introduction to hypothesis testing}
\subsubsection{Details}
If we have a random sample $x_1, \ldots , x_n$ from a normal distribution, then we consider them to be outcomes of independent random variables $X_1, \ldots , X_n$ where $X_i \sim n(\mu, \sigma^2)$.
Typically, $\mu$ and $\sigma^2$ are unknown but assume for now that $\sigma^2$ is known.\\

Consider the hypothesis:\\

$H_0: \mu = \mu_0$  vs. $H_1: \mu > \mu_0$\\

where $\mu_0$ is a specified number.\\

Under the assumption of independence, the sample mean

$$\overline{x} = \frac{1}{n} \sum^n_{i=1}x_i$$

is also an observation from a normal distribution, with mean $\mu$ but a smaller variance.Specifically, $\overline{x}$ is the outcome of

$$\overline{X} = \frac{1}{n} \sum^n_{i=1}X_i$$

and 

$$X \sim n(\mu, \frac{ \sigma^2}{n})$$

so the standard deviation of X is $\frac{\sigma}{\sqrt{n}}$, so the appropriate error measure for $\overline{x}$ is $frac{\sigma}{\sqrt{n}}$, when $\sigma$ is unknown.\\

If $H_0$ is true, then

$$z:= \frac{\overline{x}-\mu_0}{\sigma / \sqrt{n}}$$

is an observation from an $n \sim n (0,1)$ distribution, i.e. an outcome of 

$$Z= \frac{\overline{X}-\mu_0}{\sigma / \sqrt{n}}$$

where $Z \sim n(0,1)$ when $H_0$ is correct. It follows that e.g. $P[\vert Z \vert > 1.96] = 0.05$ and if we observe $\vert Z \vert > 1.96$ then we reject the null hypothesis.\\

Note that the value z* = 1.96 is a quantile of the normal distribution and we can obtain other quantiles with the $pnorm$ function, e.g. $pnorm(0.975)$ gives 1.96.

{\bf Copyright}
2021, Gunnar Stefansson (editor) with contributions from very many students

This work is licensed under the Creative Commons
Attribution-ShareAlike License. To view a copy of this license, visit
http://creativecommons.org/licenses/by-sa/1.0/ or send a letter to
Creative Commons, 559 Nathan Abbott Way, Stanford, California 94305,
USA.
\clearpage
%% Lecture https://tutor-web.net/math/math612.0/lecture310
\section{Multivariate probability distributions}
%% Slide https://tutor-web.net/math/math612.0/lecture310/slide10
\subsection{Joint probability distribution}
\begin{fbox}
\begin{minipage}{0.97\textwidth}
If

$ X_1,\ldots, X_n$ are discrete random variables with

$P[X_1 = x_1, X_2 = x_2,\ldots, X_n = x_n] = p(x_1,\ldots, x_n) $, where $x_1, \ldots, x_n$ are numbers, then the function $p$ is the joint probability mass function (p.m.f.) for the random variables $X_1, \ldots, X_n$.\\

For continuous random variables $Y_1, \ldots, Y_n$, a function $f$ is called the joint probability density function if, 

$P [Y\in {A}] = \int\int\ldots\int f(y_1,\ldots y_n)dy_1dy_2 \cdots dy_n$.

% corrected
\end{minipage}
\end{fbox}
\subsubsection{Details}
\begin{defn}
If $ X_1, \ldots, X_n$ are discrete random variables with $P[X_1 = x_1, X_2 = x_2,\ldots, X_n = x_n] = p(x_1,\ldots, x_n) $ where $x_1 \ldots x_n$ are numbers, then the function $p$ is the joint \textbf{ probability mass function (p.m.f.)} for the random variables $X_1, \ldots, X_n$.
\end{defn}

\begin{defn}
For continuous random variables $Y_1, \ldots, Y_n$, a function $f$ is called the joint probability density function if, 

$P [Y\in {A}] = \underbrace{\int\int\ldots\int}_{A} f(y_1,\ldots y_n)dy_1dy_2 \cdots dy_n$.
\end{defn}

\begin{notes}
Note that if $X_1, \ldots, X_n$ are independent and identically distributed, each with p.m.f. $p$, then $p(x_1, x_2, \ldots, x_n) = q(x_1)q(x_2)\ldots q(x_n)$,
i.e, $P [X_1 = x_1,  X_2 = x_2,\ldots, X_n= x_n] = P [X_1 = x_1] P[X_2 = x_2]\ldots P[X_n= x_n]$.
\end{notes}

\begin{notes}
Note also that if $A$ is a set of possible outcomes  $ (A  \subseteq \mathbb{R}^n)$, then
we have 
$$P[X \in {A}] = \sum_{(x_1,\ldots,x_n)\in A} p(x_1,\ldots, x_n).$$
\end{notes}

\subsubsection{Examples}
\begin{xmpl}

An urn contains blue and red marbles, which are either light or heavy. Let $X$ denote the color and $Y$ the weight of a marble, chosen at random\\

$$
\begin{array}{c c c c}
\hline\hline
  X \setminus Y	&	\text{L}	&	\text{H}	&	\text{Total} \\
  B	&	5	&	6	&	11\\
  R	&	7	&	2	&	9\\
  TT	&	12	&	8	&	20\\
\hline
\end{array}
$$

We have $P[X="'b"', Y ="l"'] = \frac{5}{20}$.\\

The joint p.m.f. is:

$$
\begin{array}{c c c c}
\hline\hline
X \setminus Y	&	\text{L}	&	\text{H}	&	\text{Total} \\
\text{B}	  & \frac{5}{20}	& \frac{6}{20}	& \frac{11}{20}\\
\text{R}	  & \frac{7}{20}	& \frac{2}{20}$	& \frac{9}{20}\\
\text{Total}  & \frac{12}{20}	& \frac{8}{20}	&	1\\
\hline
\end{array}
$$
\end{xmpl}

%% Slide https://tutor-web.net/math/math612.0/lecture310/slide20
\subsection{The random sample}
\begin{fbox}
\begin{minipage}{0.58\textwidth}
A set of random variables $X_1, \ldots, X_n$ is a random sample if they are independent and identically distributed (i.i.d.).\\


A set of numbers $x_1, \ldots, x_n$ are called a random sample if they can be viewed as an outcome of such random variables.

\end{minipage}
\hspace{0.5mm}
\begin{minipage}{0.38\textwidth}
\begin{picture}
32
\end{picture}


\end{minipage}
\end{fbox}
\subsubsection{Details}
Samples from populations can be obtained in a number of ways. However, to draw valid conclusions about populations, the samples need to obtained randomly.
\begin{defn}
In {\bf random sampling}, each item or element of the population has an equal and independent chance of being selected.
\end{defn}
A set of random variables; $X_1 \ldots X_n$ is a random sample if they are independent and identically distributed (i.i.d.).
\begin{defn}
If a set of numbers $x_1 \ldots x_n$ can be viewed as an outcome of random variables, these are called a {\bf random sample}.
\end{defn}
\subsubsection{Examples}
\begin{xmpl}

If $X_1, \ldots, X_n  \sim U(0,1)$, i.i.d., i.e.  $X_1$ and $X_n$ are independent and each have a uniform distribution between 0 and 1. Then they have a joint density which is the product of the densities of $X_1$ and $X_n$.\\

Given the data in the above figure and if  $x_1 , x_2 \in \mathbb{R}$

$$
f(x_1, x_2) = f_1(x_1) f_2(x_2) =
\begin{cases}
  1 & \text{if } 0 \leq x_1, x_2 \leq 1 \\
  0 & \text{elsewhere}
\end{cases}.
$$

\end{xmpl}
\begin{xmpl}

 Toss two dice independently, and let $X_1, X_2$ denote the two (future) outcomes.\\

Then 


$$
 P[X_1 = x_1, X_2 =  x_2]=
 \begin{cases}
  \frac{1}{36} & \text{if } 1 \leq x_1, x_2 \leq 6 \\
  0 & \text{elsewhere}
\end{cases}.
$$

is the joint p.m.f.

\end{xmpl}

%% Slide https://tutor-web.net/math/math612.0/lecture310/slide30
\subsection{The sum of discrete random variables}
\subsubsection{Details}
Suppose $X$ and $Y$ are discrete random values with a probability mass function $p$.  Let $Z=X+Y$. Then

\begin{eqnarray*}
P(Z=z) & = &\sum_{\{ (x,y): x+y=z\}} p(x,y)
\end{eqnarray*}

\subsubsection{Examples}
\begin{xmpl}
$(X,Y) = \text{outcomes}$,

\begin{verbatim}
    [,1] [,2] [,3] [,4] [,5] [,6]
[1,]    2    3    4    5    6    7
[2,]    3    4    5    6    7    8
[3,]    4    5    6    7    8    9
[4,]    5    6    7    8    9   10
[5,]    6    7    8    9   10   11
[6,]    7    8    9   10   11   12

\end{verbatim}

$$P[X+Y =7] =\frac{6}{36}=\frac{1}{6}$$

Because there are a total of 36 equally likely outcomes and 7 occurs six times this means that 
$P[X + Y = 7] =\frac{1}{6}$.  

Also
$$P[X+Y = 4] = \frac{3}{36} = \frac{1}{12}$$

% this is complete rubbish:
%A matrix may be constructed in R that may be used to calculate the the probability of two outcomes.   
%\begin{verbatim}
%r<-rep(1:6,6)
%mat<-matrix(r, ncol=6, nrow=6)
%mat2<-t(mat)
%z<-mat+mat2
%\end{verbatim}
\end{xmpl}

%% Slide https://tutor-web.net/math/math612.0/lecture310/slide40
\subsection{The sum of two continuous random variables}
\begin{fbox}
\begin{minipage}{0.58\textwidth}
If X and Y are continuous random variables
with joint p.d.f. $f$ and $Z=X+Y$, then we can find the density of $Z$ by calculating the cumulative distribution function.

\end{minipage}
\hspace{0.5mm}
\begin{minipage}{0.38\textwidth}
\begin{picture}
33
\end{picture}


\end{minipage}
\end{fbox}
\subsubsection{Details}
If X and Y are c.r.v. with joint p.d.f. $f$ and $Z=X+Y$, then we can find the density of $Z$ by first finding the cumulative distribution function $$P[Z \leq z]=P[X+Y \leq z]=\int\int_{\{(x,y):x+y \leq z\}} f(x,y)dxdy.$$
\subsubsection{Examples}
\begin{xmpl}

If X and Y $\sim$ U(0,1), independent and $Z=X+Y$ then 
 
$$
P[Z \leq z]= 
\begin{cases}
  0 & \text{for} & z \leq 0\\
  \frac{z^2}{2} & \text{for} & 0< z <1\\
  1 & \text{for}& z>2\\
  1-\frac{(2-z)^2}{2} & \text{for} & 1< z <2
\end{cases}
$$

the density of $z$ becomes 

$$
g(z)= 
\begin{cases}
  z & \text{for} & 0 < z \leq 1\\
  2-z & \text{for} & 1 < z \leq 2\\
  0 & \text{for} & \text{elsewhere}
\end{cases}.
$$
\end{xmpl}

\begin{xmpl}

To approximate the distribution of $Z=X+Y$ where $X,Y \sim U(0,1)$ i.i.d., we can use Monte Carlo simulation. So, generate 10.000 pairs, set them up in a matrix and compute the sum. 
\end{xmpl}

%% Slide https://tutor-web.net/math/math612.0/lecture310/slide50
\subsection{Means and variances of linear combinations of independent random variables}
\begin{fbox}
\begin{minipage}{0.97\textwidth}
If $X$ and $Y$ are random variables and $a,b\in\mathbb{R}$, then

$$
E[aX+bY] = aE[X]+bE[Y].
$$
\end{minipage}
\end{fbox}
\subsubsection{Details}
If $X$ and $Y$ are random variables, then

$$
E[X+Y] = E[X]+E[Y]
$$

i.e. the expected value of the sum is just the sum of the expected values. The same applies to a finite sum, and more generally

$$
E[\sum_{i=1}^{n} a_i X_i] = \sum_{i=1}^{n} a_i E[X_i] 
$$
when $X_i,...,X_n$ are random variables and $a_1,...,a_n$ are numbers (if the expectations exist).

If the random variables are independent, then the variance also add

$$
V[X+Y] = V[X] + V[Y]
$$
and

$$
V[\sum_{i=1}^{n} a_i X_i] = \sum_{i=1}^{n} a_i^2 V[X_i]
$$
\subsubsection{Examples}
\begin{xmpl}
$X,Y \sim U(0,1)$, i.i.d. then

$$
E[X+Y]=E[X] + E[Y] = \int_0^1 x\cdot 1dx+\int_0^1 x\cdot 1dx = [\frac{1}{2}x^2]_0^1+[\frac{1}{2}x^2]_0^1=1.
$$
\end{xmpl}
\begin{xmpl}


Let $X,Y\sim N(0,1)$. Then $E[X^2+Y^2] = 1+1=2$.
\end{xmpl}

%% Slide https://tutor-web.net/math/math612.0/lecture310/slide60
\subsection{Means and variances of linear combinations of measurements}
\begin{fbox}
\begin{minipage}{0.97\textwidth}
If $x_1,....,x_n$ and $y_1,....,y_n$ are numbers, and we set $$ z_i=x_i + y_i $$

$$w_i=ax_i$$ where a>0, then
$$\overline{z} = \frac{1}{n} \sum_{i=1}^{n} z_i= \overline{x} + \overline{y}$$

$$\overline{w}= a\overline{x}$$

$$s_w^2=\frac{1}{n-1}\sum_{i=1}^{n}(w_i-\overline{w})^2$$

$$ = \frac{1}{n-1}\sum_{i=1}^{n}(ax_i-a\overline{x})^2$$

$$ = a^2s_x^2 $$

and $$s_w=as_x$$

\end{minipage}
\end{fbox}
\subsubsection{Examples}
\begin{xmpl}
We set:
\begin{lstlisting}
a<-3
x<-c(1:5)
y<-c(6:10)
\end{lstlisting}

Then:
\begin{lstlisting}
z<-x+y
w<-a*x
n<-length(x)
\end{lstlisting}

Then $\overline{z}$ is:
\begin{lstlisting}
(sum(x)+sum(y))/n 
[1] 11
mean(z) 
[1] 11
\end{lstlisting}

and $\overline{w}$ becomes:
\begin{lstlisting}
a*mean(x)
[1] 9
mean(w)
[1] 9
\end{lstlisting}

and $s_w^2$ equals:
\begin{lstlisting}
sum((w-mean(w))^2))/(n-1)
[1] 22.5
sum((a*x - a*mean(x))^2)/(n-1)
[1] 22.5
a^2*var(x)
[1] 22.5
\end{lstlisting}

and $s_w$ equals:
\begin{lstlisting}
a*sd(x)
[1] 4.743416
sd(w)
[1] 4.743416
\end{lstlisting}
\end{xmpl}

%% Slide https://tutor-web.net/math/math612.0/lecture310/slide70
\subsection{The joint density of independent normal random variables}
\begin{fbox}
\begin{minipage}{0.97\textwidth}
If $Z_1, Z_2 \sim n(0,1)$ are independent then they each have density\\
$$\phi(x)=\frac{1}{\sqrt{2\pi}}e^{-\frac{x^2}{2}},x\in\mathbb{R}$$
and the joint density is the product $f(z_1,z_2)=\phi(z_1)\phi(z_2)$ or
$$f(z_1,z_2)=\frac{1}{(\sqrt{2\pi})^2} e^{\frac{-z_1^2}{2}-\frac{z_2^2}{2}} .$$
\end{minipage}
\end{fbox}
\subsubsection{Details}
If $X\sim n (\mu_1,\sigma_1^2)$ and $Y\sim n(\mu_2, \sigma_2^2)$ are independent, then their densities are
$$
f_X (x)= \frac{1}{\sqrt{2\pi}\sigma_1} \; e^\frac{-(x-\mu_1)^2}{2\sigma_1^2}$$ 
and
$$
f_Y(y) = \frac{1}{\sqrt{2\pi}\sigma_2} \; e^\frac{-(y-\mu_2)^2}{2\sigma_2^2}$$ 
and the joint density becomes\\
$$\frac{1}{2\pi\sigma_1\sigma_2} \: e^{-\frac{(x-\mu_1)^2}{2\sigma_1^2}-\frac{(y-\mu_2)^2}{2\sigma_2^2}}$$

Now, suppose $X_1,\ldots,X_n\sim n(\mu,\sigma^2)$ are i.i.d., then\\
$$f(\underline{x})=\frac{1}{(2\pi)^\frac{n}{2}\sigma^n}\; e^{-\displaystyle\sum^{n}_{i=1} \frac{(x_i-\mu)^2}{a\sigma^2}}$$

\\
is the multivariate normal density in the case of i.i.d. variables.


%% Slide https://tutor-web.net/math/math612.0/lecture310/slide80
\subsection{More general multivariate probability density functions}
\subsubsection{Examples}
\begin{xmpl}
 Suppose X and Y have the joint density


$ f(x,y) =
\begin{cases}
2 & \text{   } 0\leq y \leq x \leq 1\\
0 & \text{   otherwise}
\end{cases}$

First notice that $\int_{\mathbb{R}}\int_{\mathbb{R}}f(x,y)dxdy=\int_{x=0}^1\int_{y=0}^x2dydx=\int_0^12xdx=1$, so $f$ is indeed a density function. 

Now, to find the density of $X$ we first find the c.d.f. of $X$, first note that for $a<0$ we have $P[X\leq a]=0$ but if $a\geq 0$, we obtain
$$F_X(a)=P[X\leq a]=\int_{x_0}^a\int_{y=0}^x2dydx=[x^2]_0^a=a^2.$$
The density of $X$ is therefore

$ f_X(x) = \frac{dF(x)}{dx}
\begin{cases}
2x & \text{   } 0\leq x \leq 1\\
0 & \text{   otherwise}
\end{cases}.$
\end{xmpl}

\subsubsection{Handout}
If

$ f: \mathbb{R}^n\rightarrow\mathbb{R}$

is such that

$P[X \in A] = \int_A\ldots\int f(x_1,\ldots, x_n)dx_1\cdots dx_n$

and $f(x)\geq 0$ for all $\underline{x}\in \mathbb{R}^n$

then $f$ is the {\em joint density} of

$\mathbf{X}= \left( \begin{array}{ccc}
  X_1 \\ 
  \vdots \\
  X_n
  \end{array}\right).$

If we have the joint density of some multidimensional random variable $X=(X_1,\ldots,X_n)$ given in this manner, then we can find the individual density functions of the $X_i$'s by integrating the other variables.

{\bf Copyright}
2021, Gunnar Stefansson (editor) with contributions from very many students

This work is licensed under the Creative Commons
Attribution-ShareAlike License. To view a copy of this license, visit
http://creativecommons.org/licenses/by-sa/1.0/ or send a letter to
Creative Commons, 559 Nathan Abbott Way, Stanford, California 94305,
USA.
\clearpage
%% Lecture https://tutor-web.net/math/math612.0/lecture320
\section{Some distributions related to the normal}
%% Slide https://tutor-web.net/math/math612.0/lecture320/silde10
\subsection{The normal and sums of normals}
\begin{fbox}
\begin{minipage}{0.97\textwidth}
The sum of independent normally distributed random variables is also normally distributed. 
\end{minipage}
\end{fbox}
\subsubsection{Details}
The sum of independent normally distributed random variables is also normally distributed. More specifically, if $X_1 \sim n(\mu_1, \sigma_{1}^2)$ and $X_2 \sim n(\mu_2, \sigma_{2}^2)$ are independent then $X_1 + X_2 \sim n(\mu, \sigma^2)$ since $\mu = E \left[ X_1 + X_2 \right] = \mu_1 + \mu_2$ and \\
$\sigma^2 = V \left[ X_1 + X_2 \right]$ with $\sigma^2 = \sigma_{1}^2 + \sigma_{2}^2 $ \\
if $X_1$ and $X_2$ are independent. \\

Similarly $$\sum_{i=1}^{n} X_i $$

is normal if $X_1 , \ldots , X_n$ are normal and independent.
\subsubsection{Examples}
\begin{xmpl}
Simulating and plotting a single normal distribution.
$Y \sim n(0,1)$

\begin{lstlisting}

library(MASS)           # for truehist
par(mfcol=c(2,2))       
y<-rnorm(1000)          # generating 1000 n(0,1)
mn<-mean(y)
vr<-var(y)
truehist(y,ymax=0.5)    # plot the histogram
xvec<-seq(-4,4,0.01)    # generate the x-axis
yvec<-dnorm(xvec)       # theoretical n(0,1) density
lines(xvec,yvec,lwd=2,col="red")
ttl<-paste("Simulation and theory n(0,1)\n",
           "mean=",round(mn,2),
           "and variance=",round(vr,2))
title(ttl)
\end{lstlisting}
\end{xmpl}
\begin{xmpl}

Sum of two normal distributions.

$$Y_1 \sim n(2, 2^2)$$

and  $$Y_2 \sim n(3, 3^2)$$

\begin{lstlisting}
y1<-rnorm(10000,2,2)    # n(2,2^2)
y2<-rnorm(10000,3,3)    # n(3, 3^2)
y<-y1+y2
truehist(y)
xvec<-seq(-10,20,0.01)
# check
mn<-mean(y)
vr<-var(y)
cat("The mean is",mn,"\n")
cat("The variance is ",vr,"\n")
cat("The standard deviation is",sd(y),"\n")
yvec<-dnorm(xvec,mean=5,sd=sqrt(13)) # n() density
lines(xvec,yvec,lwd=2,col="red")
ttl<-paste("The sum of n(2,2^2) and n(3,3^2)\n",
           "mean=",round(mn,2),
           "and variance=",round(vr,2))
title(ttl)
\end{lstlisting}
\end{xmpl}
\begin{xmpl}
Sum of nine normal distributions, all with $\mu = 42$ and $\sigma^2=2^2$


\begin{lstlisting}
ymat<-matrix(rnorm(10000*9,42,2),ncol=9)
y<-apply(ymat,1,mean)
truehist(y)
# check
mn<-mean(y)
vr<-var(y)
cat("The mean is",mn,"\n")
cat("The variance is ",vr,"\n")
cat("The standard deviation is",sd(y),"\n")
# plot the theoretical curve
xvec<-seq(39,45,0.01)
yvec<-dnorm(xvec,mean=5,sd=sqrt(13)) # n() density
lines(xvec,yvec,lwd=2,col="red")
ttl<-paste("The sum of nine n(42^2) \n",
           "mean=",round(mn,2),
           "and variance=",round(vr,2))
title(ttl)
\end{lstlisting}
\end{xmpl}

%% Slide https://tutor-web.net/math/math612.0/lecture320/slide20
\subsection{The Chi-square distribution}
\begin{fbox}
\begin{minipage}{0.58\textwidth}
If $X \sim n$ (0,1),then $Y = X^2$ has a distribution which is called the Chi - square distribution ($\chi^2$) on one degree of freedom. This can be written as:

$$Y \sim \chi^2$$

\end{minipage}
\hspace{0.5mm}
\begin{minipage}{0.38\textwidth}
\begin{picture}
34
\end{picture}


\end{minipage}
\end{fbox}
\subsubsection{Details}
\begin{defn}
If $X_1, X_2, \ldots, X_n$ are i.i.d. $N(0,1)$ then the distribution of 

$Y = X_1^2 + X_1^2 + \ldots + X_n^2$ has a {\bf Chi square ($\chi^2$)distribution}.
\end{defn}


%% Slide https://tutor-web.net/math/math612.0/lecture320/slide30
\subsection{Sum of Chi square Distributions}
\begin{fbox}
\begin{minipage}{0.58\textwidth}
Let $Y_1$ and $Y_2$ be independent variables. If  $Y_1 = \chi^2_{\nu_1}$ and  $Y_2 = \chi^2_{\nu_2}$,

then the sum of these two variables also follows  a chi-squared ($\chi^2$)distribution 
 
$$Y_1 + Y_2 = \chi^2_{\nu_1+ \nu_2}$$

\end{minipage}
\hspace{0.5mm}
\begin{minipage}{0.38\textwidth}
\begin{picture}
35
\end{picture}


\end{minipage}
\end{fbox}
\subsubsection{Details}
\begin{notes}
Recall that if 
$$ X_1, \ldots, X_n \sim n (\mu, \sigma^2)$$

 
are i.i.d., then 

$$\sum_{i=1}^n \left ( \frac {\bar{X} - \mu} {\sigma}\right ) ^2= \sum_{i=1}^n \frac {\left ( \bar{X} - \mu\right ) ^2} {\sigma}\sim \chi^2 $$
\end{notes}

%% Slide https://tutor-web.net/math/math612.0/lecture320/slide40
\subsection{Sum of squared deviation}
\begin{fbox}
\begin{minipage}{0.97\textwidth}
If $X_1,\cdots,X_n \sim n(\mu,\sigma^2)$ i.i.d, then



$$\sum_{i=1}^n \left ( \frac{X_i-\mu}{\sigma} \right )^2 \sim \chi_{n}^2,$$

 but we are often interested in

$$\frac{1}{n-1}\sum_{i=1}^n (X_i-\bar{X})^2\sim \chi_{n-1}^2.$$

 

\end{minipage}
\end{fbox}
\subsubsection{Details}
Consider a random sample of Gaussian random variables, i.e. $X_1,\cdots,X_n \sim n(\mu,\sigma^2)$ i.i.d. Such a collection of random variables have properties which can be used in a number of ways.



$$\sum_{i=1}^n \left ( \frac{X_i-\mu}{\sigma} \right )^2 \sim \chi_{n}^2,$$

 but we are often interested in

$$\frac{1}{n-1}\sum_{i=1}^n (X_i-\bar{X})^2\sim \chi_{n-1}^2.$$

 

\begin{notes}
A degree of freedom is lost because of subtracting the estimator of the mean as opposed to the true mean. 
\end{notes}
The correct notation is:

$\mu$ = population mean

$\bar{X}$ = sample mean (a random variable)

$\bar{x}$ = sample mean (a number)





%% Slide https://tutor-web.net/math/math612.0/lecture320/slide50
\subsection{The t-distribution}
\begin{fbox}
\begin{minipage}{0.97\textwidth}
If $U\sim n(0,1)$ and $W\sim\chi^{2}_{\nu}$ are independent, then the random variable $$T=\frac{U}{\sqrt{\frac{w}{\nu}}}$$

has a distribution which we call the t-distribution on $\nu$ degrees of freedom denoted $T \sim t_{\nu}$. 
\end{minipage}
\end{fbox}
\subsubsection{Details}
\begin{defn}
% Lagaði slurk - GS
If $U\sim n(0,1)$ and $W\sim\chi^{2}_{\nu}$ are independent, then the
random variable 
$$
T:=\frac{U}{\sqrt{\frac{w}{\nu}}}
$$ 
has a
distribution which we call the {\bf $t$-distribution} on $\nu$ degrees of
freedom, denoted $T \sim t_\nu$. 
\end{defn}
It turns out that if
$X_1, \ldots ,X_n \sim n(\mu,\sigma ^2)$ 
and we set
$$
\bar{X}=\frac{1}{n}\sum_{i=1}^n X_i
$$ 
and 
$$
S=
\sqrt{\frac{1}{1-n}\sum_{i=1}^n (X_i-X)^2}
$$ 
then
$$
\frac{\bar{X}-\mu}{S/\sqrt{n}} \sim t_{n-1}.
$$

This follows from $\bar{X}$ and $\sum_{i=1}^n(X_i-\bar{X})^2$ being
independent and $\frac{\bar{X}-\mu}{\sigma/\sqrt{n}}\sim n(0,1)$, $\sum
\frac{(X_i-\bar{X})^2}{\sigma^2}\sim \chi_{n-1}^2$.


{\bf Copyright}
2021, Gunnar Stefansson (editor) with contributions from very many students

This work is licensed under the Creative Commons
Attribution-ShareAlike License. To view a copy of this license, visit
http://creativecommons.org/licenses/by-sa/1.0/ or send a letter to
Creative Commons, 559 Nathan Abbott Way, Stanford, California 94305,
USA.
\clearpage
%% Lecture https://tutor-web.net/math/math612.0/lecture330
\section{Estimation, estimates and estimators}
%% Slide https://tutor-web.net/math/math612.0/lecture330/slide10
\subsection{Ordinary least squares for a single mean}
\begin{fbox}
\begin{minipage}{0.97\textwidth}
% Lisa, lal@hi.is

If $\mu$ is unknown and $x_i,\ldots,x_n$ are data, we can estimate $\mu$ by finding
$$
\min_{\mu} \sum_{i=1}^{n}(x_i-\mu)^2
$$

In this case the resulting estimate is simply

$$
\mu = \overline{x}
$$
and can easily be derived by setting the derivative to zero.
\end{minipage}
\end{fbox}
\subsubsection{Examples}
\begin{xmpl} 

Consider the numbers $x_1, \ldots, x_5$ to be
$$
13,7,4,16 \textrm{ and } 9
$$

We can plot $\sum(x_i-\mu)^2$ vs. $\mu$ and find the minimum.
\end{xmpl}

%% Slide https://tutor-web.net/math/math612.0/lecture330/slide20
\subsection{Maximum likelihood estimation}
\begin{fbox}
\begin{minipage}{0.97\textwidth}
If $\left (Y_1, \ldots , Y_n\right )'$ is a random vector from a density $f_{\theta}$ where $\theta$ is an unknown parameter, and $\mathbf{y}$ is a vector of observations then we define the \textbf{likelihood function} to be
$$
L_{\mathbf{y}}(\theta)=f_{\theta}(y).
$$

\end{minipage}
\end{fbox}
\subsubsection{Examples}
\begin{xmpl}
If, $x_1,\ldots,x_n$ are assumed to be observations of independent random variables 
with a normal distributions and mean of $\mu$ and variance of $\sigma^2$, then the joint density is 
$$f(x_1)\cdot f(x_2)\cdot\ldots\cdot f(x_n)$$

$$= \frac{1}{\sqrt{2\pi}\sigma}e^{-\frac{(x_1-\mu)^2}{2\sigma^2}} \cdot \ldots\cdot \frac{1}{\sqrt{2\pi}\sigma}e^{-\frac{(x_n-\mu)^2}{2\sigma^2}}$$

$$=\Pi_{i=1}^n \frac{1}{\sqrt{2\pi}\sigma}e^{-\frac{(x_i-\mu)^2}{2\sigma^2}}$$

$$=\frac{1}{(2\pi)^{n/2}\sigma^n}e^{-\frac{1}{2\sigma^2}\sum_{i=1}^n(x_i-\mu)^2}$$

and if we assume $\sigma^2$ is known then the likelihood function is
$$L(\mu)=\frac{1}{(2\pi)^{n/2}\sigma^n}e^{-\frac{1}{2\sigma^2}\sum_{i=1}^n(x_i-\mu)^2}$$

Maximizing this is done by maximizing the log, i.e. finding the $\mu$ for which:
$$\frac{d}{d\mu}\ln L(\mu)=0,$$
which again results in the estimate
$$\hat{\mu}=\overline{x}$$

\end{xmpl}

\subsubsection{Detail}
\begin{defn}
If $\left (Y_1, \ldots , Y_n\right )'$ is a random vector from a density $f_{\theta}$ where $\theta$ is an unknown parameter, and $\mathbf{y}$ is a vector of observations then we define the \textbf{likelihood function} to be
$$
L_{\mathbf{y}}(\theta)=f_{\theta}(y).
$$
\end{defn}


%% Slide https://tutor-web.net/math/math612.0/lecture330/slide30
\subsection{Ordinary least squares}
\begin{fbox}
\begin{minipage}{0.58\textwidth}
Consider the regression problem where we fit a line through $(x_i,y_i)$ pairs with $x_1, \ldots, x_n$ fixed numbers but where $y_i$ is measured with error.
\end{minipage}
\hspace{0.5mm}
\begin{minipage}{0.38\textwidth}
\begin{picture}
36
\end{picture}

Figure:  Regression line through data pairs.
\end{minipage}
\end{fbox}
\subsubsection{Details}
The ordinary least squares (OLS) estimates of the parameters $\alpha$ and $\beta$ in the model $y_i=\alpha + \beta x_i + \epsilon_i$ are 
obtained by minimizing the sum of squares
$$
\sum_i \left ( y_i -(\alpha +\beta x_i)  \right )^2
$$

\begin{align*}
a&=\overline{y} - b\overline{x} \\
\\
b&= \frac{\displaystyle\sum^n_{i=1} (x_i-\overline{x})(y_i-\overline{y})}{\displaystyle\sum^n_{i=1} (x_i-\overline{x})^2}
\end{align*}


%% Slide https://tutor-web.net/math/math612.0/lecture330/slide40
\subsection{Random variables and outcomes}
\subsubsection{Details}
Recall that $X_1, \ldots, X_n$ are random varibles (reflecting the population distribution) and $x_1, \ldots, x_n$ are numerical outcomes of these distributions. We use upper case letters to denote random variables and lower case letters to denote outcome or data.
\subsubsection{Examples}
\begin{xmpl}

Let the mean of a population be zero and the $\sigma=4$. Then draw three samples from this population with size, n, either 4, 16 or 64. The sample mean $\bar{X}$ will have a distribution with mean zero and standard deviation of $\frac{\sigma}{\sqrt{n}}$ where n= 4, 16 or 64.
\end{xmpl}

%% Slide https://tutor-web.net/math/math612.0/lecture330/slide50
\subsection{Estimators and estimates}
\begin{fbox}
\begin{minipage}{0.58\textwidth}
In OLS regression, note that the values of $a$ and $b$

$$a = \overline{y} - b \overline{x}$$

$$b = \frac{\sum_{i=1}^{n} (x_i - \overline{x}) (y_i - \overline{y})}{\sum_{i=1}^{n} (x_i - \overline{x})^2}$$

are outcomes of random variables e.g. $b$ is the outcome of 

$$\hat{\beta} = \frac{\sum_{i=1}^{n} (x_i - \overline{x}) (Y_i - \overline{Y})}{\sum_{i=1}^{n} (x_i - \overline{x})^2}$$

the estimator which has some distribution.

\end{minipage}
\hspace{0.5mm}
\begin{minipage}{0.38\textwidth}
\begin{picture}
37
\end{picture}

Figure:  Shows an example of the distribution of the estimator $\hat{\beta}$
\end{minipage}
\end{fbox}
\subsubsection{Details}
The following R commands can be used to generate a distribution for the estimator $\hat{\beta}$
\begin{lstlisting}
library(MASS)
nsim <- 1000 #  replicates
betahat <- NULL
for (i in 1:nsim){
  n <- 20
  x <- seq(1:n)  # Fixed x vector
  y <- 2 + 0.4*x + rnorm(n, 0, 1)
  xbar <- mean(x)
  ybar <- mean(y)
  b <- sum((x-xbar)*(y-ybar))/sum((x-xbar)^2)
  a <- ybar - b* xbar
 betahat <- c(betahat, b)
}
truehist(betahat)
\end{lstlisting}

{\bf Copyright}
2021, Gunnar Stefansson (editor) with contributions from very many students

This work is licensed under the Creative Commons
Attribution-ShareAlike License. To view a copy of this license, visit
http://creativecommons.org/licenses/by-sa/1.0/ or send a letter to
Creative Commons, 559 Nathan Abbott Way, Stanford, California 94305,
USA.
\clearpage
%% Lecture https://tutor-web.net/math/math612.0/lecture340
\section{Test of hypothesis, P values and related concepts}
%% Slide https://tutor-web.net/math/math612.0/lecture340/slide10
\subsection{The principle of the hypothesis test}
\begin{fbox}
\begin{minipage}{0.97\textwidth}
The principle is to formulate a hypothesis and an alternative hypothesis, $H_0$ and $H_a$ respectively, and then select a statistic with a given distribution when $H_0$ is true and select a rejection region which has a specified probability $(\alpha)$ when $H_0$ is true. 

The rejection region is chosen to reflect $H_a$, i.e to ensure a high probability of rejection when $H_a$ is true.

\end{minipage}
\end{fbox}
\subsubsection{Examples}
\begin{xmpl}
Suppose we want to evaluate whether a coin is biased. We can plan an experiment for this. Suppose we toss the coin 5 times and count the number of heads. We can test the following hypothesis simply.\\


$H_0: p = \frac {1} {2}$ where $H_0$ is the null hypothesis

$H_a ; p > \frac {1} {2}$ where $H_a$ is an alternative hypothesis

and $p$ is probability of having a head.\\

We reject $H_0$ if we get all heads. (Assuming the only interest is in a tendency towards larger probabilities). So the probability of rejecting the null hypothesis $H_0$ is:

P[reject $H_0$]= P [ all  heads in 5 trials] $\equiv p^5$

If $H_0$ is true, then P [reject $H_0$] = $\frac {1} {2}$\\

Need to choose 5 trials to ensure $\frac {1} {2^5} = \frac {1} {32} < \frac {1} {32} < 0.05 $\\

i.e. The probability of incorrectly rejecting $H_0$ is less than $\alpha = 0.05$
\end{xmpl}
\begin{xmpl}

Flip a coin to test 

$ H_0: P = \frac {1} {2}$  vs
$H_a: P \neq \frac {1} {2}$\\

Reject, if no heads or all heads are obtained in 6 trials, where the error rate is

P [reject $H_0$ when true] = P [all heads or all tails]
												
 = P[all heads] + P [all tails]
												
 = $\frac {1} {2^6} + \frac {1} {2^6} = 2 \frac {1} {64} = \frac {1} {32} < 0.05$\\\\


A variation of this test is called the sign test, which is used to test hypothesis of the form,

$ H_0$: true median = 0 using a count of the number of positive values.
\end{xmpl}

%% Slide https://tutor-web.net/math/math612.0/lecture340/slide20
\subsection{The one sided z test for normal mean}
\begin{fbox}
\begin{minipage}{0.97\textwidth}
Consider testing

$$ H_0: \mu = \mu_0$$

vs

$$ H_a: \mu > \mu_0$$

 

Where data $x_1 \ldots x_n$ are collected as independent observations of $X_1 \ldots X_n \sim n(\mu, \sigma^2)$ and $\sigma^2$ is known. If $H_0$ is true, then

$$ \bar {x} \sim n (\mu_0, \frac{\sigma^2}{n})$$
So, 

$$Z = \frac{\bar {x} - \mu_0}{\frac{\sigma} {\sqrt{n}}} \sim n (0,1)$$

It follows that,

$$P[Z>z*] = \alpha$$

Where 

$$z* = z_{1-\alpha}$$

So if the data $x_1 \ldots x_n$ are such that,


$$z = \frac{\bar {x} - \mu_0}{\frac{\sigma} {\sqrt{n}}}  > z*$$

Then $H_0 $ is rejected.


\end{minipage}
\end{fbox}
\subsubsection{Examples}
\begin{xmpl}

Consider the following data set:47, 42, 41, 45, 46.

Suppose we want to test the following hypothesis

$$H_0 : \mu = 42 $$

vs

$$H_a : \mu > 42 $$

 $\sigma = 2$ is given

The mean of the given data set can be calculated as

$$\bar {x} = 44.2$$

we can calculate $z$ by using following equation

$$z = \frac{\bar {x} - \mu}{\frac{\sigma} {\sqrt{n}}} = \frac{44.2 - 42}{\frac{2} {\sqrt{5}}} $$

$$z = \frac{2.2}{0.8944} = 2.459$$

$$z* = 1.645$$

Here

 $z> z*$ 

So $ H_0$ is rejected with $\alpha = 0.05$
\end{xmpl}


%% Slide https://tutor-web.net/math/math612.0/lecture340/slide30
\subsection{The two-sided z test for a normal mean}
\begin{fbox}
\begin{minipage}{0.97\textwidth}
$$z: =\frac{\overline{x}-\mu_0}{s\sqrt{n}} \sim n(0,1)$$
\end{minipage}
\end{fbox}
\subsubsection{Details}
Consider testing $H_0: \mu=\mu_0$ versus $H_a: \mu \ne \mu_0$ based on observation from $\overline{X_1},..., \overline{X} \sim n(\mu, \sigma^2)$  i.i.d. where $\sigma^2$ is known.  If $H_0$ is true, then
$$Z: = \frac{\overline{x}-\mu_0}{\sigma \sqrt{n}} \sim n(0,1)$$

and
$$P[|z| > z^\star] = \alpha$$

with $$z^\star = z_{1}$$

We reject $H_0$ if $|z| > z^\star$. If $|z| > z^\star$ is not true, then we "Cannot reject the $H_0$".




 
\subsubsection{Examples}
\begin{xmpl}

In R, you may generate values to calculate the $z$ value. The command that is generally used is: \texttt {quantile}  

To illustrate: 

\begin{lstlisting}
z<-rnorm(1000,0,1)
quantile(z,c(0.025,0.975))
     2.5%     97.5% 
-1.995806  2.009849 
\end{lstlisting}

So, the $z$ value for a two-sided normal mean is $\left |-1.99 \right |$.  
\end{xmpl}

%% Slide https://tutor-web.net/math/math612.0/lecture340/slide40
\subsection{The one-sided t-test for a single normal mean}
\begin{fbox}
\begin{minipage}{0.97\textwidth}
Recall that if $X_1,...,X_n \sim N(\mu,\sigma^2)$ i.i.d. then $$\frac{\overline{X}-\mu}{S/\sqrt{n}}\sim t_{n-1}$$
\end{minipage}
\end{fbox}
\subsubsection{Details}
Recall that if $X_1,\ldots,X_n \sim N(\mu,\sigma^2)$ i.i.d. then $$\frac{\overline{X}-\mu}{S/\sqrt{n}}\sim t_{n-1}$$

To test the hypothesis $H_0:\mu=\mu_{0}$ vs $H_1:\mu > \mu_{0}$ first note that if $H_0$ is true, then 
$$T= \frac{\overline{X}-\mu_{0}}{S/\sqrt{n}} \sim t_{n-1}$$

so $$P[T>t*]=\alpha$$

if $$t*=t_{n-1,1-\alpha}$$

Hence, we reject $H_0$ if the data $x_1,...,x_n$ results in a a value of 
$t:=\frac{\overline{x}-\mu_0}{S/\sqrt{n}}$ such that t>t*, otherwise $H_0$ can not be rejected. 
\subsubsection{Examples}
\begin{xmpl}
 Suppose the following data set (12,19,17,23,15,27) comes independently from a normal distribution and we need to test $H_0:\mu=\mu_0$ vs $H_a:\mu>\mu_0$. Here we have $n=6,\overline{x}=18.83, s=5.46, \mu_0=18$ so we obtain
$$t=\frac{\overline{x}-\mu_0}{s/\sqrt{n}}= 0.37$$

so $H_0$ cannot be rejected.\\

In R, t* is found using qt(n-1,0.95) but the entire hypothesis can be tested using
\begin{lstlisting}
t.test(x,alternative="greater",mu=<$\mu_0$>)
\end{lstlisting}
\end{xmpl}

%% Slide https://tutor-web.net/math/math612.0/lecture340/slide50
\subsection{Comparing means from normal populations}
\begin{fbox}
\begin{minipage}{0.97\textwidth}
Suppose data are gathered independently from two normal populations resulting in \\
$x_1,....,x_n$    and       $y_1,...y_m$
\end{minipage}
\end{fbox}
\subsubsection{Details}
We know that if\\ 
$$X_1, ...., X_n \sim n(\mu_1,\sigma)$$

$$Y_1, ...., Y_m \sim n(\mu_2,\sigma)$$

 
are all independent then
$$\bar{X}-\bar{Y} \sim n(\mu_1-\mu_2,\frac{\sigma^2}{n}+\frac{\sigma^2}{m})$$
Further,
$$\sum_{i=1}^{n} \frac{(X_i-\bar{X})^2}{\sigma^2} \sim X_{n-1}^{2}$$
and
$$\sum_{j=1}^{m} \frac{(Y_j-\bar{Y})^2}{\sigma^2} \sim X_{m-1}^{2}$$
so
$$\frac {\sum_{i=1}^{n}(X_i-\bar{X})^2 + \sum_{j=1}^{m}(Y_j-\bar{Y})^2}{\sigma^2} \sim X_{n+m-2}^2$$
and it follows that
$$\frac {\bar{X}-\bar{Y}-(\mu_1-\mu_2)}{S\sqrt{(\frac{1}{n}+\frac{1}{m})}} \sim t_{n+m-2}$$
where
$$S=\sqrt{\frac{\sum_{i=1}^{n}(X_1-\bar{X})^2+\sum_{j=1}^{m}(Y_j-\bar{Y})^2}{n+m-2}}$$

consider testing $H_0:\mu_1=\mu_2$ vs $H_1=mu_1>\mu_2$. Hence, if $H_0$ is true then the observed value
$$t=\frac{\bar{x}-\bar{y}}{S\sqrt{\frac{1}{n}+\frac{1}{m}}}$$
comes from a t-test with $n+m-2$ df and we reject $H_0$ if $\left|t\right|>t^\ast$. Here,
$$S=\sqrt{\frac{\sum_{i}(x_i-\bar{x})^2+\sum_{j}(y_j-\bar{y})^2}{n+m-2}}$$
and $t^\ast=t_{n+m-2,1-\alpha}$


%% Slide https://tutor-web.net/math/math612.0/lecture340/slide60
\subsection{Comparing means from large samples <Ól.B.M.>}
\begin{fbox}
\begin{minipage}{0.97\textwidth}
If $X_1,....X_n$ and $Y_1,.....Y_m$, are all independent (with finite variance) with expected values of $\mu_1$ and $\mu_2$ respectively, and variances of $\sigma_1^2$,and $\sigma_2^2$ respectively, then 
$$\frac{\overline{X}-\overline{Y}-(\mu_1-\mu_2)}{\sqrt{\frac{\sigma_1^2}{n}+\frac{\sigma_2^2}{m}}} \dot{\sim}n(0,1)$$
if the sample sizes are large enough.\\

This is the central limit theorem.
\end{minipage}
\end{fbox}
\subsubsection{Details}
Another theorem (Slutzky) stakes that replacing $\sigma_1^2$ and $\sigma_2^2$ with $S_1^2$ and $S_2^2$ will result in the same (limiting) distribution.\\

It follows that for large samples we can test
$$H_0: \mu_1=\mu_2 \qquad vs. \qquad H_a:\mu_1 > \mu_2$$
by computing $$z=\frac{\overline{x}-\overline{y}}{\sqrt{\frac{s_1^2}{n}+\frac{s_2^2}{m}}}$$
and reject $H_0$ if $z>z_{1-\alpha}$.


%% Slide https://tutor-web.net/math/math612.0/lecture340/slide70
\subsection{The P-value}
\begin{fbox}
\begin{minipage}{0.97\textwidth}
The p-value of a test is an evaluation of the probability of obtaining results which are as extreme as those observed in the context of the hypothesis.
\end{minipage}
\end{fbox}
\subsubsection{Examples}
\begin{xmpl}
Consider a dataset and the following hypotheses \\
$$H_0:\mu=42$$

vs. $$H_a:\mu>42$$
and suppose we obtain\\
$$z=2.3$$
We reject $H_0$ since $$2.3>1.645+z_{0.95}$$
The p-value is
 $$P[Z>2.3]= 1-\Phi(2.3)$$
obtained in R using
\begin{lstlisting}
1-pnorm(2.3)
[1] 0.01072411
\end{lstlisting}

If this had been a two tailed test, then 
\begin{align*}
P &= P[|Z|>2.3]\\
\quad &=P[Z<-2.3]+P[Z>2.3]\\
\quad &=2\cdot P[Z>2.3]
\end{align*}
\end{xmpl}

%% Slide https://tutor-web.net/math/math612.0/lecture340/slide80
\subsection{The concept of significance}
\subsubsection{Details}
Two sample means are statistically $\underline{significantly}$ $\underline{different}$ if their null hypothesis ($\mu_1 = \mu_2$)can be $\underline{rejected}$. In this case, one can make the following statements:
\begin{itemize}
\item
The population means are different.
\item
The sample means are significantly different.
\item
$\mu_1 \ne \mu_2$
\item
$\bar{x}$ is significantly different from $\bar{y}$.
\end{itemize}
But one does not say:
\begin{itemize}
\item
The sample means are different.
\item
The population means are different with probability 0.95.
\end{itemize}
Similarly, if the hypothesis $H_0: \mu_1 = \mu_2$ can not be rejected, we can say:
\begin{itemize}
\item
There is no significant difference between the sample means.
\item
We can not reject the equality of population means.
\item
We can not rule out...
\end{itemize}
But we can not say:
\begin{itemize}
\item
The sample means are equal.
\item
The population means are equal.
\item
The population means are equal with probability 0.95.
\end{itemize}

{\bf Copyright}
2021, Gunnar Stefansson (editor) with contributions from very many students

This work is licensed under the Creative Commons
Attribution-ShareAlike License. To view a copy of this license, visit
http://creativecommons.org/licenses/by-sa/1.0/ or send a letter to
Creative Commons, 559 Nathan Abbott Way, Stanford, California 94305,
USA.
\clearpage
%% Lecture https://tutor-web.net/math/math612.0/lecture350
\section{Power and sample sizes}
%% Slide https://tutor-web.net/math/math612.0/lecture350/slide10
\subsection{The power of a test}
\begin{fbox}
\begin{minipage}{0.97\textwidth}
Suppose we have a method to test a null hypothesis against an alternative hypothesis. The test would be "controlled" at some level $\alpha$, i.e.
$ P[reject~H_0] \leq \alpha $ whenever $H_0$ is true.\\ 

On the other hand, when $H_0$ is false one wants $P[reject ~ H_0]$ to be as high as possible.\\

If the parameter to be tested is $\theta$ and $\theta_0$ is a value within
$H_0$ and $\theta_a$ is in $H_a$ then we want $ P_{\theta_0}[reject~H_0] \leq \alpha $ and
$ P_{\theta_a}[reject~H_0] $ as large as possible.\\

For a general $\theta$ we write
$$ \beta(\theta) = P_{\theta} [reject~H_0] $$
for the power of the test 

\end{minipage}
\end{fbox}
\subsubsection{Details}
Do not use the phrase "accept".




%% Slide https://tutor-web.net/math/math612.0/lecture350/slide15
\subsection{The power of tests for proportions}
\begin{fbox}
\hspace{0.5mm}
\begin{minipage}{0.97\textwidth}
\begin{picture}
38
\end{picture}


\end{minipage}
\end{fbox}
\subsubsection{Examples}
\begin{xmpl}
Suppose 7 students are involved in an experiment which is comprised of 7 trails and each trial consists of rolling a dice 9 times.\\
\\
Experiment 1: A student records a 0 if they toss an even number (2,4,6), and records a 1 if they toss an odd number (1,3,5). After tossing the dice 9 times and recording a 0 or 1 the student tabulates the number of 1s. This process is repeated 6 more times. \\
\\
Data and outcomes: 
x = number of successes in n trials =$\sum_{i=1}^n$.
Thus, x = number of odd numbers\\\\

Question:Test whether $p=P[odd number]=\frac{1}{2}$ that is 
		
$H_0: p=\frac{1}{2}$ vs. $H_a: p\neq \frac{1}{2}$
 \\\\
Solution:
Now, x is an outcome of $X \sim Bin(n,p)$. We know from the CLT that
$$\frac{X-np}{\sqrt{np(1-p)}} \sim\dot N(0,1)$$

write $p_0=\frac{1}{2}$ so if $H_0:p=p_0$ is true then
$$Z:=\frac{X-np_0}{\sqrt{np_0(1-p_0)}}\sim\dot N(0,1)$$
so we reject $H_0$ if the observed value
$$z=\frac{x-np_0}{\sqrt{np_0(1-p_0)}}$$
is such that 
$\left | z\right | >z_{1-\frac{\alpha}{2}}$



Outcomes from 21 trials 


$\begin{matrix}
7	& 4 & 4\\
3 & 4 & 6\\
5 & 3 & 4\\
5 & 5 & 3\\
6 & 4 & 5\\
4 & 3 & 5\\
3 & 6 & 7
\end{matrix}$

$$z=\frac{7-9\cdot \frac{1}{2}}{\sqrt{9\cdot\frac{1}{2}\cdot \frac{1}{2}}}=\frac{7-4.5}{3\cdot\frac{1}{2}}=\frac{14-9}{3}=\frac{5}{3} < 1.96$$
So we do not reject the null hypothesis! 
\\
\begin{notes}
Note that we can rewrite the test statistics slightly

$$z=\frac{x-\frac{n}{2}}{\sqrt{n\cdot\frac{1}{2}\cdot\frac{1}{2}}}=\frac{x-\frac{9}{2}}{3\cdot\frac{1}{2}}=\frac{2x-9}{3}$$
\end{notes}
\begin{notes}
Note that we reject if $\frac{2x-9}{3}>1.96$ i.e. if $2x>9+3\cdot1.96 \approx 9+6=15$\\

$x>7.5$ [for x=8 or 9] or $2x<9-3\cdot1.96, x<1.5$ [for x=0 or 1].
\end{notes}
\end {xmpl}
\begin{xmpl}
Suppose 7 students are involved in an experiment which is comprised of 7 trails and each trial consists of rolling a dice 9 times.\\
\\

Experiment 2: The procedure is the same as in experiment 1,  but now the student records 0 for a 1 or 2 and a 1 for a 3,4,5,or 6. \\\\

Data and outcomes: 

x = number of successes in n trials =$\sum_{i=1}^n$ 
Thus, x = number of 'b's\\\\

Solution:
Outcomes from 21 experiments


$\begin{matrix}
5 & 4 & 3\\
8 & 5 & 7\\
5 & 7 & 3\\ 
7 & 6 & 5\\ 
7 & 8 & 8\\ 
5 & 6 & 4\\ 
2 & 5 & 7
\end{matrix}$

This time our test is $H_0:p=\frac{2}{3}$ vs $H_a:p=\frac{2}{3}$. Note that we reject $H_0$ if $\frac{6x-4n}{9}>1,96$ [for x=9] or if $\frac{6x-4n}{9}<-1,96$ [for x=0,1,2,3]. \\

We reject $H_0$ in 3 out of 21 trials.
\end{xmpl}
\begin{xmpl}
Suppose 7 students are involved in an experiment which is comprised of 7 trails and each trial consists of rolling a dice 9 times.\\
\\

Experiment 3: Same as experiment 1 except 0 is recorded for 1,2,3,4,5 and a 1 is recorded for 6. \\
\\
Data and outcomes: 

x = number of successes in n trials =$\sum_{i=1}^n$ 
Thus, x = number of '1's \\\\

Solution:
Outcomes from 21 experiments

$\begin{matrix}
0 & 1 & 2\\
1 & 2 & 1\\
1 & 4 & 2\\ 
1 & 1 & 1\\
1 & 3 & 1\\ 
1 & 1 & 2\\ 
0 & 2 & 0
\end{matrix}$

With the same kind of calculations as above, we find that we reject the null hypothesis $H_0:p=\frac{1}{6}$ in 14 out of 21 trials. 
\end{xmpl}


%% Slide https://tutor-web.net/math/math612.0/lecture350/slide20
\subsection{The Power of the one sided z test for the mean}
\begin{fbox}
\begin{minipage}{0.97\textwidth}
The one sided z-test for the mean $(\mu)$ is based on a random sample where $X_1 \ldots X_n \sim n(\mu, \sigma^2)$ are independent and $\sigma^2$ is known.\\

The power of the test for an arbitrary $\mu $ can be computed as:

$$
\beta(\mu) = 1 - \Phi \left ( \frac{\mu_0 - \mu }{\frac{\sigma} {\sqrt{n}}} + z_{1- \alpha} \right )
$$

\end{minipage}
\end{fbox}
\subsubsection{Details}
The one sided z-test for the mean $(\mu)$ is based on a random sample where $X_1 \ldots X_n \sim n(\mu, \sigma^2)$ are independent and $\sigma^2$ is known.\\

If the hypotheses are:

$ H_0 : \mu = \mu_0$ vs

$ H_a : \mu > \mu_0$\\

Then we know that, if $H_0$ is true

$$Z = \frac{\bar {X} - \mu_0}{\frac{\sigma} {\sqrt{n}}} \sim n (0,1)$$


Given data $x_1, \ldots x_n$, the z-value is

$$z = \frac{\bar {x} - \mu_0}{\frac{\sigma} {\sqrt{n}}}$$

We reject $H_0$ if $z > z_{1-\alpha}$

The level of this test is 

$$
P_{\mu_0} [Reject H_0]= P_{\mu_0}[\frac{\bar {X} - \mu_0}{\frac{\sigma} {\sqrt{n}}} > z_{1- \alpha}]
$$

 $$= P[z > z_{1- \alpha}] = {\alpha} $$

since $ Z \sim n (0,1)$ when $\mu_0$ is the true value.\\\\
 

The power of the test for an arbitrary $\mu $ can be computed as follows.
 
 $$\beta(\mu) = P_{\mu} [reject H_0]$$

 $$= P_{\mu}[\frac{\bar {X} - \mu_0}{\frac{\sigma} {\sqrt{n}}} > z_{1- \alpha}]$$

 $$= P_{\mu} [\bar {X}> \mu_0 + z_{1- \alpha}{\frac{\sigma} {\sqrt{n}}}]$$

$$= P_{\mu} [\frac{\bar {X} - \mu}{\frac{\sigma} {\sqrt{n}}}> \frac{\mu_0 - \mu }{\frac{\sigma} {\sqrt{n}}}+ z_{1- \alpha}]$$

$$ = P[Z > \frac{\mu_0 - \mu}{\frac{\sigma} {\sqrt{n}}} + z_{1- \alpha}]$$

We obtain

$$
\beta(\mu) = 1 - \Phi \left ( \frac{\mu_0 - \mu }{\frac{\sigma} {\sqrt{n}}} + z_{1- \alpha} \right )
$$


\subsubsection{Examples}
\begin{xmpl}
Suppose we know $\sigma=2$ and we will take a sample 
from $n\left ( \mu, \sigma^2\right)$
intending to test the hypothesis $\mu=3$ at level $\alpha=0.05$. We want to know the power
against a one-tailed alternative when the true mean is actually $\mu=4$ when the sample size is $n=25$. \\\\

We can set this up in R with:
\begin{lstlisting}
alpha<-0.05
n<-25
sigma<-2
mu0<-3
mu<-4
zcrit<-qnorm(1-alpha)
\end{lstlisting}


Sticking the formula into R gives
\begin{lstlisting}
1-pnorm((mu0-mu)/(sigma/sqrt(n))+zcrit)
[1] 0.803765
\end{lstlisting}

On the other hand, one can also use a simple simulation approach. First, decide how many samples are to be simulated (Nsim). Then, generate all of these samples, arrange them in a matrix and compute the mean of each sample.  The z-value of each of these Nsim tests are then computed and a check is made whether it exceeds the critical point (1) or not (0).
\begin{lstlisting}
Nsim<-10000
m<-matrix(rnorm(Nsim*n,mu,sigma),ncol=n)
mn<-apply(m,1,mean)
z<-(mn-mu0)/(sigma/sqrt(n))
i<-ifelse(z>zcrit,1,0)
sum(i/Nsim)
[1] 0.8081
\end{lstlisting}
\end {xmpl}

%% Slide https://tutor-web.net/math/math612.0/lecture350/slide30
\subsection{Power and sample size for the one-sided z-test for a single normal mean}
\begin{fbox}
\begin{minipage}{0.97\textwidth}
Suppose we want to test $H_0:\mu=\mu_0$ vs $H_a:\mu>\mu_0$. We will reject $H_0$ if the observed value $$z=\frac{\overline{x}-\mu_0}{\sigma/\sqrt{n}}$$

is such that $z>z_{1-\alpha}$. 

\end{minipage}
\end{fbox}
\subsubsection{Details}
Suppose we want to test $H_0:\mu=\mu_0$ vs $H_a:\mu>\mu_0$. So based on $X_1,...,X_n\sim n(\mu,\sigma ^2)$ i.i.d. with $\sigma^2$ known we will reject $H_0$ if the observed value $$z=\frac{\overline{x}-\mu_0}{\sigma/\sqrt{n}}$$

is such that $z>z_{1-\alpha}$. The power is given by: $$\beta(\mu)=1-\Phi(\frac{\mu-\mu_0}{\sigma/\sqrt{n}}+z_{1-\alpha})$$

and describes the probability of rejecting $H_0$ when $\mu$ is the correct value of the parameter. Suppose we want to reject $H_0$ with a prespecified probability $\beta_1$, when $\mu_1$ is the true value of $\mu$. For this, we need to select the sample size so that 
$$\beta(\mu_1) \geq \beta_1$$

i.e. find n which satisfies
$$1-\Phi(\frac{\mu_1-\mu_0}{\sigma/\sqrt{n}}+z_{1-\alpha})\geq \beta_1$$

\subsubsection{Examples}
\begin{xmpl}
\begin{lstlisting}
mu0<-10
sigma<-2
mu1<-11
n<-50
d<-(mu1-mu0)

power.t.test(n=n,delta=d,sd=sigma,sig.level=0.05,type="one.sample",alternative="one.sided",strict
+ = TRUE)

     One-sample t test power calculation 

              n = 50
          delta = 1
             sd = 2
      sig.level = 0.05
          power = 0.9672067
    alternative = one.sided
\end{lstlisting}
\end{xmpl}

%% Slide https://tutor-web.net/math/math612.0/lecture350/slide40
\subsection{The non central t - distribution}
\begin{fbox}
\begin{minipage}{0.97\textwidth}
Recall that if $Z \sim n(0, 1)$ and $ U \sim {\chi^2}_v$ are independent then

$$\frac{Z}{\sqrt{\frac{U}{v}}}\sim t_v$$

and it follows for a random sample $X_1 \ldots X_n \sim n(\mu, \sigma^2)$ independent; that


$$\frac{\bar {X} - \mu}{\frac{s} {\sqrt{n}}} = \frac{\frac{\bar {X} - \mu}{\frac{\sigma} {\sqrt{n}}}}{\sqrt{\frac{\sum ({X_i} -\bar {X})^2} { \frac {{\sigma}^2} {n-1}}}} \sim t_{n-1}$$

\end{minipage}
\end{fbox}
\subsubsection{Details}
On the other hand, if $W \sim n (\Delta,1) $ and $U \sim {\chi}^2_v $ are independent, then $\frac{W}{\sqrt{\frac{U}{v}}}$ has a non central t-distribution with $v$ degrees of freedom and non centrality parameter $\Delta$. This distribution arises, if $X_1 \ldots X_n \sim n(\mu, \sigma^2)$ independent and we want to consider the distribution of: 

$$\frac{\bar {X} - \mu}{\frac{S} {\sqrt{n}}} = \frac{\frac{\bar {X} - \mu}{\frac{\sigma} {\sqrt{n}}} + \frac{\mu - \mu_0 }{\frac{\sigma} {\sqrt{n}}}} {\frac{S}{\sqrt{n}}} = \frac {Z + \frac{\mu - \mu_0 }{\frac{\sigma} {\sqrt{n}}}}{\sqrt{\frac{U}{v}}}$$

Where $\mu \neq \mu_0$ which is a non central t with non centrality parameters

$$ \Delta = \frac{\mu - \mu_0 }{\frac{\sigma} {\sqrt{n}}}$$

with $n-1$ df. Here $ v = n-1 df$ since  $Z \sim n (0,1) $ and $U \sim {\chi}^2_{n-1} $ in this equation 




%% Slide https://tutor-web.net/math/math612.0/lecture350/slide50
\subsection{The power of t-test for a normal mean (warning: errors)}
\subsubsection{Details}
\Large{\textbf{WARNING: This is all wrong and needs to be rewritten}}

Consider $X_1,\ldots,X_n \sim n (\mu, \sigma^2)$ i.i.d. where $\sigma^2$ is unknown and we want to test $H_0:\mu = \mu_0$ vs. $H_a: \mu > \mu_0$.  We know that

$$T:= \frac{\overline{X} - \mu}{s/\sqrt{n}} \sim_{t_n-1}$$

and we will reject $H_0$ if the computed value $$t:= \frac{\overline{x} - \mu_0}{s/\sqrt{n}}$$

is such that $$t>t^{\star}=t_{n-1, 1-\alpha}.$$

The power of this test is:

\begin{align*}
B(\mu) = P_{\mu}[reject\ H_0] &= P_{\mu} \lbrack \frac{\overline{x} - \mu_0}{s/\sqrt{n}} > t^\star \rbrack\\
&=P_{\mu} \lbrack \overline{x} - \mu_0 > t^\star\cdot s/\sqrt{n} \rbrack\\
&=P_{\mu} \lbrack \frac{\overline{x} - \mu}{s/\sqrt{n}} > t^\star+\frac{\mu_0-\mu}{s/\sqrt{n}} \rbrack.
\end{align*}

Which is the probability that a  $t_{n-1,1-\alpha}$-variable exceed $t^{\star}+\frac{\mu_0-\mu}{s/\sqrt{n}}$.

\Large{\textbf{WARNING: This is all wrong and needs to be rewritten (the s in the above line is a random variable so this make no sense at all)}}




%% Slide https://tutor-web.net/math/math612.0/lecture350/slide60
\subsection{Power and sample size for the one sided t-test for a mean}
\begin{fbox}
\begin{minipage}{0.97\textwidth}
Suppose we want to calculate the power of a one sided t-test for a single mean (one sample), this can easily be done in R with the \texttt{power.t.test} command.
\end{minipage}
\end{fbox}
\subsubsection{Details}
$\triangle = \mu_1-\mu_2$\\

$\delta = \frac{\mu_1-\mu_2}{\sigma/\sqrt{n}}$
\subsubsection{Examples}
\begin{xmpl}
For a one sided power analysis we wish to test the following hypotheses:\\

For a one sample test:\\
$H_0: \mu = \mu_0$ vs. $H_a: \mu > \mu_0$\\

For a two sample test:\\
$H_0: \mu_1 = \mu_2$ vs. $H_a: \mu_1 > \mu_2$\\

In R, the \texttt{power.t.test} command is useful to calculate how many samples one needs to obtain
a certain power of a test, but also to calculate the power when we have a given number of samples.
\end{xmpl}
\begin{xmpl}
How many samples do I need to get a power of .9?
\begin{lstlisting}
power.t.test(power = .95, delta=1.5,sd=2, type="one.sample", alternative = "one.sided")

 One-sample t test power calculation 

              n = 20.67702
          delta = 1.5
             sd = 2
      sig.level = 0.05
          power = 0.95
    alternative = one.sided
    
\end{lstlisting}
We would thus need a sample size of n = 31.15 or $\approx 32$ samples to obtain a power of 0.9 for
our analysis.\\~
\end{xmpl}
\begin{xmpl}
With a sample size of n = 45, what will the power of my test be?\\

\begin{lstlisting}
power.t.test(n=45,delta=1.5,sd=2,sig.level=0.05,type="one.sample",alternative="one.sided")

      One-sample t test power calculation 
     
                   n = 45
               delta = 1.5
                  sd = 2
           sig.level = 0.05
               power = 0.9995287
         alternative = one.sided

\end{lstlisting}

This is done the same way for two samples only by changing the alternative to "two.sample". For two sided power analysis, one only needs to change the alternative to "two.sided".\\~
\end{xmpl}
\begin{xmpl}
If one is interested in doing a power analysis for an ANOVA test, this is done in a fairly similar way.\\~\\
With a given sample size of n=20:
\begin{lstlisting}
power.anova.test(groups=4, n=20, between.var=1, within.var=3)
\end{lstlisting}
  Balanced one-way analysis of variance power calculation 
\begin{lstlisting}
         groups = 4
              n = 20
    between.var = 1
     within.var = 3
      sig.level = 0.05
          power = 0.9679022

\end{lstlisting}

To calculate the sample size needed to obtain a power of 0.90 for a test:

\begin{lstlisting}
power.anova.test(groups=4, between.var=1, within.var=3, power=.9) 
\end{lstlisting}
Balanced one-way analysis of variance power calculation 
\begin{lstlisting}
         groups = 4
              n = 15.18834
    between.var = 1
     within.var = 3
      sig.level = 0.05
          power = 0.9
\end{lstlisting}
\end{xmpl}

%% Slide https://tutor-web.net/math/math612.0/lecture350/slide80
\subsection{The power of the 2-sided t-test}
\begin{fbox}
\begin{minipage}{0.97\textwidth}
A power analysis on a two-sided t-test can be done in R using the $power.t.test$ command.
\end{minipage}
\end{fbox}
\subsubsection{Details}
For a one sample test:\\
$H_0: \mu=\mu_0$ vs. $H_a:\mu\neq\mu_0$\\
\\
The $power.t.test$ command is useful to provide information for determining the minimum sample size one needs to obtain a certain power of a test:
\begin{lstlisting}
 power.t.test(n= ,delta= ,sd= ,sig.level= ,power= ,type=c("two.sample","one.sample","paired"),alternative=c("two.sided"))
\end{lstlisting}
where:\\
n=sample size\\
d=effect size\\
sd=standard deviation\\
sig.level=significance level\\
power= normally 0.8, 0.9 or 0.95 \\
type= two sample, one sample or paired (the type selected depends on the research)\\
alternative= either one sided or two sided\\
\subsubsection{Examples}
\begin{xmpl}
How many samples do I need in my research to obtain a power of 0.8?
\begin{lstlisting}
power.t.test(delta=1.5,sd=2,sig.level=0.05,power=0.8,type=c("two.sample"),alternative=c("two.sided"))

     Two-sample t test power calculation 

              n = 28.89962
          delta = 1.5
             sd = 2
      sig.level = 0.05
          power = 0.8
    alternative = two.sided

\end{lstlisting}
So, one needs 29 samples (n=29) to obtain a power level of 0.8 for this analysis.
\end{xmpl}

%% Slide https://tutor-web.net/math/math612.0/lecture350/slide90
\subsection{The power of the 2-sample one and two-sided t-tests}
\begin{fbox}
\begin{minipage}{0.97\textwidth}
The power of a two sample, one-sided t-test can be computed as follows:

$$ \beta_{(\mu_1\mu_2)} = P_{\mu_1\mu_2}\left[ \frac{Z + \Delta}{\sqrt{U/(n+m-2)}} > t^\ast_{1-\alpha,n+m-2} \right] $$

and the power of a two sample, two-sided t-test is give by: 

$$\beta_{(\mu_1\mu_2)} = P_{\mu_1\mu_2}\left[ \frac{Z + \Delta}{\sqrt{U/(n+m-2)}} > t^\ast_{1-\alpha,n+m-2} \right]\\
+P_{\mu_1\mu_2}\left[ \frac{Z + \Delta}{\sqrt{U/(n+m-2)}} < -t^\ast_{1-\alpha,n+m-2} \right]$$

where $\Delta = \frac{(\mu_1-\mu_2)}{\sigma\sqrt{\frac{1}{n}+\frac{1}{m}}} $ and 
$U$ is the SSE.\\



\end{minipage}
\end{fbox}
\subsubsection{Details}
{\emph Two Sample, One-sided t-Test}:

Suppose data are gathered independently from two normal populations resulting in
 $$X_1, \ldots, X_n \sim n(\mu_1, \sigma^2)$$
 $$Y_1, \ldots, Y_m \sim n(\mu_2, \sigma^2)$$
where all data are independent then $$ \overline{X}-\overline{Y} \sim n(\mu_1 - \mu_2, \frac{\sigma^2}{n}\ + \frac{\sigma^2}{m})$$
The null hypothesis in question is $H_o: \mu_1 = \mu_2$ versus alternative $H_a: \mu_1 > \mu_2$.
If $H_o$ is true then the observed value
$$ t = \frac{\overline{x}-\overline{y}}{s\sqrt{\frac{1}{n}+\frac{1}{m}}} $$
comes from a t-distribution with $n+m-2$ degrees of freedom and we reject $H_o$ if $\left|t \right|> t^\ast_{1-\alpha,n+m-2}$\\

The power of the test can be computed as follows:

\begin{eqnarray*}
\beta_{(\mu_1\mu_2)}& = & P_{\mu_1\mu_2}\left[reject~H_o \right]\\
& = & P_{\mu_1\mu_2}\left[\frac{\overline{X}-\overline{Y}}{S\sqrt{\frac{1}{n}+\frac{1}{m}}} > t^\ast_{1-\alpha,n+m-2} \right]\\
& = &  P_{\mu_1\mu_2}\left[\frac{\frac{\overline{X}-\overline{Y}-(\mu_1-\mu_2)}{\sigma\sqrt{\frac{1}{n}+\frac{1}{m}}}+ \frac{(\mu_1-\mu_2)}{\sigma\sqrt{\frac{1}{n}+\frac{1}{m}}}}{S/\sigma} > t^\ast_{1-\alpha,n+m-2}\right]\\
& = &  P_{\mu_1\mu_2}\left[\frac{Z +\frac{(\mu_1-\mu_2)}{\sigma\sqrt{\frac{1}{n}+\frac{1}{m}}}}
{S/\sqrt{(n+m-2)}} > t^\ast_{1-\alpha,n+m-2} \right]\\
& = &  P_{\mu_1\mu_2}\left[ \frac{Z + \Delta}{\sqrt{U/(n+m-2)}} > t^\ast_{1-\alpha,n+m-2} \right]
\end{eqnarray*}

where $\Delta = \frac{(\mu_1-\mu_2)}{\sigma\sqrt{\frac{1}{n}+\frac{1}{m}}} $ and 
$U$ is the SSE of the samples which is divided by the appropriate degrees of freedom to give a $\chi^2$ distribution.\\

This is the probability that a non-central $t$-variable exceeds $t^\ast$.\\\\

{\emph Two Sample, Two-sided t-Test}:

In this case the null hypothesis is defined as $H_o: \mu_1 = \mu_2$ versus alternative $H_a: \mu_1 \neq \mu_2$. \\

The power of the test can be computed as follows:\\

\begin{eqnarray*}
\beta_{(\mu_1\mu_2)}& = & P_{\mu_1\mu_2}\left[reject~H_o \right]\\
& = & P_{\mu_1\mu_2}\left[\left|\frac{\overline{X}-\overline{Y}}{S\sqrt{\frac{1}{n}+\frac{1}{m}}}\right| > t^\ast_{1-\alpha,n+m-2} \right]\\
& = & P_{\mu_1\mu_2}\left[\frac{\overline{X}-\overline{Y}}{S\sqrt{\frac{1}{n}+\frac{1}{m}}} > t^\ast_{1-\alpha,n+m-2} \right] \\
&& +\: P_{\mu_1\mu_2} \left[\frac{\overline{X}-\overline{Y}}{S\sqrt{\frac{1}{n}+\frac{1}{m}}} <-t^\ast_{1-\alpha,n+m-2} \right] \\
& = &  P_{\mu_1\mu_2}\left[\frac{\frac{\overline{X}-\overline{Y}-(\mu_1-\mu_2)}{\sigma\sqrt{\frac{1}{n}+\frac{1}{m}}}+\frac{(\mu_1-\mu_2)}{\sigma\sqrt{\frac{1}{n}+\frac{1}{m}}}}{S/\sqrt{(n+m-2)}} > t^\ast_{1-\alpha,n+m-2}\right] \\
&& +\:   P_{\mu_1\mu_2}\left[\frac{\frac{\overline{X}-\overline{Y}-(\mu_1-\mu_2)}{\sigma\sqrt{\frac{1}{n}+\frac{1}{m}}}+\frac{(\mu_1-\mu_2)}{\sigma\sqrt{\frac{1}{n}+\frac{1}{m}}}}{S/\sqrt{(n+m-2)}} < -t^\ast_{1-\alpha,n+m-2}\right]
\\
& = &  P_{\mu_1\mu_2}\left[ \frac{Z + \Delta}{\sqrt{U/(n+m-2)}} > t^\ast_{1-\alpha,n+m-2} \right]\\
&& +\: P_{\mu_1\mu_2}\left[ \frac{Z + \Delta}{\sqrt{U/(n+m-2)}} < -t^\ast_{1-\alpha,n+m-2} \right]
\end{eqnarray*}

where $\Delta = \frac{(\mu_1-\mu_2)}{\sigma\sqrt{\frac{1}{n}+\frac{1}{m}}} $ and 
$U$ is the SSE of the samples which is divided by the appropriate degrees of freedom to give a $\chi^2$ distribution.\\

\begin{notes}
Note that the power of a test can be obtained using the $power.t.test$ function in R.
\end{notes}

%% Slide https://tutor-web.net/math/math612.0/lecture350/slide95
\subsection{Sample sizes for two-sample one and two-sided t-tests}
\begin{fbox}
\begin{minipage}{0.97\textwidth}
The sample size should always satisfy the desired power. 



\end{minipage}
\end{fbox}
\subsubsection{Details}
Suppose we want to reject the $H_o$ with a pre-specified probability $\beta_1$ when $\mu_1$ and $\mu_2$ are true values of $\mu$. For this, we need to select the sample size $n$ and $m$ so that $\beta_(\mu_1\mu_2) \geq \beta_1$ i.e. find $n$ and $m$ which satisfies 
$$ P_{\mu_1\mu_2} \left[ \frac{Z + \Delta}{\sqrt{U/(n+m-2)}} > t^\ast_{1-\alpha,n+m-2} \right] $$

 
for a two sample, one-sided t-test. \\

Similarly for a two sample, two-sided t-test we need to find $n$ and $m$ that satisfies 

$ P_{\mu_1\mu_2}\left[ \frac{Z + \Delta}{\sqrt{U/(n+m-2)}} > t^\ast_{1-\alpha,n+m-2} \right]$ +
$ P_{\mu_1\mu_2} \left[\frac{Z + \Delta}{\sqrt{U/(n+m-2)}} < -t^\ast_{1-\alpha,n+m-2} \right]$



%% Slide https://tutor-web.net/math/math612.0/lecture350/slide97
\subsection{A case study in power}
\begin{fbox}
\begin{minipage}{0.97\textwidth}
Want to compute power in analysis of covariance:
$$
y_{ij}=\mu_i+\beta x_{ij}+\epsilon_{ij} , \ i=1, 2,\ j=1,\ldots J,
$$
where $\epsilon_{ij}\sim n(0,\sigma^ 2)$ are i.i.d.?\\

This can be done by simulation and can easily be expanded to other cases.


\end{minipage}
\end{fbox}
\subsubsection{Handout}
\begin{xmpl}
If you want to compute a power analysis in analysis of covariance:
$$
y_{ij}=\mu_i+\beta x_{ij}+\epsilon_{ij} , \ i=1, 2,\ j=1,\ldots J,
$$
where $\epsilon_{ij}\sim n(0,\sigma^ 2)$ are i.i.d. then use simulation. \\


To do this one needs to first define the task in more detail, along with what exactly is known and what the assumptions are. 

\begin{notes}
Note that there are only two groups, with intercepts $\mu_1$ and $\mu_2$.  The "power" will refer to the power of a test for $\mu_1=\mu_2$, 
i.e. we want to test whether the group means are equal, correcting for the effect of the continuous variable $x$.  
\end{notes}

In principle, the $x$-values will be either fixed a priori or they may be a random part of the experiment.  Here we will assume that the $x$-values are randomly selected in the range 20-30 (could e.g. be the ages of patients).\\

Since this is in the planning stage of the experiment, we also have a choice of the sample size within each group.  For convenience, the sample sizes are taken to be the same in each group, $J$ so the total number of measurements will be $n=2J$.  We also need to decide at which levels of $\mu_1$ and $\mu_2$ the power is to be computed (but it is really only a function of the difference, $\mu_1-\mu_2$).\\

The following pieces of R code can be saved into a file,  "ancovapow.r" and then command 
\begin{lstlisting}
source("ancovapow.r")
\end{lstlisting}
can be used to run the whole thing.\\

The beginning of the command sequence merely consists of comments and definitions of parameter values.  These need to be changed for each case separately.

\begin{lstlisting}
#
# ancovapow.r - power computations for analysis of covarariance
#             - one factor, two levels mu0, mu1
#             - one covariate x, x0 stores possible values from which a random set is chosen
#
# first set values of parameters
#
alpha<-0.05
sigma<-7.5        # the common standard deviation
x0<-20:30         # the set of x values
delta<-10         # the difference in the means
mu0<-0            # the first mean
mu1<-mu0+delta    # the second mean
slope<-2.5        # the slope in the ancova
J<-10             # the common sample size per factor level
n<-2*J            # the total sample size
Nsim<- 40000      # the number of simulations for power computations
\end{lstlisting}

Rather than head straight for the ancova, start with a simpler case, namely ignoring the covariate ($x$) and merely doing a regular two-sample, two-tailed t-test.  This should be reasonably similar to the ancova power computations anyway.

\begin{lstlisting}
#
# Next do the power computations just for a regular two-sided, two-sample t-test
# and use simulation
#
Y1<-matrix(rnorm(J*Nsim,mu0,sigma),ncol=J) # Simulate Nsim samples of size J, ea n(mu1,sigma^2)
Y2<-matrix(rnorm(J*Nsim,mu1,sigma),ncol=J) # Simulate Nsim samples of size J, ea n(mu2,sigma^2)
y1mn<-apply(Y1,1,mean)                      # compute all the simulated y1-means
y2mn<-apply(Y2,1,mean)                      # compute all the simulated y2-means
sy1<-apply(Y1,1,sd)                         # compute all the simulated y1-std.devs
sy2<-apply(Y2,1,sd)                         # compute all the simulated y2-std.devs
s<-sqrt(((J-1)*sy1^2+(J-1)*sy2^2)/(n-2))    # compute all the pooled std.devs
t<-(y1mn-y2mn)/(s*sqrt(1/J+1/J))            # compute all the Nsim t-statistics
i<-ifelse(abs(t)>qt(1-alpha/2,n-2),1,0)     # for ea t, compute 1=reject, 0=do not reject
powsim2<-sum(i)/Nsim                        # the simulated power
cat("The simulated power is ",powsim2,"\n")
\end{lstlisting}

The above gave the simulated power.  In R there is a function to do the same computations and it is worth while to verify the code (and approach) by checking whether these give the same thing:

\begin{lstlisting}
#
# Then compute the exact power for the t-test
#
pow2<-power.t.test(delta=delta,sd=sigma,sig.level=alpha,n=J ,type=c("two.sample"),alternative=c("two.sided"))
cat("The exact power:\n")
print(pow2)
\end{lstlisting}

Finally, start setting up the code to do the ancova simulations.  Note that for this we need to generate the x-values.  In this example it is assumed that the x-values are not under the control of the experimenter but arrive randomly, in the range from 20 to 30 (could e.g. be the age of participants in an experiment).

\begin{lstlisting}
#
# Finally compute the power in the ancova - note we already have simulated Y1, Y2-values but have not added the x-part yet
#
x1<-matrix(sample(x0,Nsim*J,replace=T),ncol=J) # simulate x-values for y1
x2<-matrix(sample(x0,Nsim*J,replace=T),ncol=J) # simulate x-values for y2
Y1<-Y1+slope*x1
Y2<-Y2+slope*x2
fulldat<-cbind(Y1,Y2,x1,x2) # a row now contains all y1, then all y2, then all x1, then all x2; Nsim rows
\end{lstlisting}

Rather than try to write code to do an ancova, it is natural to use the R function lm to do this.  The ``trick'' below is to extract the P-value from the summary command.  By defining a ``wrapper'' function which takes a single line as an argument, it will subsequently be possible to use the ``apply'' function to extract the P-values using a one-line R command.

\begin{lstlisting}
ancova.pval<-function(onerow){ # extract the ancova p-value for diff in means
  J<-length(onerow)/4
  n<-2*J
  y<-onerow[1:n]                         # get the y-data from the row
  x<-onerow[(n+1):(2*n)]                 # get the x-data from the row
  grps<-factor(c(rep(1,J),rep(2,J)))     # define the groups
  sm<-summary(lm(y~x+grps))              # fit the ancova model
  pval<-sm$coefficients[3,4] # extract exactly the right thing from the summary command-the P-value for H0:mu1=mu2
  return(pval)
}
\end{lstlisting}
Everything has now been defined so it is possible to compute all the P-values in a single command line:
\begin{lstlisting}
pvec<-apply(fulldat,1,ancova.pval)
i2<-ifelse(pvec<alpha,1,0)              # for ea test, compute 1=reject, 0=do not reject
ancovapow<-sum(i2)/Nsim                  # the simulated power
cat("The simulated ancova power is ",ancovapow,"\n")
\end{lstlisting}

When run, this script returns:
\begin{lstlisting}
The simulated power is  0.803025 
The exact power:

     Two-sample t test power calculation 

              n = 10
          delta = 10
             sd = 7.5
      sig.level = 0.05
          power = 0.8049123
    alternative = two.sided

 NOTE: n is number in *each* group 

The simulated ancova power is  0.775175 
\end{lstlisting}

It is seen that when the $x$-values are not included in any way (in particular, $\beta=0$), the power is 80.5\%.  However, this is not the correct model in the present situation.  Using the above value of $\beta$ and taking this into account, the power is actually a bit lower or 77.5\%.

\end{xmpl}


{\bf Copyright}
2021, Gunnar Stefansson (editor) with contributions from very many students

This work is licensed under the Creative Commons
Attribution-ShareAlike License. To view a copy of this license, visit
http://creativecommons.org/licenses/by-sa/1.0/ or send a letter to
Creative Commons, 559 Nathan Abbott Way, Stanford, California 94305,
USA.
\clearpage
%% Lecture https://tutor-web.net/math/math612.0/lecture410
\section{Vectors and Matrix Operations}
%% Slide https://tutor-web.net/math/math612.0/lecture410/slide10
\subsection{Numbers, vectors, matrices}
\begin{fbox}
\begin{minipage}{0.97\textwidth}
Recall that the set of real numbers is $\mathbb{R}$ and that a vector , $v \in \mathbb{R}^n$ is just an n-tuple of numbers.\\

Similarly, an $n x m$ matrix is just a table of numbers, with n rows and m columns and we can write 

$$A_{mn} \in \mathbb{R}^{mn}$$

Note that a vector is normally considered equivalent to a  $n\times1$ matrix i.e. we view these as column vectors.

\end{minipage}
\end{fbox}
\subsubsection{Examples}
\begin{xmpl}
In R, a vector can be generated with:

\begin{lstlisting}
X<- 3:6
X
[1] 3 4 5 6
\end{lstlisting}

A matrix can be generated in R as follows,
\begin{lstlisting}
matrix(X)
   [,1]
[1,]    3
[2,]    4
[3,]    5
[4,]    6
\end{lstlisting}

\begin{notes}
We note that R distinguishes between vector and matrices.
\end{notes}
\end{xmpl}

%% Slide https://tutor-web.net/math/math612.0/lecture410/slide20
\subsection{Elementary Operations}
\begin{fbox}
\begin{minipage}{0.97\textwidth}
We can define multiplication of a real number $k$ and a vector $v=(v_1,\ldots,v_n)$ by $k\cdot v=(kv_1,\ldots,kv_n)$. The sum of two vectors in $\mathbb{R}^n$, $v=(v_1,\ldots,v_n)$ and $u=(u_1,\ldots,u_n)$ as the vector $v+u=(v_1+u_1,\ldots,v_n+u_n)$. We can define multiplication of a number and a matrix and the sum of two matrices (of the same sizes) similarly.
\end{minipage}
\end{fbox}
\subsubsection{Examples}
\begin{xmpl}
\begin{lstlisting}
A <- matrix(c(1,2,3,4), nr=2, nc=2)
A
     [,1] [,2]
[1,]    1    3
[2,]    2    4
\end{lstlisting}

\begin{lstlisting}
B <- matrix(c(1,0,2,1), nr=2, nc=2)
B
     [,1] [,2]
[1,]    1    2
[2,]    0    1
\end{lstlisting}

\begin{lstlisting}
A+B
  [,1] [,2]
[1,]    2    5
[2,]    2    5
\end{lstlisting}
\end{xmpl}

%% Slide https://tutor-web.net/math/math612.0/lecture410/slide30
\subsection{The tranpose of a matrix}
\begin{fbox}
\begin{minipage}{0.97\textwidth}
In R, matrices may be constructed using the "matrix" function and the transpose of $A$, $A^\prime$, may be obtained in R by using the "t" function:


\texttt{A<-matrix(1:6, nrow=3)}

$\texttt{t(A)}$


\end{minipage}
\end{fbox}
\subsubsection{Details}
If $A$ is an $n \times m$ matrix with element $a_{ij}$ in row $i$ and column $j$, then $A^\prime$ or $A^T$ is the $m\times n$ matrix with element $a_{ij}$ in row $j$ and column $i$.
\subsubsection{Examples}
\begin{xmpl}

Consider a vector in R
\begin{lstlisting}
x<-1:4
x
[1] 1 2 3 4
t(x)
     [,1] [,2] [,3] [,4]
[1,]    1    2    3    4
matrix(x)
     [,1]
[1,]    1
[2,]    2
[3,]    3
[4,]    4
t(matrix(x))
     [,1] [,2] [,3] [,4]
[1,]    1    2    3    4
 
\end{lstlisting}
\begin{notes}
Note that the first solution gives a $1 \times n$ matrix and the second solution gives a $n \times 1$ matrix.
\end{notes}
\end{xmpl}

%% Slide https://tutor-web.net/math/math612.0/lecture410/slide40
\subsection{Matrix multiplication}
\begin{fbox}
\begin{minipage}{0.58\textwidth}
Matrices A and B can be multiplied together if A is an $n \times p$ matrix and B is an $p\times m$ matrix. The general element $c_ij$ of $n\times m$; $C=AB$ is found by pairing the $i^th$ row of C with the $j^th$ column of B, and computing the sum of products of the paired terms. 
\end{minipage}
\hspace{0.5mm}
\begin{minipage}{0.38\textwidth}
\begin{picture}
39
\end{picture}


\end{minipage}
\end{fbox}
\subsubsection{Details}
Matrices A and B can be multiplied together if A is a $n\times p$ matrix and B is a $p\times m$ matrix. Given the general element $c_{ij}$ of $n x m$ matrix, $C=AB$ is found by pairing the $i^th$ row of C with the $j^th$ column of B, and computing the sum of products of the paired terms. 
\subsubsection{Examples}
\begin{xmpl}
Matrices in R
\begin{lstlisting}
A<-matrix(c(1,3,5,2,4,6),3,2)
A
     [,1] [,2]
[1,]    1    2
[2,]    3    4
[3,]    5    6
B<-matrix(1,1,2,3)2,2)

B<-matrix(c(1,1,2,3),2,2)
 B
     [,1] [,2]
[1,]    1    2
[2,]    1    3
A%*%B
     [,1] [,2]
[1,]    3    8
[2,]    7   18
[3,]   11   28
\end{lstlisting}
\end{xmpl}

%% Slide https://tutor-web.net/math/math612.0/lecture410/slide60
\subsection{More on matrix multiplication}
\begin{fbox}
\begin{minipage}{0.97\textwidth}
Let $A$, $B$, and $C$ be $m\times n$, $n\times l$, and $l\times p$ matrices, respectively. Then we have
$$
(AB)C=A(BC).
$$
In general, matrix multiplication is not commutative, that is $AB\neq BA$.\\
We also have
$$
(AB)'=B'A'.
$$
In particular, $(Av)'(Av)=v'A'Av$, when $v$ is a $n\times1$ column vector.\\\\
More obvious are the rules
\begin{enumerate}
\item $A+(B+C)=(A+B)+C$
\item k(A+B)=kA+kB
\item A(B+C)=AB+AC,
\end{enumerate}
where $k\in\mathbb{R}$ and when the dimensions of the matrices fit.
\end{minipage}
\end{fbox}

%% Slide https://tutor-web.net/math/math612.0/lecture410/slide70
\subsection{Linear equations}
\subsubsection{Details}
{\bf Detail:}\\
General linear equations can be written in the form $Ax=b$.
\subsubsection{Examples}
\begin{xmpl}
The set of equations\\
\\
$2x+3y=4$ \\
$3x+y=2$ \\

can be written in matrix formulation as\\

$$\begin{bmatrix}
2 & 3  \\
3 & 1 
\end{bmatrix}
\begin{bmatrix}
x \\
y
\end{bmatrix} =
\begin{bmatrix}
4 \\
2
\end{bmatrix}
$$

i.e. $A\underline{x} = \underline{b}$ for an appropriate choice of of $A, \underline{x}$ and $\underline{b}$
\end{xmpl}

%% Slide https://tutor-web.net/math/math612.0/lecture410/slide90
\subsection{The unit matrix}
\begin{fbox}
\begin{minipage}{0.97\textwidth}
The $n$ x $n$ matrix \\

\begin{equation*}
\mathbf{I}=
\left[
\begin{array}{cccc}
  1 & 0 & \ldots & 0 \\
  0 & 1 & 0  & \vdots \\
  \vdots & 0  & \dots & 0 \\
  0 & \ldots & 0 & 1
\end{array} \right]
\end{equation*}

is the identity matrix. This is because if a matrix $\mathbf{A}$ is $n$ x $n$ then $\mathbf{A} \mathbf{I} = \mathbf{A}$ and $\mathbf{I} \mathbf{A}  = 
\mathbf{A}$
\end{minipage}
\end{fbox}

%% Slide https://tutor-web.net/math/math612.0/lecture410/slide95
\subsection{The inverse of a matrix}
\begin{fbox}
\begin{minipage}{0.97\textwidth}
If $A$ is an $n \times n$ matrix and $B$ is a matrix such that

$$ BA = AB = I$$

Then $B$ is said to be the inverse of $A$, written 

$$ B = A ^{-1} $$

Note that if $A$ is an $n \times n$ matrix for which an inverse exists, then the equation $Ax = b$ can be solved and the solution is $x = A^{-1} b$.

\end{minipage}
\end{fbox}
\subsubsection{Examples}
\begin{xmpl}

If matrix $A$ is:

$\begin{bmatrix}
2 & 3  \\
3 & 1 
\end{bmatrix}$

then $A ^{-1}$ is:

$\begin{bmatrix}
\frac{-1}{4} & \frac{3}{4}  \\
\frac{3}{4} & \frac{1}{2} 
\end{bmatrix}$
\end{xmpl}

{\bf Copyright}
2021, Gunnar Stefansson (editor) with contributions from very many students

This work is licensed under the Creative Commons
Attribution-ShareAlike License. To view a copy of this license, visit
http://creativecommons.org/licenses/by-sa/1.0/ or send a letter to
Creative Commons, 559 Nathan Abbott Way, Stanford, California 94305,
USA.
\clearpage
%% Lecture https://tutor-web.net/math/math612.0/lecture420
\section{Some notes on matrices and linear operators}
%% Slide https://tutor-web.net/math/math612.0/lecture420/slide10
\subsection{The matrix as a linear operator}
\begin{fbox}
\begin{minipage}{0.97\textwidth}
Let $A$ be an $m\times  n$ matrix, the function
 
 $$T_A:\mathbb{R}^n\to\mathbb{R}^m, T_A(\underline{x}) = A\underline{x}, $$
is linear, that is
 
$$ T_A (a\underline{x} + b\underline{y}) = aT_A(\underline{x}) + bT_A(\underline{y}) $$
 
 if $ \underline{x}, \underline{y} \in \mathbb{R}^n$ and $a, b \in \mathbb{R}$.
\end{minipage}
\end{fbox}
\subsubsection{Examples}
\begin{xmpl}

 If A=
$\begin{bmatrix}
1 & 2 
\end{bmatrix}$ then $T_A(\underline{x}) = x + 2y$ where $ \underline{x} = {x \choose y}\in \mathbb{R}^2$
\end{xmpl}


\begin{xmpl}
If A=
$\begin{bmatrix}
0 & 1  \\
1 & 0 
\end{bmatrix}$ then $ T_A{x \choose y}$ =  
$\begin{bmatrix}
y  \\
x 
\end{bmatrix}$
\end{xmpl}
\begin{xmpl}

 If A=
$\begin{bmatrix}
0 & 2 & 3\\
1 & 0 & 1
\end{bmatrix}$ then $T_A$
$\left(
\begin{array}{ccc}
  x \\
  y\\
  z\\
 \end{array} \right)$ = 
$\begin{bmatrix}
2y + 3z\\
x + z
\end{bmatrix}$
\end{xmpl}
\begin{xmpl}

If $T{x \choose y }$ = 
$\left(
 \begin{array}{cc}
   x+y  \\
  2x-3y\\
 \end{array} \right)$ then $T (\underline{x}) = A \underline{x}$ if  we set A = 
$\begin{bmatrix}
1 & 1 \\
2 & -3
\end{bmatrix}$
 \end{xmpl}

%% Slide https://tutor-web.net/math/math612.0/lecture420/slide20
\subsection{Inner products and norms}
\begin{fbox}
\begin{minipage}{0.97\textwidth}
Assuming $x$ and $y$ are vectors, then we define their inner product by 

$$x \cdot y = x_1y_1 + x_2y_2 + \cdots + x_ny_n$$

where $x =\begin{pmatrix}
x_1\\
\vdots\\
x_n
\end{pmatrix}$ and $y= \begin{pmatrix}
y_1\\
\vdots\\
y_n
\end{pmatrix}$ 


\end{minipage}
\end{fbox}
\subsubsection{Details}
If $x$, $y$ $\in \mathbb{R}^n$ are arbitrary (column) vectors, then we define their inner product by
$$x \cdot y = x_1y_1 + x_2y_2 + \cdots + x_ny_n$$

where $x= \begin{pmatrix}
x_1\\
\vdots\\
x_n
\end{pmatrix}$ and $y =\begin{pmatrix}
y_1\\
\vdots\\
y_n
\end{pmatrix}$.
\begin{notes}
Note that we can also view $x$ and $y$ as $n \times 1$ matrices and we see that $x \cdot y = x^\prime y$.  
\end{notes}
\begin{defn}
The normal length of a vector is defined by $\left \| x \right \|^2 = x \cdot x$.  It may also be expressed as $\left \| x \right \| = \sqrt{x_1^2 + x_2^2 + \cdots + x_n^2}$. 
\end{defn}
It is easy to see that for vectors $a, b$ and $c$ we have $(a+b)\cdot c=a\cdot c+ b\cdot c$ and $a\cdot b=b\cdot a$.
\subsubsection{Examples}
Two vectors $x$ and $y$ are said to be orthogonal if $x \cdot y = 0$

\begin{xmpl}

If $ x= \begin{pmatrix}
3\\
4
\end{pmatrix}$ and $y= \begin{pmatrix}
2\\
1
\end{pmatrix}$, then $$x \cdot y = 3 \cdot 2 + 4 \cdot 1 = 10,$$

and

$$\left \| x \right \|^2 = 3^2 + 4^2 = 25,$$

so

$$\left\| x \right \| = 5$$

\end{xmpl}

%% Slide https://tutor-web.net/math/math612.0/lecture420/slide30
\subsection{Orthogonal vectors}
\begin{fbox}
\begin{minipage}{0.97\textwidth}
Two vectors $x$ and $y$ are said to be orthogonal if $x\cdot y=0$ denoted $x \perp y$
\end{minipage}
\end{fbox}
\subsubsection{Details}
\begin{defn}
Two vectors $x$ and $y$ are said to be {\bf orthogonal} if $x\cdot y=0$ denoted $x \perp y$
\end{defn}

If $a,b \in \mathbb{R}^n$ then 
$$\left\|a+b\right\|^2=a\cdot a+2a\cdot b+b\cdot b$$
so 
$$\left\|a+b\right\|^2=\left\|a\right\|^2+\left\|b\right\|^2 + 2\underline{a}\underline{b}.$$
\begin{notes}
Note that if $a \perp b$ then $\left\|a+b\right\|^2=\left\|a\right\|^2+ \left\|b\right\|^2$, which is Pythagoras' theorem in $n$ dimensions.
\end{notes}

%% Slide https://tutor-web.net/math/math612.0/lecture420/slide50
\subsection{Linear combinations of i.i.d. random variables}
\begin{fbox}
\begin{minipage}{0.97\textwidth}
Suppose $X_1,....,X_n$ are i.i.d. random variables and have mean $\mu_1, ...., \mu_n$ and variance $\sigma^2$ 
then the expected value of $Y$ of the linear combination is 
$$Y=\sum a_i X_i$$
and if $a_1,....,a_n$ are real constants then the mean is:

$$\mu_Y = \sum a_i \mu_i$$

and the variance is:

$$\sigma^2 = \sum a^2_i \sigma^2_i$$
\end{minipage}
\end{fbox}
\subsubsection{Examples}
\begin{xmpl} 
Consider two i.i.d. random variables, 
$Y_1,Y_2$ 
and a specific  
linear combination of the two, 
$W=Y_1+3Y_2$.\\

We first obtain 
$$E[W]=E[Y_1+3Y_2]=E[Y_1]+3E[Y_2]=2+3\cdot 2=2+6=8.$$

Similarly, we 
can first use independence 
to obtain
$$V[W]=V[Y_1+3Y_2]=V[Y_1]+V[3Y_2]$$

and then (recall that $V[aY]=a^2V[Y]$)
$$V[Y_1]+V[3Y_2]=V[Y_1]+3^2V[Y_2]=1^2+3^2= 1(4) + 9(4)= 40$$

Normally, 
we just write this up in 
a simple sequence
$$V[W]=V[Y_1+3Y_2]=V[Y_1]+3^2V[Y_2]=1^2+3^2= 1(4) + 9(4)= 40$$
\end{xmpl}

%% Slide https://tutor-web.net/math/math612.0/lecture420/slide60
\subsection{Covariance between linear combinations of i.i.d random variables}
\begin{fbox}
\begin{minipage}{0.97\textwidth}
Suppose $Y_1,\ldots,Y_n$ are i.i.d., each with mean $\mu$ and variance $\sigma^2$ and $a,b\in \mathbb{R}^n$.
Writing 
$
 Y= \left(
 \begin{array}{ccc}
   Y_1  \\
   \vdots\\
   Y_n 
 \end{array} \right)
$, consider the linear combination $a'Y$ and $b'Y$.
\end{minipage}
\end{fbox}
\subsubsection{Details}
The covarience between random variables $U$ and $W$ is defined by\\
$$Cov(U,W)= E[(U-\mu_u)(W-\mu_w)]$$
where 
$$\mu_u=E[U],\mu_w=E[W]$$
Now, let $U=a'Y=\sum Y_ia_i$ and $W=b'Y=\sum Y_ib_i$, where $Y_1,\ldots,Y_n$ are i.i.d. with mean $\mu$ and variance $\sigma^2$, then we get

$$Cov(U,W)= E[(\underline{a}'Y-\Sigma a_\mu)(\underline{b}'Y-\Sigma b\mu)]$$

$$= E[(\Sigma a_iY_i -\Sigma a_i\mu)(\Sigma b_jY_j -\Sigma b_j\mu )]$$

        

and after some tedious (but basic) calculations we obtain
$$Cov(U,W)=\sigma^2a\cdot b$$
\subsubsection{Examples}
\begin{xmpl}
If $Y_1$ and $Y_2$ are i.i.d., then

$$Cov(Y_1+Y_2, Y_1-Y_2)=Cov((1,1) \left( \begin{array}{ccc}Y_1  \\Y_2 \end{array} \right),  (1,-1) \left( \begin{array}{ccc}Y_1  \\Y_2 \end{array} \right) )$$

$$=(1,1) \left( \begin{array}{ccc} 1  \\-1 \end{array} \right)\sigma^2 $$

$$=0 $$

and in general, $Cov(\underline{a}'\underline{Y}, \underline{b}'\underline{Y})=0$ if $\underline{a}\bot \underline{b}$ and $Y_1,\ldots,Y_n$ are independent.
\end{xmpl}


%% Slide https://tutor-web.net/math/math612.0/lecture420/slide70
\subsection{Random vectors}
\begin{fbox}
\begin{minipage}{0.97\textwidth}
$Y= (Y_1, \ldots , Y_n)$ is a random vector if $Y_1, \ldots , Y_n$ are random variables.
\end{minipage}
\end{fbox}
\subsubsection{Details}
\begin{defn}
If $EY_i = \mu_i$ then we typically write

$E(Y)=\left(
\begin{array}{ccc}
  \mu_1 \\
  \vdots \\
  \mu_n
\end{array} \right)=\mu$

If $Cov(Y_i, Y_j) = \sigma{ij}$ and $V[Y_i]=\sigma_{ii} = \sigma_i^2$, then we define the matrix $$\boldsymbol{\Sigma} = (\sigma_{ij})$$
containing the variances and covariances. We call this matrix the {\bf covariance matrix} of $Y$, typically denoted $V[Y] = \boldsymbol{\Sigma}$ or $Cov[Y] = \boldsymbol{\Sigma}$.
\end{defn}
\subsubsection{Examples}
\begin{xmpl}
If $Y_i, \ldots , Y_n$ are i.i.d., $EY_i = \mu$, $VY_i = \sigma^2$, $a,b\in\mathbb{R}^n$ and $U=a'Y$, $W=b'Y$,

and T=
$\begin{bmatrix}
U \\
W
\end{bmatrix}$

then

$ET=
\begin{bmatrix}
\Sigma a_i  \mu \\
\Sigma b_i  \mu
\end{bmatrix}$


$VT=
\boldsymbol{\Sigma}$ 
= $\sigma^2
\begin{bmatrix}
\Sigma a_i^2 & \Sigma a_i b_i \\
\Sigma a_ib_i & \Sigma b_i^2
\end{bmatrix}$
\end{xmpl}

\begin{xmpl}

If $\underline{Y}$ is a random vector with mean $\boldsymbol{\mu}$ and variance-covariance matrix 
$\boldsymbol{\Sigma}$, then

$$
E[a'Y] = a'\mu
$$


and 

$$
V[a'Y] = a' \boldsymbol{\Sigma} a.
\end{xmpl}

%% Slide https://tutor-web.net/math/math612.0/lecture420/slide80
\subsection{Transforming random vectors}
\begin{fbox}
\begin{minipage}{0.97\textwidth}
Suppose 

\[\mathbf{Y}=
\left(
\begin{array}{c}
Y_1\\
\vdots \\
Y_n
\end{array} \right)\]

is a random vector with $E \mathbf{Y} = \mu$ and $V \mathbf{Y} = \boldsymbol{\Sigma} $ where the variance-covariance matrix 
$$ \boldsymbol{\Sigma} = \sigma^2 \mathbf{I} $$
\end{minipage}
\end{fbox}
\subsubsection{Details}
Note that if $Y_1, \ldots, Y_n$ are independent with common variance $\sigma^2$ then

\[
\boldsymbol{\Sigma}=
\left[
\begin{array}{ccccc}
\sigma_{1}^{2} & \sigma_{12} & \sigma_{13} & \ldots & \sigma_{1n} \\
\sigma_{21} & \sigma_2^{2} & \sigma_{23} & \ldots & \sigma_{2n} \\
\sigma_{31} &\sigma_{32}  &\sigma_3^{2}  & \ldots & \sigma_{3n}\\
\vdots & \vdots & \vdots & \ddots & \vdots \\
\sigma_{n1} & \sigma_{n2} & \sigma_{n3} & \ldots & \sigma_n^{2}\\ 
\end{array} \right]
\]




\[
 =
\left[
\begin{array}{ccccc}
\sigma_{1}^{2} & 0 & \ldots & \ldots & 0 \\
 0 & \sigma_2^{2} & \ddots & 0  & \vdots \\
 \vdots & \ddots  &\sigma_3^{2}  & \ddots & \vdots \\
\vdots & 0 & \ddots & \ddots & 0 \\
0 & \ldots & \ldots & 0 & \sigma_n^{2}\\ 
\end{array} \right]\]



\[
 = 
\sigma^2
\left[
\begin{array}{ccccc}
1 & 0 & \ldots & \ldots & 0 \\
 0 & 1 & \ddots & 0  & \vdots \\
 \vdots & \ddots  & 1  & \ddots & \vdots \\
\vdots & 0 & \ddots & \ddots & 0 \\
0 & \ldots & \ldots & 0 & 1 \\ 
\end{array} \right]
= \sigma^2 \mathbf{I}
\]



If $A$ is an $m$ x $n$ matrix, then \\

$$ E[A\mathbf{Y}] = A \mathbf{\mu} $$

and
$$ V[A\mathbf{Y}] = A \boldsymbol{\Sigma} A' $$

{\bf Copyright}
2021, Gunnar Stefansson (editor) with contributions from very many students

This work is licensed under the Creative Commons
Attribution-ShareAlike License. To view a copy of this license, visit
http://creativecommons.org/licenses/by-sa/1.0/ or send a letter to
Creative Commons, 559 Nathan Abbott Way, Stanford, California 94305,
USA.
\clearpage
%% Lecture https://tutor-web.net/math/math612.0/lecture430
\section{Ranks and determinants}
%% Slide https://tutor-web.net/math/math612.0/lecture430/slide10
\subsection{The rank of a matrix}
\begin{fbox}
\begin{minipage}{0.97\textwidth}
The rank of an $n x p$ matrix, $A$, is the largest number of columns of  $A$, which are not linearly dependent (i.e. the number of linearly independent columns).
\end{minipage}
\end{fbox}
\subsubsection{Details}
Vectors $a_1, a_2, \ldots, a_n$ are said to be linearly dependent if the constant $k_1 , \ldots, k_n$ exists and are not all zero, such that 
$$ k_1\mathbf{a}_1 + k_2 \mathbf{a}_2 + \ldots + k_n\mathbf{a}_n = 0$$

Note that if such constants exist, then we can write one of the $a$'s as a linear combination of the rest, e.g. if $k_1 \neq 0$ then

$$ a_1=\mathbf{c_1} =  -\frac{k_2}{k_1} a_2 - \ldots - \frac{k_2}{k_1} a_n $$

It can be shown that the rank of  $A$ is the same as the rank of  $A'$ i.e. the maximum number of linearly independent rows of  $A$. 
\begin{notes}
Note that if rank $(A)= p$, then the columns are linearly independent.
\end{notes}

\subsubsection{Examples}
\begin{xmpl}

If \[A= 
  \left[ 
   \begin{array}{cc}
     1 & 0 \\
     0 & 1 \\
   \end{array}
  \right]\]

the rank of  $A$ = 2, since

 \[  k_1
  \left(
   \begin{array}{cc}
     1 \\
     0 \\
   \end{array}
  \right) +  k_2
  \left(
   \begin{array}{cc}
     0 \\
     1 \\
   \end{array}
  \right) =  \left(
    \begin{array}{cc}
      0 \\
      0 \\
    \end{array}
   \right) \]

if and only if  

 \[ \left( 
  \begin{array}{cc}
     k_1 \\
     k_2 \\
   \end{array}
  \right) =   \left(
    \begin{array}{cc}
      0 \\
      0 \\
    \end{array}
   \right) \]

so the columns are linearly independent.
\end{xmpl}

\begin{xmpl}


 If \[A = 
  \left[ 
   \begin{array}{ccc}
     1 & 0 & 1 \\
     0 & 1 & 1\\
     0 & 0 & 0 \\
   \end{array}
  \right]\]

the rank of  $A$ = 2.
\end{xmpl}

\begin{xmpl}

 If \[ A = 
  \left[ 
   \begin{array}{ccc}
     1 & 1 & 1 \\
     0 & 1 & 0 \\
     0 & 1 & 0 \\
   \end{array}
  \right]\]

the rank of  $A$ = 2,
since  
\[  1
  \left(
   \begin{array}{ccc}
     1 \\
     0 \\
     0 \\
   \end{array}
  \right) +  0
  \left(
   \begin{array}{ccc}
     0 \\
     1 \\
     1 \\
   \end{array}
  \right) + (-1)
  \left(
    \begin{array}{ccc}
      1 \\
      0 \\
      0 \\
    \end{array}
   \right) = 0 \] 
(and hence the rank can not be more than 2) 
but  
\[  k_1
  \left(
   \begin{array}{ccc}
     1 \\
     0 \\
     0 \\
   \end{array}
  \right) +  k_2
  \left(
   \begin{array}{ccc}
     0 \\
     1 \\
     1 \\
   \end{array}
  \right)\] \\ if and only if $k_1=k_2=0$ (and hence the rank must be at least 2).
\end{xmpl}

%% Slide https://tutor-web.net/math/math612.0/lecture430/slide20
\subsection{The determinant}
\begin{fbox}
\begin{minipage}{0.97\textwidth}
Recall that for a 2x2 matrix, 

$A=
\begin{bmatrix}
a & b  \\
c & d 
\end{bmatrix}$

the inverse of $A$ is 

$A^{-1}=
\frac{1}{ad-bc}
\begin{bmatrix}
2 & 3  \\
3 & 1 
\end{bmatrix}$
\end{minipage}
\end{fbox}
\subsubsection{Details}
\begin{defn}
The number $ad-bc$ is called the {\bf determinant}of the 2x2 matrix $A$.
\end{defn}
\begin{defn}
Now suppose $A$ is an $n x n$ matrix. An {\bf elementary product} from the matrix is a product of $n$ terms based on taking  exactly one term from each column of row $x$. Each such term can be written in the form 
$a_{1j_1} \cdot a_{2j_2} \cdot a_{3j_3} \cdot \ldots \cdot a_{nj_n}$
where $j_1, \ldots , j_n$ is a permutation of the integers $1,2, \ldots , n$. Each permutation $\sigma$ of the integers $1,2,\ldots,n$ can be performed by repeatedly interchanging two numbers. 
\end{defn}

\begin{defn}
A {\bf signed elementary product} is an elementary product with a positive sign if the number of interchanges in the permutation is even but negative otherwise.
\end{defn}

The determinant of A, det(A) or $\vert A \vert$ is the sum of all signed elementary products.
\subsubsection{Examples}
\begin{xmpl}

$A=
\begin{bmatrix}
a_{11} & a_{12}  \\
a_{21} & a_{22} 
\end{bmatrix}$

then

$\vert A \vert = a_{1\underline{1}} a_{2\underline{2}} - a_{1\underline{2}}a_{2\underline{1}}$.
\end{xmpl}
\begin{xmpl}


$A=
\begin{bmatrix}
a_{11} & a_{12} & a_{13}  \\
a_{21} & a_{22} & a_{23} \\
a_{31} & a_{32} & a_{33}
\end{bmatrix}$

                                                
$\vert A \vert$ 

=  $a_{11} a_{22} a_{33}$   This is the identity permutation and has positive sign

  $-a_{11} a_{23} a_{32}$   This is the permutation that only interchanges $2$ and $3$

  $-a_{12} a_{21} a_{33}$   Only one interchange

  $+a_{12} a_{23} a_{31}$   Two interchanges

  $+a_{13} a_{21} a_{32}$   Two interchanges

  $-a_{13} a_{22} a_{31}$   Three interchanges
\end{xmpl}
\begin{xmpl}


$A=
\begin{bmatrix}
1 & 1  \\
1 & 0 
\end{bmatrix}$

$\vert A \vert = -1$

\end{xmpl}
\begin{xmpl}

$A=
\begin{bmatrix}
1 & 0 & 0  \\
0 & 2 & 0  \\
0 & 0 & 3 
\end{bmatrix}$

$\vert A \vert = 1 \cdot 2 \cdot 3 = 6$
\end{xmpl}
\begin{xmpl}


$A=
\begin{bmatrix}
1 & 0 & 0  \\
0 & 2 & 0  \\
0 & 3 & 0 
\end{bmatrix}$

$\vert A \vert = 0$


\end{xmpl}
\begin{xmpl}

$A=
\begin{bmatrix}
1 & 0 & 0  \\
0 & 0 & 2  \\
0 & 3 & 0 
\end{bmatrix}$

$\vert A \vert = -6$
\end{xmpl}
\begin{xmpl}


$A=
\begin{bmatrix}
2 & 1  \\
2 & 1 
\end{bmatrix}$

$\vert A \vert = 0$
\end{xmpl}
\begin{xmpl}


$A=
\begin{bmatrix}
1 & 0 & 1  \\
0 & 1 & 1  \\
1 & 1 & 2 
\end{bmatrix}$

$\vert A \vert = 0$
\end{xmpl}

%% Slide https://tutor-web.net/math/math612.0/lecture430/slide30
\subsection{Ranks, inverses and determinants}
\begin{fbox}
\begin{minipage}{0.97\textwidth}
The following statements are true for an $n\times n$ matrix $A$:
\begin{itemize}
\item $rank (A)= n$
\item $det(A)\neq 0$
\item $A$ has an inverse
\end{itemize}
\end{minipage}
\end{fbox}
\subsubsection{Details}
Suppose $A$ is an $n\times n$ matrix.  Then the following are truths:
\begin{itemize}
\item $rank (A)= n$
\item $det(A)\neq 0$
\item $A$ has an inverse
\end{itemize}

{\bf Copyright}
2021, Gunnar Stefansson (editor) with contributions from very many students

This work is licensed under the Creative Commons
Attribution-ShareAlike License. To view a copy of this license, visit
http://creativecommons.org/licenses/by-sa/1.0/ or send a letter to
Creative Commons, 559 Nathan Abbott Way, Stanford, California 94305,
USA.
\clearpage
%% Lecture https://tutor-web.net/math/math612.0/lecture440
\section{Multivariate calculus}
%% Slide https://tutor-web.net/math/math612.0/lecture440/slide20
\subsection{Vector functions of several variables}
\begin{fbox}
\begin{minipage}{0.97\textwidth}
A vector-valued function of several variables is a function
$$
f: \mathbb{R}^{m} \rightarrow \mathbb{R}^{n}
$$
i.e. a function of $m$ dimensional vectors, which returns $n$ dimensional vectors.
\end{minipage}
\end{fbox}
\subsubsection{Examples}
\begin{xmpl}
A real valued function of many variables: $f: \mathbb{R}^3\to\mathbb{R}$, $f(x_1,x_2,x_3)=2x_1+3x_2+4x_3$. 
\begin{notes}
Note that $f$ is linear and $f(x)=Ax$ where
$x=\begin{pmatrix} x_1\\x_2\\x_3\end{pmatrix}$ and $A=\begin{bmatrix}2&3&4\end{bmatrix}$.
\end{notes}
\end{xmpl}
\begin{xmpl}
Let 
$$
f: \mathbb{R}^{2} \rightarrow \mathbb{R}^{2}
$$
where:
$$
f(x_1,x_2) = \left(
\begin{array}{c}
x_1+x_2 \\
x_1-x_2
\end{array} \right)
$$
\begin{notes}
Note that $f(x)=Ax$, where $A=\begin{bmatrix}1&1\\1&-1\end{bmatrix}$.
\end{notes}
\end{xmpl}
\begin{xmpl}
Let 
$$
f: \mathbb{R}^{3} \rightarrow \mathbb{R}^{4}
$$
be defined by
$$
f(x)
= \left(
\begin{array}{c}
x_1+x_2 \\
x_1-x_3 \\
y-z \\
x_1+x_2+x_3
\end{array} \right)
$$
\begin{notes}
Note that:
$$
f(x) = Ax
$$
where
$$
A =
\begin{bmatrix}
1 & 1 & 0\\
1 & 0 & -1\\
0 & 1 & -1\\
1 & 1 & 1
\end{bmatrix}
$$
\end{notes}
\end{xmpl}
\begin{xmpl}
These multi-dimensional functions do not have to be linear, for example the function $f:\mathbb{R}^2\to\mathbb{R}^2$
$$
f(x) = \left(
\begin{array}{c}
x_1x_2 \\
x_1^{2}+x_2^{2}
\end{array} \right),
$$ 
is obviously not linear.
\end{xmpl}

%% Slide https://tutor-web.net/math/math612.0/lecture440/slide30
\subsection{The gradient}
\begin{fbox}
\begin{minipage}{0.97\textwidth}
Suppose the real valued function $f:
\mathbb{R}^m \rightarrow \mathbb{R}$ is differentiable in 
each coordinate. Then the gradient of $f$, denoted $\nabla f$ is given by 


$$\nabla f(x)=\begin{pmatrix}\frac{\partial f}{\partial x_1},&\dots &,\frac{\partial f}{\partial x_1}\end{pmatrix}.$$

\end{minipage}
\end{fbox}
\subsubsection{Details}
\begin{defn}
Suppose the real valued function $f:\mathbb{R}^m \rightarrow \mathbb{R}$ is differentiable in 
each coordinate. Then the {\bf gradient} of $f$, denoted $\nabla f$ is given by 

$$\nabla f(x)=\begin{pmatrix}\frac{\partial f}{\partial x_1},&\dots &,\frac{\partial f}{\partial x_1}\end{pmatrix},$$

where each partial derivative $\frac{\partial f}{\partial x_i}$ is computed by differentiating f with respect to that variable, regarding the others as fixed. 
\end{defn}
\subsubsection{Examples}
\begin{xmpl}

$$f(\underline{x})= x^2+y^2+2xy;\ \frac{\partial f}{\partial x}=2x+2y,  \frac{\partial f}{\partial y}=2y+2x, \nabla f =\begin{pmatrix}2x+2y, & 2y+2x\end{pmatrix}$$
\end{xmpl}
\begin{xmpl}


$$f(\underline{x})=x_1-x_2; \nabla f= \begin{pmatrix}1, & -1\end{pmatrix}$$
\end{xmpl}

%% Slide https://tutor-web.net/math/math612.0/lecture440/slide40
\subsection{The Jacobian}
\begin{fbox}
\begin{minipage}{0.97\textwidth}
Now consider a function 
$f:\mathbb{R}^m\to\mathbb{R}^n$. 
Write $f_i$ for the $i^{th}$ coordinate of $f$, so we can write $f(x)=(f_1(x),f_2(x),\ldots,f_n(x))$, 
where $x\in\mathbb{R}^m$. 
If each coordinate function $f_i$ is differentiable in each variable we can form the {\em Jacobian matrix} of $f$: 
$$\begin{pmatrix}\nabla f_1\\ \vdots \\ \nabla f_n\end{pmatrix}.$$

\end{minipage}
\end{fbox}
\subsubsection{Details}
Now consider a function 
$f:\mathbb{R}^m\to\mathbb{R}^n$. 
Write $f_i$ for the $i^{th}$ coordinate of $f$, so we can write $f(x)=(f_1(x),f_2(x),\ldots,f_n(x))$, 
where $x\in\mathbb{R}^m$. 
If each coordinate function $f_i$ is differentiable in each variable we can form the {\em Jacobian matrix} of $f$: 
$$\begin{pmatrix}\nabla f_1\\ \vdots \\ \nabla f_n\end{pmatrix}.$$


In this matrix, the element in the $i^th$ row and $j^th$ column is $\frac{\partial f_i}{\partial x_j}$.
\subsubsection{Examples}
\begin{xmpl}

For the function 

$$f(x,y)=\begin{pmatrix}
  x^2 +y \\
  x y  \\
  x 
\end{pmatrix}=
\begin{pmatrix}
  f_1(x,y) \\
  f_2(x,y) \\
  f_3(x,y) 
\end{pmatrix},$$

the Jacobian matrix of $f$ is the matrix

$$J= \begin{bmatrix}
      \nabla f_1 \\
      \nabla f_2 \\
      \nabla f_3 
      \end{bmatrix}=
\begin{bmatrix}
      2x & 2y \\
      y & x \\
      1 & 0 
      \end{bmatrix}.
$$
\end{xmpl}

%% Slide https://tutor-web.net/math/math612.0/lecture440/slide50
\subsection{Univariate integration by substitution}
\begin{fbox}
\begin{minipage}{0.97\textwidth}
If $f$ is a continuous function and $g$ is strictly increasing and differentiable then,


$$ \int_{g(a)}^{g(b)} f(x)dx =  \int_a^b f(g(t))g^\prime (t)dt$$


\end{minipage}
\end{fbox}
\subsubsection{Details}
If $f$ is a continuous function and $g$ is strictly increasing and differentiable then,


$$ \int_{g(a)}^{g(b)} f(x)dx =  \int_a^b f(g(t))g^\prime (t)dt$$


It follows that if $X$ is a continuous random variable with density $f$ and $ Y = h(X)$ is a function of $X$ that has the inverse $g=h^{-1}$, so  $ X = g(Y)$ , then the density of $Y$ is given by,

$$f_Y(y)   = f (g(y)) g^\prime (y)$$

This is a consequence of
$$ P [Y \leq b] = P [g(Y) \leq g(b)] = P [X \leq g(b)] = \int_{- \infty} ^{g(b)}f(x)dx = \int_{- \infty} ^b f (g(y))g^\prime (y)dy.$$

%% Slide https://tutor-web.net/math/math612.0/lecture440/slide60
\subsection{Multivariate integration by substitution}
\begin{fbox}
\begin{minipage}{0.97\textwidth}
Suppose $f$ is a continuous function 
$f: \mathbb{R}^n \rightarrow \mathbb{R}$ and $g: \mathbb{R}^n \rightarrow \mathbb{R}^n$ is a one-to-one function with continuous partial derivatives. Then if $ U \subseteq \mathbb{R}^n$ is a subset,

$$\int_{g(U)} f(\mathbf {x})d\mathbf {x}  = \int_{U}({g}(\mathbf {y}))|J|d\mathbf {y}$$

where $J$ is the Jacobian matrix and |J| is the absolute value of it's determinant.


$$J=
\left|\begin{bmatrix}
\frac{\partial g_1}{\partial y_1} & \frac{\partial g_1}{\partial y_2} & \cdots &\frac{\partial g_1}{\partial y_n} \\
\vdots & \vdots & \cdots & \vdots \\
\frac{\partial g_n}{\partial y_1} & \frac{\partial g_n}{\partial y_2} & \cdots & \frac{\partial g_n}{\partial y_n} 
\end{bmatrix}\right| = 
\left|\begin{bmatrix}
\nabla g_1 \\
 \vdots \\
\nabla g_n 
\end{bmatrix}\right|$$


\end{minipage}
\end{fbox}
\subsubsection{Details}
Suppose $f$ is a continuous function 
$f: \mathbb{R}^n \rightarrow \mathbb{R}$ and $g: \mathbb{R}^n \rightarrow \mathbb{R}^n$ is a one-to-one function with continuous partial derivatives. Then if $ U \subseteq \mathbb{R}^n$ is a subset,

$$\int_{g(U)} f(\mathbf {x})d\mathbf {x}  = \int_{U}({g}(\mathbf {y}))|J|d\mathbf {y}$$

where $J$ is the Jacobian determinant and |J| is its absolute value.


$$J= 
\left|\begin{bmatrix}
\frac{\partial g_1}{\partial y_1} & \frac{\partial g_1}{\partial y_2} & \cdots &\frac{\partial g_1}{\partial y_n} \\
\vdots & \vdots & \cdots & \vdots \\
\frac{\partial g_n}{\partial y_1} & \frac{\partial g_n}{\partial y_2} & \cdots & \frac{\partial g_n}{\partial y_n} 
\end{bmatrix}\right| = 
\left|\begin{bmatrix}
\nabla g_1 \\
 \vdots \\
\nabla g_n 
\end{bmatrix}\right|$$

Similar calculations as in 28.4 give us that if $X$ is a continuous multivariate random variable, $X = (X_1, \ldots, X_n)^\prime$ with density $f$ and $\mathbf{Y}  = \mathbf{h} (\mathbf{X})$, where $\mathbf{h}$ is 1-1 with inverse $ \mathbf g= \mathbf{h}^{-1}$. So, $\mathbf{X} = g(\mathbf{Y})$, then the density of $\mathbf{Y}$ is given by;

$$f_Y(\mathbf y)   = f (g(\mathbf y)) |J|$$


\subsubsection{Examples}
\begin{xmpl}

If $\mathbf{Y} = A \mathbf X$ where $A$ is an $n \times n$ matrix with $det(A)\neq0$ and $X = (X_1, \ldots, X_n)^\prime $ are i.i.d. random variables, then we have the following results:\\\\

The joint density of $ X_1 \cdots X_n$ is the product of the individual (marginal) densities,

$$f_X(\mathbf x)= f(x_1) f(x_2) \cdots f(x_n)$$

The matrix of partial derivatives corresponds to  $\frac{\partial g}{\partial y}$ where $\mathbf X = \mathbf g(\mathbf{Y})$, i.e. these are the derivatives of the transformation: $\mathbf X = g (\mathbf{Y}) = A^{-1}\mathbf{Y}$, or $\mathbf X = B \mathbf{Y}$ where $B = A^{-1}$.\\
 
But if $\mathbf X = B \mathbf{Y}$, then
 
$$X_i = b_{i1}y_1 + b_{i2}y_2 + \cdots b_{ij}y_j\cdots b_{in}y_n$$

So, $ \frac{\partial x_i}{\partial y_j} = b_{ij}$ and thus,
 
 $$ J =\left|\frac{\partial d\mathbf x}{\partial d\mathbf y}\right| = |B| = |A^{-1}| = \frac {1}{|A|}$$
 
 The density of $\mathbf{Y}$ is therefore;
 $$f_Y(\mathbf{y})   =  f_X(g(\mathbf{y})) |J| = f_X(A^{-1}\mathbf{y}) |A^{-1}|$$
 
 \end{xmpl}



{\bf Copyright}
2021, Gunnar Stefansson (editor) with contributions from very many students

This work is licensed under the Creative Commons
Attribution-ShareAlike License. To view a copy of this license, visit
http://creativecommons.org/licenses/by-sa/1.0/ or send a letter to
Creative Commons, 559 Nathan Abbott Way, Stanford, California 94305,
USA.
\clearpage
%% Lecture https://tutor-web.net/math/math612.0/lecture450
\section{The multivariate normal distribution and related topics}
%% Slide https://tutor-web.net/math/math612.0/lecture450/slide10
\subsection{Transformations of random variables}
\begin{fbox}
\begin{minipage}{0.97\textwidth}
Recall that if $X$ is a vector of continuous random variables with a joint probability density function and if $Y=h(X)$ such that $h$ is a 1-1 function and continuously differentiable with inverse $g$ so $X= g(Y)$, then the density of $Y$ is given by 
$$ f_Y(y)=f(g(y))|J|$$
\end{minipage}
\end{fbox}
\subsubsection{Details}
$J$ is the Jacobian determinant of $g$. In particular if $Y=AX$ then 
$$ f_Y(y)=f(A^{-1}y)|det(A^{-1})|$$

if $A$ has an inverse.

%% Slide https://tutor-web.net/math/math612.0/lecture450/slide20
\subsection{The multivariate normal distribution}
\subsubsection{Details}
Consider i.i.d. random variables, $Z_1, \ldots ,Z_n \sim (0,1)$, written $\underline{Z}=\left( \begin{array}{ccc}
   Z_1 \\
  \vdots\\
  Z_n\\
 \end{array} \right)$ and let $\underline{Y}=A \underline{Z} + \underline{\mu}$ where $A$ is an invertible $n x n$ matrix and $\underline{\mu} \in \mathbb{R}^n$ is a vector, so $ Z= A^{-1}(Y-\underline{\mu})$.\\

Then the p.d.f. of $Y$ is given by $$f_{\underline{Y}}(\underline{y})= f_{\underline{Z}}(A^{-1}(\underline{y}- \underline{\mu})) \vert det(A^{-1}) \vert$$

But the joint p.d.f. of $\underline{Z}$ is the product of the p.d.f.'s of $Z_1, \ldots , Z_n$, so $f_{\underline{Z}}(\underline{z})= f(z_1) \cdot f(z_2) \cdot \ldots \cdot f(z_n)$ where

$$f(z_i) = \frac{1}{\sqrt{2 \pi}} e^{-\frac{z^2}{2}}$$

and hence

$$f_{\underline{Z}}(\underline{z}) = \prod_{i=1}^n \frac{1}{\sqrt{2 \pi}} e^{\frac{-z^2}{2}}$$

$$ = (\frac{1}{\sqrt{2 \pi}})^n e^{-\frac{1}{2} \sum_{i=1}^n z_i^2}$$

$$=\frac{1}{(2 \pi)^\frac{n}{2}} e^{-\frac{1}{2} \underline{z}'\underline{z}}$$

since

$$\sum_{i=1}^n z_i^2 = \Vert \underline{z} \Vert ^2 = \underline{z} \cdot \underline{z} = \underline{z}'  \underline{z}$$

The joint p.d.f. of $\underline{Y}$ is therefore 

$$f_{\underline{Y}}(\underline{y}) = f_{\underline{Z}}(A^{-1}(\underline{y} - \underline{\mu})) \vert det(A^{-1}) \vert $$

$$=\frac{1}{(2 \pi)^{\frac{n}{2}}} e^{-\frac{1}{2}(A^{-1}(\underline{y}-\underline{\mu}))'(A^{-1}(\underline{y}-\underline{\mu}))}\frac{1}{\vert det(A)\vert}$$

We can write $det(AA')=det(A)^2$ so $\vert det(A)\vert = \sqrt{det(AA')}$ and if we write $\Sigma=AA'$, then 
$$\vert det(A) \vert = \vert \boldsymbol{\Sigma} \vert ^\frac{1}{2}$$


Also, note that 
$$(A^{-1}(\underline{y}-\underline{\mu}))'(A^{-1}(\underline{y}-\underline{\mu})) = (\underline{y} - \underline{\mu})'(A^{-1})' A^{-1}(\underline{y} - \underline{\mu}) = (\underline{y} - \underline{\mu})' \boldsymbol{\Sigma}^{-1}(\underline{y}-\underline{\mu})$$

We can now write

$$f_{\underline{Y}}(\underline{y}) = \frac{1}{(2 \pi)^\frac{n}{2} \vert \boldsymbol{\Sigma} \vert ^{\frac{1}{2}}} e^{-\frac{1}{2} (\underline{y}-\underline{\mu}) \boldsymbol{\Sigma}^{-1} (\underline{y}-\underline{\mu})}$$

This is the density of the multivariate normal distribution.

Note that 
$$E[\underline{Y}] = \mu$$

$$V[\underline{Y}] = V[A\underline{Z}] = AV[\underline{Z}]A' = AIA' = \boldsymbol{\Sigma}$$

Notation: $\underline{Y}\sim n(\underline{\mu}, \boldsymbol{\Sigma})$

%% Slide https://tutor-web.net/math/math612.0/lecture450/slide31
\subsection{Univariate normal transforms}
\begin{fbox}
\begin{minipage}{0.97\textwidth}
The general univariate normal distribution with density 
$$
f_Y(y) = \frac{1}{\sqrt{2\pi}\sigma}e^{-\frac{(y-\mu)^2}{2\sigma^2}}
$$
is a special case of the multivariate version.
\end{minipage}
\end{fbox}
\subsubsection{Details}
Further, if $Z\sim n(0,1)$, then clearly $X=aZ+\mu \sim n(\mu,\sigma^2)$ where $\sigma^2=a^2$

%% Slide https://tutor-web.net/math/math612.0/lecture450/slide40
\subsection{Transforms to lower dimensions}
\begin{fbox}
\begin{minipage}{0.97\textwidth}
If $Y\sim n \left ( \boldsymbol{\mu},\boldsymbol{\Sigma} \right )$ is a random vector of length $n$ and $A$ is an $m\times n$ matrix of rank $m\leq n$, then $AY \sim n(A\mu,A\Sigma A')$. 
\end{minipage}
\end{fbox}
\subsubsection{Details}
If $Y\sim n \left ( \boldsymbol{\mu},\boldsymbol{\Sigma} \right )$ is a random vector of length $n$ and $A$ is an $m\times n$ matrix of rank $m\leq n$, then $AY \sim n(A\mu,A\Sigma A')$. \\

To prove this, set up an $(n-m)\times n$ matrix, $B$, so that the $n\times n$ matrix, $C$, formed from combining the rows of $A$ and $B$ is of full rank $n$. Then it is easy to derive the density of $CY$ which also factors nicely into a product, only one of which contains $AY$, which gives the density for $AY$.

%% Slide https://tutor-web.net/math/math612.0/lecture450/slide50
\subsection{The OLS estimator}
\begin{fbox}
\begin{minipage}{0.97\textwidth}
Suppose $Y \sim n(X \beta,\sigma^2 I)$. The ordinary least squares estimator, when the $n \times p$ matrix is of full rank, $p$, where $p\leq n$, is:
$$ \hat{\beta} = (X'X)^{-1}X'Y$$
The random variable which describes the process giving the data and estimate is:
$$
b = (X'X)^{-1}X'Y
$$

It follows that
$$\hat{\beta} \sim n(\beta,\sigma^{2}(X'X)^{-1})$$

\end{minipage}
\end{fbox}
\subsubsection{Details}
Suppose $Y \sim n(X \beta,\sigma^2I)$. The ordinary least squares estimator, when the $n \times p$ matrix is of full rank, $p$, is:
$$ \hat{\beta} = (X'X)^{-1}X'Y.$$
The equation below is the random variable which describes the process giving the data and estimate:
$$ b = (X'X)^{-1}X'Y $$
If $B = (X'X)^{-1}X'$, then we know that
$$ BY \sim n(B X \beta, B(\sigma^{2}I)B')$$
Note that 
$$ BX\beta = (X'X)^{-1}X'X\beta=\beta $$
and
$$ B(\sigma^{2}I)B' = \sigma^{}(X'X)^{-1}X'[(X'X)^{-1}X']' $$

$$ = \sigma^{2}(X'X)^{-1}X'X(X'X)^{-1}$$

$$ = \sigma^{2}(X'X)^{-1} $$

It follows that
$$ \hat{\beta} \sim n(\beta,\sigma^{2}(X'X)^{-1}) $$

\begin{notes}
The earlier results regarding the multivariate Gaussian distribution also show that the vector of parameter estimates will be Gaussian even if the original $Y$-variables are not independent.
\end{notes}

{\bf Copyright}
2021, Gunnar Stefansson (editor) with contributions from very many students

This work is licensed under the Creative Commons
Attribution-ShareAlike License. To view a copy of this license, visit
http://creativecommons.org/licenses/by-sa/1.0/ or send a letter to
Creative Commons, 559 Nathan Abbott Way, Stanford, California 94305,
USA.
\clearpage
%% Lecture https://tutor-web.net/math/math612.0/lecture510
\section{Independence, expectations and the moment generating function}
%% Slide https://tutor-web.net/math/math612.0/lecture510/slide10
\subsection{Independent random variables}
\begin{fbox}
\begin{minipage}{0.97\textwidth}
Recall that two events, $A$ and $B$, are independent if,

$$ P [A \cap B] = P[A] P[B] $$

Since the conditional probability of $A$ given $B$ is defined by:

$$ P [A|B] = \frac {P [A \cap B]} {P[B]}$$

We see that A and B are independent if and only if

$$ P[A|B] = P[A](when  P [B] > 0 )$$

 

Two continuous random variables, $X$ and $Y$, are similarly independent if,

$$ P [X \in A, Y \in B] = P [X \in A] P[Y \in B] $$

\end{minipage}
\end{fbox}
\subsubsection{Details}
Two continuous random variables, $X$ and $Y$, are similarly independent if,

$$ P [X \in A, Y \in B] = P [X \in A] P[Y \in B] $$

Now suppose $X$ has p.d.f. $f_X$  and Y has p.d.f. $f_Y$. Then,

$$ P [X \in A] = \int_{A} f_X (x) dx $$

$$ P [Y \in B] = \int_{B} f_Y (y) dy $$

So $X$ and $Y$ are independent if:

$$ P [X \in , Y \in B] = \int_{A} f_X (x) dx \int_{B} f_Y (y) dy $$

$$ = \int_{A}f_X (x) (\int_{B} f_Y (y) dy) dx$$

$$ = \int_{A}\int_{B} f_X (x)f_Y (y) dydx $$

But, if $f$ is the joint density of $X$ and $Y$ then we know that

$$ P [X \in A, Y \in B] $$

$$ \int_{A}\int_{B} f (x,y) dydx $$

Hence $X$ and $Y$ are independent if and only if we can write the joint density in the form of,

$$ f(x ,y) = f_X (x)f_Y (y) $$


%% Slide https://tutor-web.net/math/math612.0/lecture510/slide20
\subsection{Independence and expected values}
\begin{fbox}
\begin{minipage}{0.97\textwidth}
If $X$ and $Y$ are independent random variables then $E[XY]=E[X]E[Y]$.\\

Further, if $X$ and $Y$ are independent random variables then $E[g(X)h(Y)]=E[g(X)]E[h(Y)]$ is true if $g$ and $h$ are functions in which expectations exist.
\end{minipage}
\end{fbox}
\subsubsection{Details}
If $X$ and $Y$ are random variables with a joint distribution function $f(x,y)$, then it is true that for $h:\mathbb{R}^2\to\mathbb{R}$ we have 
$$E[h(X,Y)]=\int\int h(x,y)f(x,y)dxdy$$

for those $h$ such that the integral on the right exists. \\

Suppose $X$ and $Y$ are independent continuous r.v., then

$$ f(x,y) = f_X (x) f_Y (y)$$

Thus, $$E[XY] = \int\int xy f (x,y) dxdy $$

$$= \int\int xy f_X (x) f_Y (y) dxdy $$

$$ = \int xf_X (x) dx \int yf_Y (y) dy $$

$$ = E [X] E [Y] $$

\begin{notes}
Note that if $X$ and $Y$ are independent then $ E [h (X)g (Y)] = E [h(X)] E[g(Y)]$ is true whenever the functions $h$ and $g$ have expected values.
\end{notes}
\subsubsection{Examples}
\begin{xmpl}
Suppose $X,Y \in U (0,2)$ are  i.i.d then,

$$
f_X (x) = 
\begin{cases}
  \frac{1}{2} & \text{if } 0 \leq x \leq 2 \\
  0 &\text{otherwise}
\end{cases}.
$$

and similarly for $f_Y$.\\

Next, note that,
$$
f(x,y) =  f_X (x) f_Y (y) = 
\begin{cases}
  \frac{1}{4} &\text{if } 0 \leq x,y \leq 2\\
  0 &\text{otherwise}
\end{cases}.
$$

Also note that $f(x,y) \geq 0$ for all $ (x,y) \in \mathbb{R}^2 $ and 

$$\int\int  f(x,y)dxdy = \int_{0}^{2}\int_{0}^{2} \frac {1}{4} dxdy =  \frac {1}{4}.4 = 1$$

It follows that,
$$ E [X Y] = \int_{-\infty}^{\infty}\int_{-\infty}^{\infty} xy f(x,y) dxdy$$

$$ = \int_{y=0}^{2}\int_{x=0}^{2} xy. \frac {1}{4}dxdy $$

$$ = \int_{y=0}^{2} (\int_{x=0}^{2} xy \frac {1}{4} dx) dy$$

$$= \int_{y=0}^{2} [\frac {1}{4}y. \frac {1}{2}x^2]_{x=0}^ {2} dy $$

 

$$ = \int_{y=0}^{2} \frac {1}{4}y (\frac {1}{2}.2^2 - \frac {1}{2}.0) dy$$

$$ \int_{0}^{2} \frac {2}{4}y dy = \int_{0}^{2} \frac {1}{2}y dy = \frac {1}{2}\cdot \frac {1}{2} y^2 | _{0}^2 = \frac {1}{4}\cdot 2^2 = 1 $$

But $$ E [X] = E[Y] = \int_{y=0}^{2} x. \frac {1}{2} dx = 1$$

So $$E[XY] = E [X] E[Y] $$

\end{xmpl}




%% Slide https://tutor-web.net/math/math612.0/lecture510/slide30
\subsection{Independence and the covariance}
\begin{fbox}
\begin{minipage}{0.97\textwidth}
If $X$ and $Y$ are independent then $Cov(X,Y)=0$.\\

In fact, if $X$ and $Y$ are independent then $Cov(h(X),g(Y))=0$ for any functions $g$ and $h$ in which expected values exist.

\end{minipage}
\end{fbox}

%% Slide https://tutor-web.net/math/math612.0/lecture510/slide40
\subsection{The moment generating function}
\begin{fbox}
\begin{minipage}{0.97\textwidth}
If $X$ is a random variable we define the moment generating function when $t$ exists as: $M(t):=E(e^{tX})$.
\end{minipage}
\end{fbox}
\subsubsection{Examples}
\begin{xmpl}
If $X\sim b(n,p)$ then $M(t)=\displaystyle\sum_{x=0}^{n} e^{tx}p(x) = \displaystyle\sum_{x=0}^{n} e^{tx} \binom{n}{x}p\cdot (1-p)^{n-x}$ 
\end{xmpl}

%% Slide https://tutor-web.net/math/math612.0/lecture510/slide50
\subsection{Moments and the moment generating function}
\begin{fbox}
\begin{minipage}{0.97\textwidth}
If $M_{X}(t)$ is the moment generating function (mgf) of $X$, then $M_{X}^{(n)}(0)=E[X^n]$.
\end{minipage}
\end{fbox}
\subsubsection{Details}
Observe that $M(t)=E[e^{tX}]=E[1+X+\frac{(tX)^2}{2!}+\frac{(tX)^3}{3!}+\dots]$ since $e^a=1+a+\frac{a^2}{2!}+\frac{a^3}{3!}+\dots$. If the random variable $e^{|tX|}$ has a finite expected value then we can switch the sum and the expected valued to obtain:
$$M(t)=E[\sum_{n=0}^{\infty}\frac{(tX)^n}{n!}]=\sum_{n=0}^{\infty}\frac{E[(tX)^n]}{n!}=\sum_{n=0}^{\infty}t^n\frac{E[X^n]}{n!}$$
This implies that the $n^{th}$ derivative of $M(t)$ evaluated at $t=0$ is exactly $E[X^n]$

%% Slide https://tutor-web.net/math/math612.0/lecture510/slide60
\subsection{The moment generating function of a sum of random variables}
\begin{fbox}
\begin{minipage}{0.97\textwidth}
$M_{X+Y}(t)=M_{X}(t)\cdot M_{Y}(t)$ if $X$ and $Y$ are independent.
\end{minipage}
\end{fbox}
\subsubsection{Details}
Let $X$ and $Y$ be independent random vaiables, then
$$M_{X+Y}(t)=E[e^{Xt+Yt}]=E[e^{Xt}e^{Xt}]=E[e^{Xt}]E[e^{Xt}]=M_{X}(t)M_{Y}(t)$$

%% Slide https://tutor-web.net/math/math612.0/lecture510/slide70
\subsection{Uniqueness of the moment generating function}
\begin{fbox}
\begin{minipage}{0.97\textwidth}
Moment generating functions (m.g.f.) uniquely determine the probability distribution function for random variables.  Thus, if two random variables have the same m.g.f, then they must also have the same distribution.

\end{minipage}
\end{fbox}

{\bf Copyright}
2021, Gunnar Stefansson (editor) with contributions from very many students

This work is licensed under the Creative Commons
Attribution-ShareAlike License. To view a copy of this license, visit
http://creativecommons.org/licenses/by-sa/1.0/ or send a letter to
Creative Commons, 559 Nathan Abbott Way, Stanford, California 94305,
USA.
\clearpage
%% Lecture https://tutor-web.net/math/math612.0/lecture520
\section{The gamma distribution}
%% Slide https://tutor-web.net/math/math612.0/lecture520/slide10
\subsection{The gamma distribution}
\begin{fbox}
\begin{minipage}{0.97\textwidth}
If a random variable $X$ has the density 

$$f(x) = \frac{x^{\alpha-1} e^{\frac{-x} {\beta}}} {\Gamma(\alpha) \beta^{\alpha}}$$

where $x>0$ for some constants $\alpha$, $\beta>0$, then $X$ is said to have a gamma distribution.
\end{minipage}
\end{fbox}
\subsubsection{Details}
The function $\Gamma$ is basically chosen so that $f$ integrates to one, i.e.


$$\Gamma(\alpha) = \int_0^\infty t^{\alpha-1} e^{-t}dt$$

It is not too hard to see that $\Gamma(n)=(n-1)!$ if $n \in \mathbb{N}$. Also, $\Gamma(\alpha + 1) = \alpha \Gamma(\alpha)$ for all $\alpha >0$.


%% Slide https://tutor-web.net/math/math612.0/lecture520/slide20
\subsection{The mean, variance and mgf of the gamma distribution}
\begin{fbox}
\begin{minipage}{0.97\textwidth}
Suppose $X \sim G (\alpha, \beta)$ i.e. $X$ has density

$$ f(x) = \frac{x^{\alpha -1} e^{-x/\beta}} {\Gamma (\alpha) \beta^{\alpha}} , x > 0  $$

Then,
$$E[X] = \alpha\beta$$

$$M(t) = (1-\beta t)^{-\alpha}$$ $$

V[X] = \alpha \beta^2  $$

\end{minipage}
\end{fbox}
\subsubsection{Details}
The expected value of $X$ can be computed as follows:

\begin{eqnarray*}
E[X] & = & \int_{-\infty}^{\infty} xf(x)dx \\
 & = & \int_{0}^{\infty} x \frac{x^{\alpha -1} e^{-x/\beta}} {\Gamma (\alpha) \beta^{\alpha}} dx \\
 & = & \frac{\Gamma(\alpha+1)\beta^{\alpha+1}}{\Gamma(\alpha)\beta^{\alpha}} \int_{0}^{\infty}  \frac{x^{(\alpha+1) -1} e^{-x/\beta}} {\Gamma (\alpha+1) \beta^{\alpha+1}} dx\\
 & = & \frac{\alpha\Gamma(\alpha)\beta^{\alpha+1}}{\Gamma(\alpha)\beta^{\alpha}} 
\end{eqnarray*} 

so $E[X] = \alpha\beta$. \\

Next, the m.g.f.is given by 

\begin{eqnarray*}
E[e^{tX}] & = & \int_{0}^{\infty} e^{tx}  
                \frac{x^{\alpha-1}e^{-x/\beta}}
                     {\Gamma(\alpha)\beta^{\alpha}} 
                dx \\
& = & \frac{1}{\Gamma(\alpha)\beta^{\alpha}} 
\int_{0}^{\infty} x^{\alpha-1} e^{tx - x/\beta} dx \\
& = &  \frac{\Gamma(\alpha) \phi^{\alpha} }
            {\Gamma(\alpha) \beta^{\alpha}} 
\int_{0}^{\infty} \frac{x^{(\alpha-1)} e^{-x/\phi}} {\Gamma (\alpha) \phi^{\alpha}}dx 
\end{eqnarray*} 


if we choose $\phi$ so that $\frac{-x}{\phi} = tx - x/\beta$ i.e. $\frac{-1}{\phi} = t - \frac{1}{\beta}$ 
i.e. $\phi = - \frac{1}{t-1/\beta} = \frac{\beta}{1 - \beta t}$
then we have 

\begin{eqnarray*}
M(t) & = & \left(\frac{\phi}{\beta}\right)^{\alpha} \\
& = & \left(\frac{\beta / (1-\beta t)}{\beta}\right)^{\alpha} \\
& = & \frac{1}{(1-\beta t)^{\alpha} }
\end{eqnarray*} 

or $M(t) = (1-\beta t)^{-\alpha}$.
It follows that

$$ M'(t) = (-\alpha) (1-\beta t)^{-\alpha-1} (-\beta) = \alpha\beta(1-\beta t)^{-\alpha-1}$$

so $ M'(0) = \alpha\beta $. Further, 

\begin{eqnarray*}
M''(t) & = & \alpha\beta (-\alpha-1)(1-\beta t)^{-\alpha-2} (-\beta) \\
& = & \alpha\beta^2 (\alpha+1)(1-\beta t)^{-\alpha-2} 
\end{eqnarray*} 

\begin{eqnarray*}
E[X^2] & = & M''(0) \\
& = & \alpha\beta^2 (\alpha+1) \\
& = & \alpha^2 \beta^2 + \alpha \beta^2
\end{eqnarray*} 

Hence, 

\begin{eqnarray*}
V[X] & = & E[X]^2 - E[X]^2\\
& = & \alpha^2 \beta^2 + \alpha \beta^2 - (\alpha\beta)^2 \\
& = & \alpha \beta^2
\end{eqnarray*} 



%% Slide https://tutor-web.net/math/math612.0/lecture520/slide30
\subsection{Special cases of the gamma distribution: The exponential and chi-squared distributions}
\begin{fbox}
\begin{minipage}{0.97\textwidth}
Consider the gamma density,


$$ f(x) = \frac {x^{\alpha - 1} e^\frac{-x}{\beta}} {\Gamma(\alpha) \beta^{\alpha}} , x > 0$$

For parameters $\alpha, \beta > 0 $.\\

If $\alpha = 1 $ then

$$f(x) = \frac {1} {\beta} e^\frac{-x}{\beta}, x > 0$$

and this is the density of exponential distribution.\\

Consider next the case $ \alpha = \frac{v}{2}$ and $ \beta = 2$ where $v$ is an integer, so the density becomes,

 $$ f(x) = \frac {x^ {\frac{v}{2}- 1} e^\frac{-x}{2}} {\Gamma (\frac{v}{2}) Z^ \frac{v}{2}},  x > 0$$
 
This is the density of a chi-squared random variable with $v$ degrees of freedom.

\end{minipage}
\end{fbox}
\subsubsection{Details}
Consider, $ \alpha = \frac{v}{2}$ and $ \beta = 2$ where $v$ is an integer, so the density becomes,

 $$ f(x) = \frac {x^ {\frac{v}{2}- 1} e^\frac{-x}{2}} {\Gamma (\frac{v}{2}) Z^ \frac{v}{2}},  x > 0$$
 
This is the density of a chi - squared random variable with $v$ degrees of freedom.\\

This is easy to see by starting with $ Z \sim n(0,1)$ and defining $ W = Z^2$ so that the c.d.f. is:
 
$$H _(w) = P [W \leq w] = P [ Z^2 \leq w]$$

$$ = P [ - \sqrt{w}\leq Z \leq \sqrt{w}]$$

$$ = 1 - P [|Z| > \sqrt{w}]$$
 
$$ = 1-2p [Z< - \sqrt{w}]$$

$$ = 1 - 2  \int_{-\alpha}^{\sqrt{w}} \frac{e \frac{-t^2}{2}} {\sqrt{2w}} dt = 1 - 2\phi (\sqrt{w})$$
 

The p.d.f. of $w$ is therefore,

$$h(w) = H ^\prime(w)$$

$$ = 0 - 2\phi ^\prime (\sqrt{w}) \frac{1} {2} w ^ {\frac{1} {2} -1}$$

but 

$$ \phi (x) = \int_{-\alpha}^{x} \frac{e \frac{-t^2}{2}} {2\Pi} dt ; \phi ^\prime (x) = \frac {d}{dx}\int_{\alpha}^{x}\frac{e \frac{-t^2}{2}} {2\Pi} dt = \frac{e \frac{-x^2}{2}} {2\Pi}$$

So 

$$ h[w] =  -2 \frac{e \frac{-w}{2}} {2\Pi}. \frac {1} {2} . w^{\frac {1}{2} -1} $$

 

$$ h[w] = \frac{w^ {\frac{-1}{2}-1} e \frac{-w}{2}} {2\Pi}, w > 0$$

We see that we must have $ h=f $ with $v = 1 $. We have also shown $ \Gamma (\frac {1}{2}) 2 ^\frac {1}{2} = \sqrt{2\Pi}$, i.e $ \Gamma (\frac {1}{2}) = \sqrt{\Pi}$. Hence we have shown the $\chi^2$ distribution on 1 df to be $G (\alpha = \frac {v}{2}, \beta = 2)$  when v = 1.


 


%% Slide https://tutor-web.net/math/math612.0/lecture520/slide40
\subsection{The sum of gamma variables}
\begin{fbox}
\begin{minipage}{0.97\textwidth}
In the general case if $ X_1 \ldots X_n \sim G (\alpha, \beta)$ are i.i.d. then $ X_1 + X_2 + \ldots X_n \sim G (n\alpha, \beta)$.\\
 
In particular, if $ X_1, X_2 , \ldots, X_v \sim \chi^2$ i.i.d. then $ \sum_{i=1}^v X_i \sim \chi^2_{v}$.

\end{minipage}
\end{fbox}
\subsubsection{Details}
If $X$ and $Y$ are i.i.d. $G (\alpha, \beta)$, then
 
$$ M_X (t) = M_Y (t) = \frac {1} {(1- \beta t)^\alpha}$$

and 
 
$$ M_{X+Y} (t) = M_X (t) M_Y (t) = \frac {1} {(1- \beta t)^{2 \alpha}}$$
 
So $$ X + Y \sim G (2\alpha, \beta)$$
 
In the general case if $ X_1 \ldots X_n \sim G (\alpha, \beta)$ are i.i.d. then $ X_1 + X_2 + \ldots X_n \sim G (n\alpha, \beta)$. In particular, if $ X_1, X_2 , \ldots, X_v \sim \chi^2$ i.i.d., then $ \sum_{i=1}^v X_i \sim \chi^2_{v}$


{\bf Copyright}
2021, Gunnar Stefansson (editor) with contributions from very many students

This work is licensed under the Creative Commons
Attribution-ShareAlike License. To view a copy of this license, visit
http://creativecommons.org/licenses/by-sa/1.0/ or send a letter to
Creative Commons, 559 Nathan Abbott Way, Stanford, California 94305,
USA.
\clearpage
%% Lecture https://tutor-web.net/math/math612.0/lecture530
\section{Notes and examples: The linear model}
%% Slide https://tutor-web.net/math/math612.0/lecture530/slide10
\subsection{Simple linear regression in R}
\begin{fbox}
\begin{minipage}{0.58\textwidth}
To test the effect of one variable on another, simple linear regression may be applied.  The fitted model may be expressed as $y=\alpha + \hat{\beta} x,$ where $\alpha$ is a constant, $\hat{\beta}$ is the estimated coefficient, and $x$ is the explanatory variable.
\end{minipage}
\hspace{0.5mm}
\begin{minipage}{0.38\textwidth}
\begin{picture}
40
\end{picture}

Figure:  Example taken from R of a fitted model using linear regression.
\end{minipage}
\end{fbox}
\subsubsection{Details}
Below is the linear regression output using the R's data set "car". Notice that the output from the model may be divided into two main categories:
\begin{enumerate}
\item output that assesses the model as a whole, and
\item output that relates to the estimated coefficients for the model 
\end{enumerate}

\begin{lstlisting}
Call:
lm(formula = dist ~ speed, data = cars)

Residuals:
    Min      1Q  Median      3Q     Max 
-29.069  -9.525  -2.272   9.215  43.201 

Coefficients:
            Estimate Std. Error t value Pr(>|t|)    
(Intercept) -17.5791     6.7584  -2.601   0.0123 *  
speed         3.9324     0.4155   9.464 1.49e-12 ***
---

Residual standard error: 15.38 on 48 degrees of freedom
Multiple R-squared: 0.6511,     Adjusted R-squared: 0.6438 
F-statistic: 89.57 on 1 and 48 DF,  p-value: 1.490e-12 
\end{lstlisting}

Notice that there are four different sets of output ({\texttt Call, Residuals, Coefficients}, and {\texttt Results}) for both the constant $\alpha$ and the estimated coefficient $\hat{\beta}$ speed variable.\\

The estimated coefficients describe the change in the dependent variable when there is a single unit increase in the explanatory variable given that everything else is held constant. \\ 

The standard error is a measure of accuracy and is used to construct the confidence interval.  Confidence intervals provide a range of values for which there is a set level of confidence that the true population mean will be within the given range. For example, if the CI is set at 95\% percent then the probability of observing a value outside the given CI range is less than 0.05.\\

The p-value is represented as a percentage.  Specifically, the p-value indicates the percentage of time, given that your null hypothesis is true, that you would find an outcome at least as extreme as the observed value. If your calculated p-value is 0.02 then 2% of the time you'd observe a mean at least as large as your observed. \\ 

In the overall model assessment the R-squared is the explained variance over the total variance.  Generally, a higher $R^2$ is better but data with very little variance makes it easy to achieve a higher $R^2$, which is why the adjusted $R^2$ is presented.\\  

Lastly, the F-statistic is given.  Since the t-Statistic is not appropriate to compare two or more coefficients, the F-statistic must be applied.  The basic methodology is that it compares a restricted model where the coefficients have been set to a certain fixed level to a model which is unrestricted.  The most common is the sum of squared residuals F-test.

  

 

%% Slide https://tutor-web.net/math/math612.0/lecture530/slide20
\subsection{Multiple linear regression}
\begin{fbox}
\begin{minipage}{0.97\textwidth}
Multiple linear regression attempts to model the relationship between two or more explanatory variables and a response variable by fitting a linear equation to observed data. Formally, the model for multiple linear regression, given $n$ observations, is

$y_i = \beta_1 x_{i,1} + \beta_2 x_{i,2} + \ldots + \beta_p x_{i,p} + e_i$ for $i = 1,2, \ldots , n$


% this was all messed up
%The definition above was taken from: http://www.stat.yale.edu/Courses/1997-98/101/linmult.htm
%the dataset from this source%
%website doesn't work and it causes the PDF not to compile%
%http://jamsb.austms.org.au/staff/dunn/Datasets/tech-regression.html#multiple%

As always, we view the data, $y_i$ as observations of random variables, so another way to describe the same model is

$Y_i = \beta_1 x_{i,1} + \beta_2 x_{i,2} + \ldots + \beta_p x_{i,p} + \epsilon_i$ for $i = 1,2, \ldots , n$,


and we note that the $x$-values are just numbers and are usually assumed to be without any measurement error.



\end{minipage}
\end{fbox}

%% Slide https://tutor-web.net/math/math612.0/lecture530/slide30
\subsection{The one-way model}
\begin{fbox}
\begin{minipage}{0.97\textwidth}
The one-way ANOVA model is of the form:
$$Y_{ij}=\mu_i+\epsilon_{ij}$$
or 
$$Y_{ij}=\mu+\alpha_i+\epsilon_{ij}$$
\end{minipage}
\end{fbox}
\subsubsection{Details}
The one-way ANOVA model is of the form:
$$Y_{ij}=\mu_i+\epsilon_{ij}$$
where $Y_{ij}$ is observation $j$ in treatment group $i$ and $\mu_i$ are the parameters of the model and are means of treatment group $i$. The $\epsilon_{ij}$ are independent and follow a normal distribution with mean zero and constant variance $\sigma^2$ often written as $\epsilon\sim N(0,\sigma^2)$.\\

The ANOVA model can also be written in the form:
$$Y_{ij}=\mu+\alpha_i+\epsilon_{ij}$$
where $\mu$ is the overall mean of all treatment groups and $\alpha_i$ is the deviation of mean of treatment group $i$ from the overall mean. The $\epsilon_{ij}$ follow a normal distribution as before.\\

The expected value of $Y_{ij}$ is $\mu_i$ as the expected value of the errors is zero, often written as $E[Y_{ij}]=\mu_i$.
\subsubsection{Examples}
\begin{xmpl}
In the rat diet experiment the model would be of the form:
$$y_{ij}=\mu_i+\epsilon_{ij}$$
where $y_{ij}$ is the weight gain for rat $j$ in diet group $i$, $\mu_i$ would be the mean weight gain in diet group $i$ and $\epsilon{ij}$ would be the deviation of rat $j$ from the mean of its diet group.
\end{xmpl}

%% Slide https://tutor-web.net/math/math612.0/lecture530/slide50
\subsection{Random effects in the one-way layout}
\begin{fbox}
\begin{minipage}{0.97\textwidth}
The simplest random effects model is the one-way layout, commonly written in the form
$$y_{ij}=\mu + \alpha_i + \epsilon_{ij},$$
where $j =1,\ldots,J$ and $i =1,\ldots,I$.

Normally one also assumes $\epsilon_{ij}\sim n (0,\sigma_A^2)$, $\alpha_i \sim n (0,\sigma_A^2)$, and that all these random variables are independent.

Note that we have stopped making a distinction in notation between random variables and measurements (the $y$-values are just random variables when distributions occur).

\end{minipage}
\end{fbox}
\subsubsection{Details}
Note that this is considerably different from the fixed effect model.

Since the factor has changed to a random variable with an expected value of zero, the expected value of all the $y$ is the same:
$$Ey_{ij}=\mu .$$

The variance of $y$ now has two components:
$$Vy_{ij}=\sigma^2_A + \sigma^2.$$


In addition we have a covariance structure between the measurements and this needs to be looked at in some detail. First, the general case of a covariance between two general $y_{ij}$ and $y_{i'j'}$, where the indices may or may not be the same:

\begin{eqnarray*}
cov(y_{ij},y_{i'j'}) &=cov(\alpha_i+\epsilon_{ij}, \alpha_{i'}+ \epsilon_{i'j'})\\
                     &=E[(\alpha_i+\epsilon_{ij})(\alpha_{i'}+\epsilon_{i'j'})]\\
                     &=E[\alpha_i\alpha_{i'}] + E[\epsilon{ij}\alpha_{i'}]+ E[\alpha_i\epsilon_{i'j'}] + E[\epsilon_{ij}\epsilon_{i'j'}]
\end{eqnarray*}

\begin{notes}
Recall that $E[UW]=E[U]E[W]$ if $U,W$ are independent
\end{notes}

So,
$$E[\epsilon_{ij}\alpha_{i'}]=E[\alpha_i\epsilon_{i'j'}] = E\alpha_iE\epsilon_{i'j'}=0 .$$

Further,
$$E[\epsilon_{ij}\epsilon{i'j'}] =
  \begin{cases}
  \sigma^2 & \text{if } i=i', j=j'\\
  0 &\text{otherwise} 
\end{cases}$$

and

$$ 
E[\alpha_{i}\alpha{i'}] =
\begin{cases}
\sigma^2_A &\text{if } i=i'\\
0 &\text{if }i \neq i' 
\end{cases}
$$

so

$$ Cov(y_{ij},y_{i'j'}) = 
\begin{cases}
  \sigma_A^2+\sigma^2 & \text{if } i=i', j=j'\\
  \sigma'_A & \text{if } i=i', j \neq j'\\
  0 & \text{otherwise}
\end{cases}. $$

It follows that the correlation between measurements $y_{ij}$ and $y_{ij'}$ (within the same group) are

$$
\begin{align*}
Cor(y_{ij},y_{ij'}) &= \frac{Cov(y_{ij},y_{ij'})}{\sqrt{v[y_{ij}]v[y_{ij'}]}}\\
                   &= \frac{\sigma_A^2}{\sqrt{(\sigma_A^2 + \sigma^2)^2}}\\
                   &\Rightarrow Cor(y_{ij}, y_{ij'}) = \frac{\sigma_A^2}{\sigma_A^2 + \sigma^2}
\end{align*}.
$$

This is the intra-class correlation.



%% Slide https://tutor-web.net/math/math612.0/lecture530/slide60
\subsection{Linear mixed effects models (lmm)}
\begin{fbox}
\begin{minipage}{0.97\textwidth}
The simplest mixed effects model is

$$y_{ij} = \mu + \alpha_i + \beta_j + \epsilon_{ij}$$

where $\mu, \alpha_1, \alpha_2, \ldots, \alpha_i$ are unknown constants, 

$\beta_j \sim n(0,\sigma^2_\beta)$

$\epsilon_{ij} \sim n(0,\sigma^2)$

($\beta_j$ and $\epsilon_{ij}$ independent).
\end{minipage}
\end{fbox}
\subsubsection{Details}
The $\mu$ and $\alpha_i$ are the fixed effects and $\beta_j$ is the random effects.\\

Recall that in the simple one-way layout with $y_{ij} = \mu + \alpha_i + \epsilon_{ij}$, we can write the model in matrix form
$\underline{y} = X \underline{\beta} + \underline{\epsilon}$ where $\underline{\beta} = (\mu, \alpha_1, \ldots, \alpha_I)'$ and $X$ is appropriately chosen. \\

The same applies to the simplest random effects model $y_{ij}= \mu + \beta_j+ \epsilon_{ij}$ where we can write $\underline{y} = \mu \cdot \underline{1}+ Z \underline{U} + \underline{\epsilon}$ where $\underline{1}=(1,1, \ldots , 1)'$, $\underline{U} = ( \beta_1 , \ldots , \beta_J )'$.\\

In general, we write the mixed effects models in matrix form with $\underline{y} = X \underline{\beta} + Z \underline{U} + \underline {\epsilon}$, where $\underline{\beta}$ contains the fixed effects and $\underline{U}$ contains the random effects.
\subsubsection{Examples}
\begin{xmpl}
\begin{enumerate}
\item $y_i = \beta_1 + \beta_2 x_i + \epsilon_i$ (SLR)
\item $y_{ij} = \mu + \alpha_i + \beta_i x_{ij} + \epsilon_{ij}$ only fixed effects (ANCOVA)
\item $y_{ijk} = \mu + \alpha_i + b_j + \epsilon{ijk}$ where $\alpha_i$ are fixed but $b_j$ are random.
\item $y_{ijk} = \mu  + \alpha_i + b_j x_{ij} + \epsilon_{ijk}$ where $\alpha_i$ are fixed but $b_j$ are random slopes.
\end{enumerate}
\end{xmpl}

%% Slide https://tutor-web.net/math/math612.0/lecture530/slide70
\subsection{Maximum likelihood estimation in lmm}
\begin{fbox}
\begin{minipage}{0.97\textwidth}
The likelihood function for the unknown parameters $L(\boldsymbol{\beta},\sigma^2_A, \sigma^2)$ is
$$ \frac{1}{(2\pi)^{n/2} \left| \boldsymbol{\Sigma}_y \right| ^{n/2}} 
e^{-1/2 (\mathbf{y}-X\boldsymbol{\beta})' \boldsymbol{\Sigma}^{-1}_y (\mathbf{y}-X\boldsymbol{\beta})}$$

where $\boldsymbol{\Sigma}_y =  \sigma^2_A Z Z' + \sigma^2 I$.\\

Maximising $L$ over $\boldsymbol{\beta},\sigma^2_A, \sigma^2$ gives the variance components and the fixed effects. May also need $\mathbf{\hat{u}}$, this is normally done using BLUP.
\end{minipage}
\end{fbox}
\subsubsection{Details}
Recall that if $W$ is a random variable vector with $EW = \mu$ and $VW= \boldsymbol{\Sigma}$ then 
$$E[AW] = A\mathbf{\mu}$$

 
$$V[AW]= A \boldsymbol{\Sigma} A'$$

In particular, if $W \sim n(\mu, \boldsymbol{\Sigma}($ then $AW \sim n(A\mu, A \boldsymbol{\Sigma} A')$.\\

Now consider the lmm with 

$$y = X \boldsymbol{\beta} + Zu + \boldsymbol{\epsilon}$$
where
 $$u = (u_1 , \ldots , u_m)' $$

 
 $$\boldsymbol{\epsilon} = (\epsilon_1 , \ldots , \epsilon_m)'$$
and the random variables  $U_i \sim n(0, \sigma^2_A)$, $\epsilon_i \sim n(0, \sigma^2)$ are all independent so that $u \sim n(0, \sigma^2_A I)$ and $\boldsymbol{\epsilon} \sim n(\mathbf{0}, \sigma^2 I)$.

Then $Ey = X\boldsymbol{\beta}$ and 
\begin{eqnarray*}
Vy& = & \boldsymbol{\Sigma}_y \\
& = & V[Zu+V[\boldsymbol{\epsilon}] \\
& = & Z(\sigma^2_A I) Z' + \sigma^2 I\\
& = & \sigma^2_A Z Z' + \sigma^2 I
\end{eqnarray*}
and hence $y \sim n(X\boldsymbol{\beta},\sigma^2_A Z Z' + \sigma^2 I )$.\\

Therefore the likelihood function for the unknown parameters $L(\boldsymbol{\beta},\sigma^2_A, \sigma^2)$ is
$$ = \frac{1}{(2\pi)^{n/2} \left| \boldsymbol{\Sigma}_y \right| ^{n/2}} e^{-1/2 (\mathbf{y}-X\boldsymbol{\beta})' \boldsymbol{\Sigma}^{-1}_y (y}-X\boldsymbol{\beta})$$

where $\boldsymbol{\Sigma}_y =  \sigma^2_A Z Z' + \sigma^2 I$. Maximizing $L$ over $\boldsymbol{\beta},\sigma^2_A, \sigma^2$ gives the variance components and the fixed effects. May also need $\hat{u}$, which is normally done using BLUP.


{\bf Copyright}
2022, Gunnar Stefansson (editor) with contributions from very many students

This work is licensed under the Creative Commons
Attribution-ShareAlike License. To view a copy of this license, visit
http://creativecommons.org/licenses/by-sa/1.0/ or send a letter to
Creative Commons, 559 Nathan Abbott Way, Stanford, California 94305,
USA.
\clearpage
%% Lecture https://tutor-web.net/math/math612.0/lecture540
\section{Some regression topics}
%% Slide https://tutor-web.net/math/math612.0/lecture540/slide10
\subsection{Poisson regression}
\begin{fbox}
\begin{minipage}{0.97\textwidth}
Data $y_i$ are from a Poisson distribution with mean $\mu_i$ and $\ln{\mu_i}=\beta_1+\beta_2 x_i$. A likelihood function can be written and the parameters can be estimated using maximum likelihood.



\end{minipage}
\end{fbox}

%% Slide https://tutor-web.net/math/math612.0/lecture540/slide20
\subsection{The generalized linear model (GLM)}
\begin{fbox}
\begin{minipage}{0.97\textwidth}
Data $y_i$ are from a distribution within the exponential family, with mean $\mu_i$ and $g(\mu_i)=\textbf{x}'_i\boldsymbol{\beta}$ for some link function, $g$. A likelihood function can now be written and the parameters can be estimated using maximum likelihood.



\end{minipage}
\end{fbox}
\subsubsection{Details}
Data $y_i$ are from a distribution within the exponential family, with mean $\mu_i$ and $g(\mu_i)=\textbf{x}'_i\boldsymbol{\beta}$ for some link function, $g$.\\

The exponential family includes distributions such as the Gaussian, binomial, Poisson, and gamma (and thus exponential and chi-squared).\\  

The link functions are typically 
\begin{itemize}
\item identity (with the Gaussian)
\item log (with the Poisson and the gamma)
\item logistic (with the binomial)
\end{itemize}

A likelihood function can be set up for each of these models and the parameters can be estimated using maximum likelihood.

The glm package in R has options to estimate parameters in these models.




{\bf Copyright}
2022, Gunnar Stefansson (editor) with contributions from very many students

This work is licensed under the Creative Commons
Attribution-ShareAlike License. To view a copy of this license, visit
http://creativecommons.org/licenses/by-sa/1.0/ or send a letter to
Creative Commons, 559 Nathan Abbott Way, Stanford, California 94305,
USA.
\clearpage
%% Lecture https://tutor-web.net/math/math612.0/lecture990
\section{Overview drills}
{\bf Copyright}
2021, Gunnar Stefansson (editor) with contributions from very many students

This work is licensed under the Creative Commons
Attribution-ShareAlike License. To view a copy of this license, visit
http://creativecommons.org/licenses/by-sa/1.0/ or send a letter to
Creative Commons, 559 Nathan Abbott Way, Stanford, California 94305,
USA.
\clearpage
\end{document}